%!TEX root = cscw2019-comic.tex
\section{Conclusion}
\label{sec:Conclusion}

Inspired by a rich history in persuasive test messages construction and benefits of abstract comics in communication, this paper examined if the abstract comic form is more persuasive than the corresponding plain text in the context of encouraging participants to donate for charitable causes. We conducted a field study on Amazon Mechanical Turk with 307 participants. In the study, participants received one of the three persuasive messages designed to ask for a donation to the Organization of Autism Research, a persuasive text message, a three-panel comic strip, and a three-panel comic strip incorporating the idea of social proof, and made donation decisions with real money. We analyzed the results using a hierarchical Bayesian framework that allows for understanding effect sizes, as well are transparent and helpful in small-$n$ studies. The results shows convincingly that the three-panel abstract comic is more persuasive than the text (a medium to large effect size = $0.59$). We also show that while the comic with social proof increases the donation level over the comic without the norm, the effect size is very small ($0.11$) and the increase is not significant. To summarize, the comic form significantly increases donations over the plain text, but the presence of the norm is not effective. We caution that the result holds for single-shot, public goods tasks; the value of the social proof in the comic, for exclusive tasks with distant rewards such as exercise, or dieting needs future research. The main implication of our work is that non-profits and Governmental agencies ought to consider using abstract comic in their online campaign as they work to alleviate public goods dilemmas. We believe that these agencies can easily include the use of the comic form as part of their overall messaging strategy because the simplicity of the abstract comic form allows it to be easily synthesized and to additionally incorporate social proofs. 

As next steps, we plan to develop an algorithmic framework that automatically maps a person's behavioral data (e.g. amount walked this week) to a three-panel persuasive comic. We also plan to conduct longitudinal field experiments with an emphasis on storytelling where individuals receive three-panel comics over time, and comics are connected with a storyline.
