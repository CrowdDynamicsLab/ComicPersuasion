%!TEX root = cscw2019-comic.tex
\section{Method}
\label{sec:Method}
In this study, we examined if the abstract comic can impact a decision with monetary consequence on Amazon Mechanical Turk. The persuasion task is to ask participants to make a charitable donation decision to the Organization for Autism Research with the real money. 

In this section, we will first discuss the choice of task, then show how we construct persuasive messages (\Cref{sub:Constructing Persuasive Messages}). Then, we introduce our experimental design (\Cref{sub:Experiment Design}), discuss our study procedure (\Cref{sub:Study Procedure}) and participant recruiting process (\Cref{sub:Participant Recruitment}).

\subsection{Choice of Task: Charitable Donation}
\label{sub:Choice of Task: Charitable Donation}
There are many different compelling behavioral contexts on which to test the role of the comic form: personal wellness goals (e.g., diet, exercise), mundane tasks (e.g. ``pick up dry cleaning''), as well as broader public-goods issues (e.g. ``take the flu shot;'' ``donate to cure cancer'').

We identify four criteria that consistent with prior literature \cite{lee2013does, sussman2015framing,saunders2016no,rumsey2003influence} to guide us, in our choice of the experimental behavioral context: nature of the reward; single shot tasks; an ecologically valid task; absence of specialized knowledge to perform the task. First, we would like the rewards to be distant, and non-exclusive, rather than proximal and exclusive so that individuals won't perform the task in anticipation of the immediate reward. Thus public goods dilemmas (e.g. ``reducing carbon footprint;'' ``taking the flu shot''; ``contributing to public knowledge'') are all candidates. Second, while some longitudinal tasks (e.g., losing weight; eating healthy) have distant rewards (losing weight, or maintaining a diet takes time), and can positively affect the public good (with more healthy people, in the long-run, insurance rates will fall), these tasks are prone to habit formation, a potential confound. Furthermore, single-shot tasks such as ``pick up yogurt at the grocery store today'', often prompted by text reminders from our calendars or task-tracking apps, have an immediate, exclusive reward. Third, we would like to ensure that the experimental task is ecologically valid---a task that these individuals would be actually asked to perform in the real world, outside of the experimental context. Fourth, we would like the task not to require specialized knowledge (e.g. ``asking doctors to make a decision''), so that other researchers could easily replicate and scale. 


Online charitable donation tasks satisfy those criteria as 1) they are single-shot tasks, 2) they contribute to the public good with distant, non-exclusive rewards, 3) requests for charitable donations frequently occur online, and 4) the task allows for experimental replication. 


We also considered two other in-the-wild experimental scenarios: buying healthy food and exercise. We considered the task of persuading individuals to buy healthy food by sending them push notifications on the smartphone when they were shopping for groceries. We encountered two challenges. First, we could not control for confounds in the decision making (if there are discounts on salad / healthy items; if the shopper was in a rush; if the shopper was shopping with friends who were pursuing a healthy lifestyle). Second, the reward had a possible confound for some participants, healthier food may make them feel better, and thus they may make the choice to buy healthier food not because of the message, but because of the anticipated reward. We also considered persuading individuals to adopt healthy behaviors, such as taking a walk or going to the gym. Then, we can measure gym visits or exercise outcomes. However, the central challenge here is that those activities are prone to habit formation, a confound. If habit forms, it may be more important in causing repeated behavior than the message received.

In our study, we choose the Organization for Autism Research (OAR) for three reasons. First, Autism Spectrum Disorder (ASD), is a well-recognized developmental disorder that impairs communication and behavior \cite{american2013diagnostic}; ASD provides basic interest for the participants to support the related charitable organization. Second, the Organization for Autism Research is one of the most visible ASD related organizations that helps individuals with autism and provides assistance to their parents, families, teachers, and caregivers. The goal of OAR is clear and reputable so participants won't question the authenticity of our message's motive. Finally, we wished to avoid a charity associated with a life-threatening condition such as cancer as it may create an experimental confound: we don't know if someone donates because their intrinsic desire to help with a life-threatening situation. While ASD can have serious consequences on the well being of those who have it, the public perception is that ASD is not-life threatening \cite{american2013diagnostic}. 


\subsection{Experiment Design}
\label{sub:Experiment Design}
Since the main goal for this study is to compare the power of a persuasive message in two forms, the abstract comics and the text, we first constructed two experimental conditions, the comic condition (see~\Cref{fig:basic three comic panel}) and the text condition. In the comic condition, participants will read a message asking if they are willing to support a charity in a three-panel abstract comic strip, whereas in the text condition, participants will receive the same message in plain text form. 

To test the idea of social proof, we then added a third condition, the comic with the social proof condition. Participants will read a three-panel comic strip that has the same content as the comic condition but added one line text indicating the normative behavior (see~\Cref{fig:basic three comic social proof}). To gather the basic statistics to create social proof, we first ran a pilot study with the first two conditions ($n=60$) on Amazon Mechanical Turk and used the donation statistics as part of the social proof message. In the pilot study, 87\% of the participants donated a non-zero amount. 

\begin{figure}[bt]
    \centering
    \includegraphics[width=\columnwidth]{./figures/abstract_comic.png}
    \caption{Messages in the abstract comic form. Same as the text messages, the three-panel comic strip communicates three points: 1) Participants will have 10\% of a chance winning \$5 bonus upon the completion of the study (see the first panel); 2) Participants are free to use the money as they please (see the second panel); 3) Participants can donate this bonus to the Organization for Autism Research (OAR) (see the third panel).}
    \label{fig:basic three comic panel}
\end{figure}

The objectives of the persuasive message in all three conditions are the same: persuading participants to donate to a charity from his/her own pocket. Similar to \textcite{lee2013does,saunders2016no}, the money participants will use is part of their study reward, a prospective bonus reward (10\% chance of winning \$5 bonus). We randomly assigned study participants to one of the three conditions; the participants are free to make a decision on the amount of donation, including not donating at all.

\subsection{Constructing Persuasive Messages}
\label{sub:Constructing Persuasive Messages}
Now we discuss construction of the three different messages: a plain text message, a comic, and a comic with social proof. Each message communicates three points: 1) Participants will have 10 \% of a chance winning \$ 5 bonus upon the completion of the study. 2) Participants are free to use the money as they please. and 3) Participants can donate this bonus to the Organization for Autism Research (OAR). Therefore, in the text condition, study participants will read the following message,

\begin{quote}
  \textit{You have a 10\% chance of winning a five dollar bonus in this study. You are free to use this money as you please. If you win the prize, would you like to donate to support the Organization for Autism Research?}
\end{quote}

In the two comic conditions, we created three-panel comic strips to communicate the \textit{same} text message. We used three panels to communicate each of the three points in the message. 

While aesthetic considerations govern the choice of the number of panels in the comic form, two factors influenced our choice of the number of panels: the number of points in the message, and communicating the idea of a conversation, over time, between two individuals. Since we wish to communicate three points (chance of winning bonus; free to use the money as they please; voluntary donation) in the short comic, to ensure that each point was salient, we chose to communicate each point in a different panel. Comic panels along with the text bubbles within a panel fragment time and space. Time flows in two ways: vertically in each panel through the dialogue between the characters and across panels. Since this is a short conversation, three panels seemed appropriate. While adding more panels would elongate the perceived sense of time, we wished to be economical in our communication of the interaction between two characters. We can control of the number of panels in our comic generator (see~\Cref{sec:Discussion}), and we leave the question of how the number of panels (and thus the perceived sense of time) affects message comprehension for future work.

To help the reader connect across panels to achieve closure~\textcite[][Chapter 3]{scott1993understanding} suggests that it is essential to match the panels. Different techniques to match exist, including ``moment-to-moment'', ``action-to-action'', ``scene-to-scene'' etc. help the reader bridge different time scales (see~\parencite[][p. 71]{scott1993understanding} for a summary). We used the ``action-to-action'' matching technique~\cite{scott1993understanding} since all the activity occurs in one scene, and where the activity depicts a brief dialogue between two characters. In particular, we matched the panels on the gesture of the first character (the message recipient), while retaining a neutral gesture for the second character (who delivers the message). 


In the comic with social proof condition, we created social-proof by adding one sentence on the last comic panel indicates the percentage of people in our study donated (\Cref{fig:basic three comic social proof}). The percentage (87\%) corresponds to the number of people who donated a non-zero amount in the pilot study.

For the purposes of this study, we created a comic generator to generate comic strips used in the study. The generator leveraged an open source comic library, ``cmx.io'' ~\cite{cmx.io}, to generate comic figures and used ``rough.js'' \cite{rough.js} to generate other elements such as text bubbles and outline frames. The generator impersonates the style found on ``XKCD''~\cite{munroe2009xkcd} comic. Our focus is not the XKCD style, but the fact that the generated comic is abstract. We believe the generator has the potential to be further developed as a general framework to automatically synthesize pure-textual persuasive messages into abstract comic forms (we discuss this aspect further in section ~\Cref{sec:Discussion}).


\subsection{Study Procedure} 
\label{sub:Study Procedure}
Once participants consented to join the study, we ask them if they are familiar with the Autism Spectrum Disorder (ASD). Then, each participant watched a short video produced by the Organization for Autism Research that promotes its fundraising activity "RUN FOR AUTISM" ~\cite{youtube_research}. After watching the video, we asked participants to summarize the video using free text and ask them to provide their opinion about the effectiveness of the video. The recruiting message specifically mentioned the task is soliciting their opinion on video message's effectiveness. There are two main goals for this part of the study: first, similar to the informational materials (e.g. text) used in other donation studies \cite{lee2013does,10362981,feiler2012mixed}, the video provides context. We want to make sure prior familiarity with autism do not confound our study; second, we want our main task less intrusive as soliciting charitable donation is not a common task on Amazon Mechanical Turk. We did consider establishing context by other means, such as an informative page with text and images on the organization of autism research. We decided against such an approach, because we could not guarantee that all subjects would have read the page. Furthermore, making subjects answer questions about the web-page to check if they had read the page carefully, seemed artificial, impinging on ecological validity. Finally, the linearity of the video is an advantage: subjects could advance to the next page only after watching the video, ensuring consistent knowledge to the best of our ability.

We then randomly assigned participants to one of the three conditions and asked them to read the corresponding persuasive message (text, three-panel comic, three-panel comic with social proof). In the message, we provided the participant with a 10\% chance of winning \$5 additional bonus reward. We also provided them with the opportunity to donate to the Organization for Autism Research (OAR) which is the charity mentioned in the video they previously watched. 

To best demonstrate the persuasiveness of the message, in our messages, we made it clear to participants that they can use the bonus freely as they please. The donation is completely voluntary and irrelevant to the compensation. The default donation amount for everyone is set to \$0 to control for the default effect. Similar to ~\textcite{lee2013does}, we also diffused the responsibility of donation amount among all participants. Before the participants make their decision, they read ``The total amount of money allocated to [the charity] by all the winning participants will be aggregated and donated at the end of the study.'' Then, we asked the participants to decide the amount of money they are willing to donate on a slider bar with \$0 and \$5 as two extreme ends. To increase the realism of our study, the donation decision is not hypothetical: the winning participants receive the amounts that they choose keep; as study organizers we donate to the Organization for Autism Research (OAR) the cumulative sum of the participant donations. 

Before leaving the study, participants need to filled a demographic questionnaire about their gender, income and education; participants had the option of declining to state an answer for each question.

At the end of the study, we randomly chose 10\% of the participants, donated to OAR based on their decision, and rewarded them with part of the bonus that they wished to keep.

\begin{figure}[bt]
    \centering
    \includegraphics[width=\columnwidth]{./figures/social_proof.png}
    \caption{Messages with social proof. In addition to the three points that we wish to communicate (chance of winning bonus; free to use the money as they please; voluntary donation), the comic with social proof communicates the idea of the social proof at the third comic panel. The figure ``87\%'' comes from our pilot study}
    \label{fig:basic three comic social proof}
\end{figure}



\subsection{Participant Recruitment}
\label{sub:Participant Recruitment}
In this study, we recruited our study participants from Amazon Mechanical Turk. Although the Amazon Mechanical Turk, a crowdsourcing platform, has been widely used to gather human intelligence in AI research and social science experiments \cite{ paolacci2014inside,berinsky2012evaluating,buhrmester2011amazon,branas2018gender,lee2013does,saunders2016no,arechar2017turking,sussman2015framing}, we should be cautious when using such platform as the participant selection criteria is not transparent \cite{landers2015inconvenient,paolacci2010running}. In our study, we chose the Amazon Mechanical Turk for the following reasons. First, our persuasion task is about online charitable donation which targets internet users. Second, crowdsourcing platforms will help us reach a more diverse sample than using the researchers' own social network to attract participants \cite{buhrmester2011amazon}. Third, the main motive for Amazon Mechanical Turk workers is monetary rewards \cite{berinsky2012evaluating}. In our study design, Amazon Mechanical Turk subjects may be more sensitive to the monetary reward they receive, making them less amenable to persuasion. Finally, the use of Amazon Mechanical Turk subject pool by multiple studies~\cite{branas2018gender,lee2013does,saunders2016no,arechar2017turking,sussman2015framing} on charitable donation decisions informed our decision to use the Amazon Mechanical Turk subject pool. 

Motivated by studies that show that populations on Amazon Mechanical Turk are diverse and mirror the US population \cite{buhrmester2011amazon,behrend2011viability,berinsky2012evaluating}, we recruited our participants from Amazon Mechanical Turk. However, we are aware of research that raises concerns in Amazon Mechanical Turk's sample representativeness \cite{landers2015inconvenient,paolacci2010running}. One potential solution is to use a panel company's population (e.g., Qualtrics). However, this method also has concerns in that the researcher can not directly cross-validate the sample's representativeness---we have to trust the company's assertion.

We published our HITs on Amazon Mechanical Turk titled ``A short survey about communicating autism campaign ads``. The compensation was \$10/hr, and the workers would get the compensation regardless of their performance. To ensure quality, the HIT is limited to English Speakers in US and people who have a 95\% Approval Rate. On the HIT page, we instructed that repeated responses would be rejected. We told the participants that they would see a link to our experiment site (Qualtrics). 
