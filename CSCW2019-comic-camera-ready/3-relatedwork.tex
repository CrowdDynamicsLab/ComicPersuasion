%!TEX root = cscw2019-comic.tex
\section{Related Work}
\label{sec:relatedwork}
Now, we discuss related works in: 1) Using Comics to Communicate Ideas, 2) Persuasion through Visual Stimuli, 3) Persuasion for Public Goods, and 4) Social Proof in Persuasive Messages.

\subsection{Using Comics to Communicate Ideas}
\textcite{scott1993understanding} defines comics as ``juxtaposed pictorial and other images in deliberate sequence, intended to convey information and/or to produce an aesthetic response in the viewer.`` While reading comics book is commonly recognized as entertaining, comics have been examined as an effective way of communicating abstract and complex ideas to broad audiences \cite{McDermottPB18,cary2004going,scott1993understanding, Zhang-Kennedy:2017:SCI:3206217.3206282}. On one hand, the simple and humorous nature of comics makes it a unique media for delivering informative and memorable messages. By combining visual elements and text, comics makes the story more appealing. \textcite{McDermottPB18} used comics to illustrate complex scientific facts. In education, comics has been used and examined as a useful media for reaching populations with various backgrounds \cite{McDermottPB18,cary2004going,scott1993understanding}. \textcite{Zhang-Kennedy:2017:SCI:3206217.3206282} created ``Secure Comics`` to educate end-users on computer security knowledge. On the other hand, the use of metaphor in comics can make the underlying meaning more vivid and memorable than using a straightforward description \cite{McDermottPB18,scott1993understanding}. Moreover, comics can contain a personal story which is incredibly powerful to create empathetic feelings, a key factor in persuasion, for readers ~\cite{weaver2017losing}. ~\textcite{matsubara2016emotional} showed a link between comic's contents and reader perceived emotions. Thus complex messages can be easily interpreted and memorized through the comic form. Although comics has shown its strength in communicating complex and memorable ideas, its utility in persuasion has not yet been fully explored. To the best of our knowledge, we are the first study looked at abstract comics in persuading people to cooperate in collective action dilemmas.  

In our study, we chose abstract comics to persuade instead of other comic forms for the following reasons, 1) As the comic becomes more abstract, readers will be more likely to project themselves onto the character and internalizing the information that the character is trying to convey \cite{scott1993understanding}. We believe such internalization gives an extra leg when the comic is designed to persuade. 2) The simplicity requires less cognitive load from the reader to consume the message, which is ideal for persuasion since persuadee's attention is scarce \cite{Janssen2016}. 3) Comparing to other forms of the comic, the abstract comics contains least elements which lowers the cost to generate. The simplicity of the abstract comics allows us to explore the idea of algorithmically synthesizing persuasive messages into the comic form. Therefore, in this study, we examined the persuasive power of messages in abstract comic form.

\subsection{Persuasion Through Visual Stimuli}
Beyond the realm of textual forms, prior research showed that using visual representations, e.g., graphics, video, and comics, are attractive in persuasion. ~\textcite{selker2015sweetbuildinggreeter} used motivational graphics or memes from 9GAG to attract people's attention and persuade people for energy conservation behaviors. \textcite{consolvo2008activity} visualized user's exercise data in the ``Ubifit Garden'' to persuade people to exercise. Sometimes, the use of visuals can evoke strong emotions that makes the message more persuasive. ~\textcite{iyer2006picture} found that people who saw the images of the Kenneth Bigley kidnapping were more engaged in the later civic campaign than those who read about the kidnapping via text. Visual stimuli were made for more memorable messages~\cite{nisbett1980human}. 
Visuals can influence decision making, especially when individuals make judgments using availability heuristics.~\textcite{dey2017art} found that the video quality in the crowdfunding campaign plays a vital role in persuading people to support. Despite visual stimulus' benefits in persuasion, creating an effective visual persuasive message is often costly. The persuader needs to put time, effort and resource to create persuasive visual stimuli. In our study, we investigated the use of an abstract comic, a simple visual form that can be relatively easily created, for its persuasiveness in public goods dilemmas.

\subsection{Persuasion for Public Goods}
Public goods are non-exclusive and non-rivalrous, and thus encouraging individuals to participate in collective action dilemmas such as charitable donations is challenging~\cite{marwell1981economists,isaac1982public}. Therefore, external nudges play an important role in encouraging people to contribute. Researchers and policymakers have extensively studied public goods dilemmas to identify who contributes and how to persuade people to contribute ~\cite{olson2009logic,becker1974theory,andreoni1990impure,miguel2005ethnic,burnett1981psychographic,pessemier1977willingness,burnett1981psychographic}.  ~\textcite{midden2008using} reported strong persuasive power for environmental sustainable behaviors when signaling personal goals in the persuasive application. \textcite{feiler2012mixed} found emphasizing altruistic reasons in a donation request can elicit more donations from people.~\textcite{mankoff2010stepgreen} successfully used social technologies to leverage public commitment and competition in appealing to energy-saving behaviors. 

A message composed with text is one of the most widely used methods to persuade individuals for behavior change or for charitable donations. Prior research indicates that textual messages are effective. ~\textcite{damgaard2017now} successfully used a simple email reminder with the decision deadline to elicit charitable donations. With a simple sign like ``Turn off the tap when not in use,'' people reduced water consumption and engaged in energy conservation behaviors ~\cite{mckenzie2011fostering}. ~\textcite{vaish2018s} showed empathizing self-serving benefits can attract more people to sign up for a pro-social peer-to-peer service.  ~\textcite{cotterill2010impact} showed sending a pledge card with simple text ``A list of everyone who donates a book will be displayed locally'' encouraged 22 \% more households to donate books for Children's Book Week. However, textual messages are less effective in catching perusadee's attention when compared to other visual forms, especially when persudaee's attention is limited. Moreover, when it comes to memorability, one of the key measures of persuasive effect, studies found that compared to graphics and video, text messages are the most difficult to recall~\cite{houts2006role,schmitt1993memory,messaris1997visual}. 

\subsection{Social Proof in Persuasive Messages }
By ``social proof'', we refer to the idea that when individual's observation of either their friends or others they can relate adopted a behavior is persuasive for the individual to adopt the same behavior~\cite{Cialdini1993, Cialdini2004}. ``Social proof'' is widely used in encouraging individuals to cooperate in collective action dilemmas \cite{goldstein2008room,schultz2007constructive}. ~\textcite{goldstein2008room} conducted an experiment in a hotel on motivating environmental conservation. They found that descriptive norms in the persuasive message (e.g., ``the majority of guests reuse their towels'') were more persuasive than only mentioning environmental protection in the message. And the normative message got more effective when the statement was about a provincial norm (e.g., ``the majority of guests in this room reuse their towels''). \textcite{amblee2011harnessing} studied ``social proof'' in online book reader communities and found that ``electronic word of mouth'' affects book quality, author  reputation, and book category popularity, which eventually influences people's buying decision. 