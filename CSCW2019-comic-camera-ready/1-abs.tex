%!TEX root = cscw2019-comic.tex
\begin{abstract}
This paper examines the use of the abstract comic form for persuading online charitable donations. Persuading individuals to contribute to charitable causes online is hard and responses to the appeals are typically low; charitable donations share the structure of public goods dilemmas where the rewards are distant and non-exclusive. In this paper, we examine if comics in abstract form are more persuasive than in the plain text form. Drawing on a rich literature on comics, we synthesized a three-panel abstract comic to create our appeal. We conducted a between-subject study with 307 participants from Amazon Mechanical Turk on the use of abstract comic form to appeal for charitable donations. As part of our experimental procedure, we sought to persuade individuals to contribute to a real charity focused on Autism research with monetary costs. We compared the average amount of donation to the charity under three conditions: the plain text message, an abstract comic that includes the plain text, and an abstract comic that additionally includes the social proof. We use Bayesian modeling to analyze the results, motivated by model transparency and its use in small-sized studies. Our experiments reveal that the message in abstract comic form elicited significantly more donations than text form (medium to large effect size=0.59). Incorporating social proof in the abstract comic message did not show a significant effect. Our studies have design implications: non-profits and governmental agencies interested in alleviating public goods dilemmas that share a similar structure to our experiment (single-shot task, distant, non-exclusive reward) ought to consider including messages in the abstract comic form as part of their online fund-raising campaign.
\end{abstract}