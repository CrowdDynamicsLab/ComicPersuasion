%!TEX root = cscw2018-comic.tex
Our analysis shows that persuadees prefer persuasive messages in comics form over plain text. We found in a persuasive comic, different character's gestures and background shading can influence persuadee's perception of the persuasiveness whereas no strong effect was found in inter-character distance. This was consistent with previous research on visual stimulus in persuasion and the benefits of comics in communication. However, previous studies on comics composition suggests the inter-character distance can affect reader's perceived relationship between characters and therefore influence their perception of the comics. We are wondering if persuadees project themselves into the comics as one of the character and did not recognize the distance between characters as the closeness interpersonal relationships. For example, people may read the comics from a third-person view. Then we looked into some feedbacks from our participants in the pilot study. One participant (p1) mention "I really like the comic I just saw and I feel bad that someone told me ..." which suggests the reader does think his/her is in the comic and having a conversation with someone else. Yet, it is unknown if the persuadee perceive his/her relationship with the persuader based on the inter-character distance. Our model also suggests persuadees perceives comics where two characters are far-away as most persuasive, even though theories in persuasion and psychology suggests people are more likely to be influenced and convinced by their close friends \cite{daddis2008influence,merga2014peer,shin2013user}. It is possible that compositionally tight-packed comics are perceived as less visually pleasing and introduce unwanted confounding effect from clustered elements.

\subsection{Comics with Color}
Our model suggests the background shading in a persuasive comic affects its persuasive power which makes us wonder the role of color as previous studies suggest the usage of color communicates the emotional intensity similar to the background shading. We then ran a small scale study comparing the perceived persuasiveness between black-white comics in our study and their corresponding colored version. With 60 participants from the Mechanical Turk,  we found our persuadee perceives colored version as more persuasive and there is potential interaction effect between negative-positive framing and different colors. In this light, further research on creating persuasive comics should further investigate the effect of color, e.g. what color should use, where should be colored.

\subsection{Towards Computational Persuasion}
Research on computational persuasion identified four major challenges including insufficient domain knowledge, constrained persuasion protocols, unrepresentative persuadee models and lack of optimal persuasion strategies\cite{huntertowards}. We broaden the design space of persuasion protocols by using comics to communicate persuasive messages. And in this study, we created a framework that can automatically create XKCD style persuasive comics based on the input of conversation texts and a set of parameters that can configure three key elements in the comic including character's gesture, inter-character distance, and background shading. With the integration of computational persuasion model in the future, the framework can synthesize persuadee's personal data to create comics with maximum persuasive power. For example, when persuading a risk-averse individual who values group norm to engage more exercise, a piece of persuasive comic showing a group of his/her friends are talking to the persuadee with negative-framed facts about him/her at a Gym can be algorithmically created by our future framework.

\subsection{Limitations}
Our current work has several limitations. First, our comics are limited to only single panel which hinders one of the most fascinating aspect of comics, storying telling. Comparing to the comic strip, it is difficult for single panel comics to show the dynamic among characters which can better bring the reader into the comics according to the previous research. Second, although comics can convey meanings for different topics, comics are rarely used in working email and official documents, which suggests comics may be better used for on topic than another. In our study, we only examined the persuasive power of comics in nudging people to exercise. Further investigations for other persuasion topics, such as stopping crime, increasing working efficiency, are needed.
