%!TEX root = proceedings.tex

In our study, we composed persuasive messages in both plain text and comic representations to compare the difference in their persuasive power and potential elements that may influence the persuasiveness of the comic representation. The main goal of all messages are the same, persuading readers to engage more exercise. The main reason to choose this topic are 1) Everyone is our target audience, since exercising is the key to maintaining health which is equally important to everyone. 2) As one of the most common activities in our life, people can easily understand the message and related to themselves. 3) Engaging more exercise is mostly based on audience own willingness instead of any subjective resources, such as money, that may confound the measure of persuasiveness.\par
\subsection{Composing Persuasive Messages in Plain Text}
Borrowing the idea from Psychology and Behavioral Economics, the key persuasive technique we adopted is implying social norm through the messages. Also, we incorporate the idea from Tversky and Kahneman that people will be influenced differently if the same message is framed as risk averse or risk taking. Therefore, we created two sets of messages that either framed from a positive standpoint or a negative standpoint. \par
The followings messages were presented in our study.\par
\textit{Positive framed messages:}
\begin{enumerate}
  \item In the past week, you spent more time at the gym than did 65\% of your friends
  \item Congrats! You have reached your goal of exercising three times a week.
  \item Over the past month, you exercised more than did 90\% of your friends.
  \item Your exercise activity is in the top 20\% of all your friends.
  \item Over the past three weeks, you went to the gym more often than 60\% of your friends did.
\end{enumerate}\par
\textit{Negative framed messages:}
\begin{enumerate}
\item	In the past week, you spent less time at the gym than did 65\% of your friends
\item Congrats! You did not reach your goal of exercising three times a week.
\item	Over the past month, you exercised less than did 90\% of your friends.
\item	Your exercise activity is in the bottom 20\% of all your friends.
\item	Over the past three weeks, you went to the gym less often than 60\% of your friends did.
\end{enumerate}\par
\subsection{Communicating through Comics}
As a form of art, the creation of comics has few limitations. Although there is no common template that could describe all comics, if we take a closer look at each comic, it is not hard to see that every comic consists of several fundamental components. We categorize these comic elements into different groups:
\begin{enumerate}
\item	characters,
\item gesture,
\item	background color/shading,
\item	word bubble,
\item	Over the past three weeks, you went to the \end{enumerate}\par
In this study, to represent persuasive messages in a comic form, we need to determine each of those four parameters.\par
As Mr. McCloud mentioned in his book “Understanding Comics”, as the comic getting abstract, the reader is more likely to project him/herself onto the character in the comic. By taking the perspective of the character, the reader will internalize the information his/her character trying to express or receive. If the information is persuasive, the internalization will imply a higher chance of expected behavior change. Therefore, in this study we choose to use an abstract yet well-recognized comic style, the stick figures.\par
Beside the abstractness of the character, we believe the relationship between characters is also important. In real world, previous research suggests that messages are more persuasive if the person communicating the ideas someone the receiver relates []. People are more likely to believe their close friends rather than strangers []. In abstract comics, the relationship between characters usually modeled by the distance between characters. It is reasonable to believe the link still holds in the world of comics as the reader tends to project his/herself onto the character. Therefore, we hypothesized that the distance between characters in a comic may influence the persuasive power.\par
The gesture of a character is another important component in the comic. The gesture of a character can help reader to understand what happens and the emotion of the character. Different gestures also imply the intensity of an emotion. As a common technique, cartoonists often use the gesture to intensify the feeling that they want to express to the reader. For a persuasive message, the intensified feeling may make the message more memorable than a plain tone, which increases its persuasive power. Thus, in this study, the gesture is another key element that we believe may moderate the effectiveness of a comic message.
A rich body of research has demonstrated the relationship between color or background shading and the emotion. In comics, the color of elements or the background shading contributes significantly to the feelings as well. However, due to the scope of our study, we only manipulate the background shading to show its influence on the persuasive power. \par
The word bubble is the most common place in comics to incorporate text information. In a persuasive comic, the word bubble is used to express the meaning of the message.  \par
In this study, we create a based template for all comic messages that includes two characters in a conversation and the scenario is 'One day, your friend has something to tell you.'\par
