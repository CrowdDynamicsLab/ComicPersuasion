%!TEX root = cscw2018-comic.tex

\section{Results}
\label{sec:Results}

\subsection{Raw Data}
\label{sub:Raw Data}
In this section we describe the raw data counts, the number of participants, the number of people whose responses we dropped. The final number of observations

\subsection{Analysis}
\label{sub:Analysis}
We use a Bayesian formulation of the problem of identifying suitable predictors for the comic form.~\textcite{Kay2016} provide an nice introduction on the appropriateness of Bayesian analysis for the HCI community. Bayesian analysis is attractive in our experiment due to two advantages: shifting the conversation from ``did it work'' to ``how strong is the effect''; and benefits to small $n$ studies.

In our experiment, the problem is that of ordinal regression, since the responses are on a Likert scale. There are five independent variables: gesture of the participants in the comic;  distance between the two characters; comic shading; whether the information was positively framed or negatively framed; and whether we presented the comic panel to the left or to the right. The last manipulation to guard against information ordering effects. Thus, we need to estimate the effect on the responses for each of these variables.

A challenge with using ordinal scale: we do not know the ``width'' of each response. That is, while we may know that for example $1<2<3$, we don't know if the range used by subjects to mark ``2'' on the scale, is the same as the range they use for ``1'' and ``3.''  We assume that each response by a subject lies in a continuous metric space, is Normally distributed and the subject then uses thresholds to identify the appropriate ordinal value. The thresholds subjects use to identify each level are unknown and we need to estimate them.


Formally let $r$ be the response of the subject to the experiment where the comic panel was generated by setting each of the five independent variables $\{x_i\}$. Then, since we assume that the response variable $r$ is Normally distributed:

\begin{align}
 r      & = N(\mu, \sigma)                    \label{eq:response-main}                   \\
 \mu    & = \beta_0 + \sum_{i=1}^k \beta_i x_i                        \label{eq:mu-main} \\
 \sigma & = hC(L) \label{eq:main-sigma}
\end{align}
\Cref{eq:response-main} says that the response variable is Normally distrubuted with mean $\mu$ and standard deviation $\sigma$.

\begin{align}
 z        & = \beta_0 + \sum_{i=1}^k \beta_i x_i \\
 \beta_0  & \sim N(\mu_0, \sigma_0)              \\
 \mu_0    & \sim N((1+L)/2, L^2)                 \\
 \sigma_0 & \sim U(a,b)                          \\
 \beta_i  & \sim  N(\mu_i, \sigma_i)             \\
 \mu_i    & \sim N(0, L^2)                       \\
 \sigma_i & \sim U(a,b)
\end{align}
