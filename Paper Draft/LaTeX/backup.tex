\subsection{Computational Persuasion}
With the advance in human technology and artificial intelligence, in very recent years, researcher starts to consider persuasion as an automated process where a computational model can algorithmically persuade target persuadee with effective persuasion strategies, e.g. engaging conversation with virtual agents\cite{huang2007design,nguyen2008designing,KangT15}. Huang and Lin proposed a virtual sales agent to persuade potential customers to offer a better price \cite{huang2007design}. By using machine-learning-based approach, an augment graph is trained based on simulated scenarios. Through a laboratory and online experiment, the results show this virtual sales agent can increase buyer's product valuation and willingness-to-pay efficiently. However, the persuasive power may be traded if such virtual agent provides inappropriate augmentation. Nguyen and Masthoff reviewed a number of argumentation-based systems and concludes comparing to the confrontational approach, arguments that based on social relationship and intrinsic motivation may be more efficacious \cite{nguyen2008designing}. Kang et. al developed a computational model called Model for Adaptive Persuasion that provides a unified framework for different persuasion strategies \cite{KangT15}. MAP is grounded in the Elaboration Likelihood Model that can select different persuasion strategies based on persuadee's feedback. With an evaluation of 26 elderly subjects, the result shows a MAP-based agents can change persuadee's attitude intentionally \cite{KangT15}. However, the existing framework cannot personalize for persuadee's profile and existing beliefs. And the framework can only accounts for a limited number of persuasive strategies.\par
However, four challenges have not been addressed in the current state of knowledge in building computational persuasion system, including insufficient domain knowledge, constrained persuasion protocols, unrepresentative persuadee models and lack of optimal persuasion strategies\cite{huntertowards}. An effective computational persuasion model requires sufficient formalize domain knowledge about persudee goals, persuade preference, and system action base etc to generate related persuasive arguments; an effective persuasion protocol that can best leverage the constrain of the meida and deliver generated persuasive messages efficiently; representative persuade model that allows the persuasion system to optimize a persuasion model algorithmically based on persuadee's beliefs and preferences; effective persuasion strategies that harness the perusadee model and produce optimal moves to persuade \cite{huntertowards}. \par
In our study, we focused on persuasion protocols and persuasion strategies. On one hand, we developed a persuasive framework can automatically generate persuasive messages in the comics form which provides a novel persuasion protocol that can be adapted in an computational persuasion system. Our comic-based persuasion protocols can produce individualized persuasive comics based on perusadee's model in terms of the text content, figure gestures and background shading. On the other hand, our experiment results suggest a unique persuasive strategy which leverage the text framing and visual comics representation. \par

Our study builds on comic's characteristic of attracting reader's attention. We want to further extend the research by understanding if comic can not only engage the readers, but also persuade them to take action. From the study by Haughney, it proposed a comic-style solution to present qualitative research findings. The study proves that the modified comic panel layout for user experience reports can make it more likely for readers to read the whole report through \cite{haughney2008using}. This study offers an innovative idea to present visual information through comic layout. In addition, text bubbles in the layout highlights additional information about the research and can better attract reader's attention compared to text label by the side of a report. However, this design is specifically for presenting visual information such as user interface feedbacks. It is difficult to apply it to other forms of information or research findings. In our study, we utilize comics to present persuasive text posts and extend the research to examine the effect in persuading behavioral change.\par
