%!TEX root = proceedings.tex

Research on making messages more persuasive has been increasingly dominant due to the fact of information explosion. We live in an era where we are immersed in the information ocean. How to make individual message stand out among others, attract attention from readers and persuade behavioral change becomes an important problem. A lot of research has worked on techniques that can make a message more persuasive. Some focused on the framing of messages, some focused on the context of the message, while others focused on more graphic and visual aspect of message to make the message more attractive to readers. We build our research of using visual comics to attract reader attention on the previous work from all focus. In the following sections, we will introduce some of the related studies categorized into groups: (1) framing and phrasing of messages, (2) comparison between text and comic, and (3) communicating through comic and graphics.\par

\subsection{Message Framing & Phrasing}
Approaches to make a message more attractive and persuasive has long been a focus for a variety of different fields including computational linguistic, social networking, and advertising. Among these tactics, message framing and phrasing is one of the most basic and intuitive methods to generate memorable and persuasive messages. This is mainly because message framing and phrasing generally doesn't require additional information or data visualization. Its simplicity contributes to the various researches on the effect of message framing in memorability. Mizil et.al shows that using unusual word choices and more general theme makes it easier to connect with reader's daily life and makes the message more memorable. Using unexpected words and phrases, the message is more likely to capture reader's attention comparing to normal phrases. As for the theme of the message, it should be as general as possible to help readers connect with the message, so it can stay in the reader's mind longer. However, using more general background and theme to connect with readers and special phrases to catch reader's attention is not only applied in text-based messages. Mr. McCloud shows that comics have been using similar techniques. Comics use general background and characters to make users feel more involved with the story, and use catchy phrases to catch user's attention to make the comic more appealing. The abstract characters and background in comics make it easier for readers to relate themselves to the story, thus attracting reader's interest and attention.\par
Tversky and Kahneman takes on a different aspect of the effect of framing. Instead of focusing on the word selection and sentence structure of a quote, this paper focused on how different reference point of a same sentence can result in reader's different response. It shows that variations of reference point of a decision can determine whether the reader will evaluate it as a gain or a loss, thus changing their decision. For example, choices involving gains are often risk averse and choices involving losses are often risk taking. Meanwhile, implying social norm through the message is another persuasive technique.  Goldstein et. al conducted an interest experiment in the hotel on motivating environmental conservation. They found employing descriptive norms (e.g., “the majority of guests reuse their towels”) has more persuasive power than solely mentioning environmental protection. And this normative message gets more persuasiveness when the described setting is closer to individuals' immediate situational circumstances (e.g., “the majority of guests in this room reuse their towels”).  All those techniques are essential for our design of messages to persuade behavioral change. We believe comics and graphics compared to plain text can better express the feeling of gain or loss to readers through the character's gestures and background. Also, the nature of the comics encourages their reader to project themselves on the characters which provides a good opportunity to bring the reader closer to the message itself. Established research supports our hypothesis that comic can be more persuasive to change reader's decision.

\subsection{Comparison Between Text and Comic
}
This study isn't the first to address the comparison between text and comic. Previous research has focused on comparing whether text messages or image/comic messages are more attractive to readers [cite]. The results showed that image/comics can raise user's awareness and have a greater chance to persuade behavioral change. However, this study retrieves the comics from the internet on a certain topic instead of customized generated comic. The drawback of this approach is that it is harder to control the quality and characteristic of the comic. Different authors of comic might have very different drawing style, resulting in differing effects on persuasion.\par
Our work builds on top of the findings of this study that comic can better attract reader's attention. We generate customized comic messages from text messages to unify the comic style and evaluate the persuading effect. Aside from the comparison between text and comics, research has also focused on comparison between other forms of data and their effect in persuading behavioral change. Lin and Xiao compared two modes of presenting information about wind energy in brochure form: (1) photographs and (2) using cartoons as visual aids. To evaluate the effect, the research focused on comparing three measures: (1) audience's knowledge of, (2) attitudes toward, and (3) behavioral intentions regarding wind energy. Results show that there is no significant difference between using photographs or comics in terms of knowledge and attitudes. However, visual aids shown in the cartoon/comics version showed stronger behavioral intentions (e.g., greater willingness to support changes) than the photo group. Because of the abstractness of comics, it can better engage its readers in the messages compared to photographs, making readers more willing to adopt changes. In this study, it mainly focuses on comics' effect on persuading readers to agree on changes conducted by others. However, in our study, we focus on comic's effect on persuading readers themselves to adopt behavior change.\par

\subsection{Comics Speak}
Using comic to convey information isn't a new idea. Throughout the years, research on evaluating the effect of comic and graphic in communicating ideas have been more and more popular. From Mr. McCloud's work, it shows that different comic elements have different effects on the ideas and emotion communicated. For example, background shading, character gestures, and word balloon border style can all change the emotion and feeling associated with the message. This makes readers more emotionally engaged and more likely to accept the ideas conveyed by the comic. Other comic elements can help engage the readers, for example, research has shown that more abstract comic character can help readers relate themselves with the characters in the comic. This makes readers more engaged to the story and the idea expressed by the comic. Although previous studies have proved that comic can better attract reader's attention and engage them in stories, the research on using comic to convey formal information and persuade behavioral change is limited.\par
Our study builds on comic's characteristic of attracting reader's attention. We want to further extend the research by understanding if comic can not only engage the readers, but also motivate them to take action. From the study by Haughney, it proposed a comic-style solution to present qualitative research findings. The study proves that the modified comic panel layout for user experience reports can make it more likely for readers to read the whole report through. This study offers an innovative idea to present visual information through comic layout. In addition, text bubbles in the layout highlights additional information about the research and can better attract reader's attention compared to text label by the side of a report. However, this design is specifically for presenting visual information such as user interface feedbacks. It is difficult to apply it to other forms of information or research findings. In our study, we utilize comic to present persuasive text posts and extend the research to examine the effect in persuading behavioral change.\par
