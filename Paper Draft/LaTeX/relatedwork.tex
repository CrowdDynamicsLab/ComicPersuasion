%!TEX root = cscw2018-comic.tex

In the following sections, we will introduce related works in: (1) Building Persuasive Technology, (2) Computational Persuasion and (3)Persuasion Through Comics.

\subsection{Building Persuasive Technology}
Starting from Goehlert's discussion on persuasion and communications technology in 1980, HCI researchers have spent a lot of effort in leveraging technology in persuasion\cite{goehlert1980information}. Goehlert argues control and dissemination of information have the ability to make attitude and behavioral changen\cite{goehlert1980information}. Inspired by this argument, two types of approaches have been used in constructing persuasive systems. Information-centric approaches focused on delivering hidden or new information which has not been perceived or recognized by the user before\cite{LeeKF11}. For example, Chi et al. created an intelligent kitchen that can provide nutritional information about ingredients while users are cooking\cite{chi2007enabling}. People start to adjust their ingredients usages to achieve a better nutritional composition. Waterbot persuades people to engage water-saving behavior by augmenting physical sink interface with water usage indicators\cite{arroyo2005waterbot}.Liao and Fu found showing a source expertise indicator can shape user's information seeking behavior and burst the filter bubble \cite{liao2014expert}. While a lot of studies on persuasive technology evident the persuasive power of this information-centric approach, previous studies also show the downside of information-center approach where the target receiver often failed to perceive and rationalize the persuasive information when the receiver is experiencing information overload \cite{goehlert1980information,LeeKF11}. Additionally, the information-centric approach often relays on the target receiver can make rational decisions based on provided information whereas the premise of rationality does not always hold.

Adapting decision-making models from previous behavioral research, some persuasive technology emphasized on human motivation and biases, e.g. behavior-centric approach \cite{LeeKF11}. Lee et al. incorporated the idea of default bias in the design of the Snackbot robot and successfully persuade people with healthy snacking in the workplace \cite{LeeKF11}. Vaish et. al used self-serving motivational framing of messages to persuade people to sign up for a prosocial peer-to-peer service\cite{vaish2018s}. Borrowing from the theory of planned behavior, Schneider et. al understood different motivational factors of mobile fitness coach users and delivered individualized messages to persuade users based on their motivations\cite{schneider2016understanding}. Although the effectiveness of behavior-centric approach has been examined in multiple studies, behavior-centric approaches often require prior knowledge of target subject in order to maximize the persuasion power, as Orji et. al and Schneider et. al suggested in their study that different people may be more valuable to one persuasive method than others\cite{schneider2016understanding,orji2014developing}. Therefore, the common challenging faced by behavior-centric approach is scalability.\par
Our study, we leveraged both information-centric and behavior-centric approaches. The comic is designed to show user's behavior stats (information-centric) but the message is framed beyond raw representation to achieve maximum persuasiveness (behavior-centric). The challenge here is mitigating the downsides of information-centric and behavior-centric approaches. We used novel representation, visual comic, to catch information receiver's attention and developed a tool that can algorithmically generate personalized persuasive comics.

\subsection{Persuasion Through Comics}
The simple and humorous nature of comic makes comics becomes an unique media for delivering informative and memorable messages. While reading comics book is commonly recognized as entertaining, comics have been examined as an effective way of communicating abstract ideas to broad audiences \cite{McDermottPB18,cary2004going,scott1993understanding}. McDermott et. al used comics to illustrate complex scientific facts \cite{McDermottPB18}. In education, comics have been used and examined as an effective tool for reaching different populations with various background \cite{McDermottPB18,cary2004going,scott1993understanding}. Meanwhile, the common usage of metaphor in comics can make the underlying meaning for vivid and therefore more memorable, which is core in persuasion, than using a straightforward description \cite{McDermottPB18,scott1993understanding}.Moreover, comics can contain persuaee's personal story which is incredibly powerful in persuasion\cite{weaver2017losing}.With personal story, comics can express emotion and create empathy for readers. Matsubara et al. showed that there is a link between comic's content and the emotions felt by the readers \cite{matsubara2016emotional}. Thus, in the form of comics, complex messages can be easily interpreted and memorized. Given the advantage of using comics to deliver meanings, our study took one step further and considered the persuasive side of comic representation.

Throughout the years, research on evaluating the effect of comic and graphic in communicating ideas have been more and more popular \cite{McDermottPB18,cary2004going,scott1993understanding,weaver2017losing,matsubara2016emotional}. From Scott McCloud's work, it shows that different comic elements have different effects on the ideas and emotion communicated. For example, background shading, character gestures, and word balloon border style can all change the emotion and feeling associated with the message \cite{scott1993understanding,mccloud2011making}. This makes readers more emotionally engaged and more likely to accept the ideas conveyed by the comic. Other comic elements can help engage the readers, for example, research has shown that more abstract comic character can help readers relate themselves with the characters in the comic. This makes readers more engaged to the story and the idea expressed by the comic. Although previous studies have proved that comic can better attract reader's attention and engage them in stories \cite{scott1993understanding,mccloud2011making,McDermottPB18}, the research on using comic to convey formal information and persuade behavioral change is limited.

However, using comics to persuade is challenging. First, generating comics is not easy. Especially, persuasive messages are better to be personalized to deliver maximum persuasive power. Traditionally, comics are created by professional cartoonists which is very costly to produce personalized comics.  Although prior work has explored methods of algorithmically generating comics, no existing method is for generating persuasive comics.
