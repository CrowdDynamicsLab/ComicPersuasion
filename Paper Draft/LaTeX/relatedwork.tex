%!TEX root = proceedings.tex

Research on persuasive messages in HCI has been increasingly popular due to the fact of information explosion. Previous research has provided solid strategies that can make text messages more persuasive, including messaging framing, information-centric approaches, and personalize information context []. However, due to the advance in technology, we are immersed in the information ocean and overwhelmed with texts various sources. How to make individual message stand out among others and attract reader's attention while nudging the reader to make behavioral and attitude changes becomes an important problem. In this study, we looked into a novel form of persuasive message, visual comics. Then, we investigated the persuasive power of different visual cues in comics beyond pure text. In the following sections, we will introduce related studies categorized into groups: (1) Framing text-form persuasive messages, (2) Persuasion beyond text, and (3) Communicating through comics.\par

\subsection{Building Persuasive Technology}
Starting from Goehlert's discussion on persuasion and communications technology in 1980, HCI researchers have spent a lot of effort in leveraging technology in persuasion. Goehlert argues control and dissemination of information have the ability to make attitude and behavioral change. Inspired by this argument, two types of approaches have been used in constructing persuasive system. Information-centric approaches focused on delivering information which did not perceived or recognized by the user before. For example, Chi et al. created an intelligent kitchen that can provide nutritional information about ingredients while users are cooking. People start to adjust their ingredients usages to achieve a better nutritional composition. Waterbot persuades people to engage water-saving behavior by augmenting physical sink interface with water usage indicators.
Liao and Fu found showing a source expertise indicator can shape user's information seeking behavior and burst the filter bubble. While a lot of studies on persuasive technology evident the persuasive power of this information-centric approach, pervious studies also show the downside of information-center approach where the target receiver often failed to perceive and rationalize the persuasive information when the receiver is experiencing information overload. Additionally, information-centric approach often relays on the target receiver can make rational decision based on provided information whereas the premise of rationality does not always hold.\par

Adapting decision-making models from previous behavioral research, some persuasive technology emphasized on human motivation and biases, e.g. behavior-centric approach []. Lee et al. incorporated the idea of default bias in the design of the Snackbot robot and successfully persuade people with healthy snacking in the workplace. Vaish et. al used self-serving motivational framing of messages to persuade people to sign up for a prosocial peer-to-peer service. Borrowing from the theory of planned behavior, Scheneider et. al understood different motivational factors of mobile fitness coach users and delivered individualized messages to persuade users based on their motivations. Although the effectiveness of behavior-centric approach has been examined in multiple studies, behavior-centric approaches often requires prior knowledge of target subject in order to maximize the persuasion power, as Orji et. al and Schneider et. al suggested in their study that different people may be more valuable to one persuasive method than others.  Therefore, the common challenging faced by behavior-centric approach is scalability.\par

Our study, we tried to incorporate both information-centric and behavior-centric approaches. The comic is designed to show user's behavior stats (information-centric) but the message is framed beyond raw representation in order to achieve maximum persuasiveness (behavior-centric). The challenge here is mitigating the down sides of information-centric and behavior-centric approaches. We used novel representation, visual comic, to catch information receiver's attention and developed a tool that has the ability to algorithmically generate personalized persuasive comics.
\par

\subsection{Text Messages Persuade}
Approaches to make a message more attractive and persuasive has long been a focus for a variety of different fields including computational linguistic, social networking, and advertising. Among these tactics, message framing is one of the most basic and intuitive methods to generate memorable and persuasive messages. This is mainly because message framing and phrasing generally doesn't require additional information or data visualization. Its simplicity contributes to the various researches on the effect of message framing in memorability. Mizil et.al shows that using unusual word choices and more general theme makes it easier to connect with reader's daily life and makes the message more memorable. Using unexpected words and phrases, the message is more likely to capture reader's attention comparing to normal phrases. As for the theme of the message, it should be as general as possible to help readers connect with the message, so it can stay in the reader's mind longer.\par

Tversky and Kahneman takes on a different aspect of the effect of framing. Instead of focusing on the word selection and sentence structure of a quote, Tversky and Kahneman focused on how different reference point of a same sentence can result in reader's different response. It shows that variations of reference point of a decision can determine whether the reader will evaluate it as a gain or a loss, thus changing their decision. For example, choices involving gains are often risk averse and choices involving losses are often risk taking. Meanwhile, implying social norm through the message is another persuasive technique.  Goldstein et. al conducted an interest experiment in the hotel on motivating environmental conservation. They found employing descriptive norms (e.g., "the majority of guests reuse their towels") has more persuasive power than solely mentioning environmental protection. And this normative message gets more persuasiveness when the described setting is closer to individuals' immediate situational circumstances (e.g., "the majority of guests in this room reuse their towels").  All those techniques are essential for our design of messages to persuade behavioral change.\par


\subsection{Communicating through Comics}
The simple and humorous nature of comic make it becomes an excellent media for delivering informative and memorable messages. While reading comics book is commonly recognized as entertaining, comics has been examined as an effective way of communicating abstract ideas to broad audiences []. Bromberg et. al used comics to illustrate complex scientific facts. In the area of education, comics have been used and examined as an effective tool for reaching different populations with various background []. Meanwhile, the common usage of metaphor in comics can make the underlying meaning for vivid and therefore more memorable than using a straightforward description[]. Moreover, simple comics can express emotion and create empathy for readers. Lima Sanches et al. showed that there is a link between comic’s content and the emotions felt by the readers. Thus, with the form of comics, complex message can be easily interpreted and memorized. Given the advantage of using comics to deliver meanings, our study took one step further and considered the persuasive side of comic representation. \par

However, using comics to persuade is challenging. First of all, generating comics is not easy. Especially, persuasive message should be personalized to deliver maximum persuasive power. Traditionally, comics is created by professional cartoonists which is very costly to produce personalized comics.  Although prior work has explored methods of algorithmically generating comics, no existing method is for generating persuasive comics. \par
