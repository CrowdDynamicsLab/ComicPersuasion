%!TEX root = cscw2018-comic.tex
Research on persuasive messages in HCI has been increasingly popular due to the fact of information explosion. Previous research has provided solid strategies that can make text messages more persuasive, including messaging framing, information-centric approaches, and personalize information context []. However, due to the advance in technology, we are immersed in the information ocean and overwhelmed with texts various sources. How to make individual message stand out among others and attract reader's attention while nudging the reader to make behavioral and attitude changes becomes an important problem. In this study, we looked into a novel form of persuasive message, visual comics. Then, we investigated the persuasive power of different visual cues in comics beyond pure text. In the following sections, we will introduce related studies categorized into groups: (1) Building Persuasive Technology, (2) Computational Persuasion and (3)Persuasion Through Comics.\par

\subsection{Building Persuasive Technology}
Starting from Goehlert's discussion on persuasion and communications technology in 1980, HCI researchers have spent a lot of effort in leveraging technology in persuasion. Goehlert argues control and dissemination of information have the ability to make attitude and behavioral change. Inspired by this argument, two types of approaches have been used in constructing persuasive systems. Information-centric approaches focused on delivering hidden or new information which has not been perceived or recognized by the user before. For example, Chi et al. created an intelligent kitchen that can provide nutritional information about ingredients while users are cooking. People start to adjust their ingredients usages to achieve a better nutritional composition. Waterbot persuades people to engage water-saving behavior by augmenting physical sink interface with water usage indicators.\par
Liao and Fu found showing a source expertise indicator can shape user's information seeking behavior and burst the filter bubble. While a lot of studies on persuasive technology evident the persuasive power of this information-centric approach, previous studies also show the downside of information-center approach where the target receiver often failed to perceive and rationalize the persuasive information when the receiver is experiencing information overload. Additionally, the information-centric approach often relays on the target receiver can make rational decisions based on provided information whereas the premise of rationality does not always hold.\par
Adapting decision-making models from previous behavioral research, some persuasive technology emphasized on human motivation and biases, e.g. behavior-centric approach []. Lee et al. incorporated the idea of default bias in the design of the Snackbot robot and successfully persuade people with healthy snacking in the workplace. Vaish et. al used self-serving motivational framing of messages to persuade people to sign up for a prosocial peer-to-peer service. Borrowing from the theory of planned behavior, Schneider et. al understood different motivational factors of mobile fitness coach users and delivered individualized messages to persuade users based on their motivations. Although the effectiveness of behavior-centric approach has been examined in multiple studies, behavior-centric approaches often require prior knowledge of target subject in order to maximize the persuasion power, as Orji et. al and Schneider et. al suggested in their study that different people may be more valuable to one persuasive method than others.  Therefore, the common challenging faced by behavior-centric approach is scalability.\par
Our study, we tried to incorporate both information-centric and behavior-centric approaches. The comic is designed to show user's behavior stats (information-centric) but the message is framed beyond raw representation to achieve maximum persuasiveness (behavior-centric). The challenge here is mitigating the downsides of information-centric and behavior-centric approaches. We used novel representation, visual comic, to catch information receiver's attention and developed a tool that can algorithmically generate personalized persuasive comics.\par


\subsection{Computational Persuasion}
With the advance in human technology and artificial intelligence, in very recent years, researcher starts to consider persuasion as an automated process where a computational model can algorithmically persuade target persuadee with effective persuasion strategies, e.g. engaging conversation with virtual agents. Huang and Lin proposed a virtual sales agent to persuade potential customers to offer a better price. By using machine learning-based approach, an augment graph is trained based on simulated scenarios. Through a laboratory and online experiment, the results show this virtual sales agent can increase buyer's product valuation and willingness-to-pay efficiently. However, the persuasive power may be traded if such virtual agent provides inappropriate augmentation. Nguyen and Masthoff reviewed a number of argumentation-based systems and concludes comparing to the confrontational approach, arguments that based on social relationship and intrinsic motivation may be more efficacious. Kang et. al developed a computational model called Model for Adaptive Persuasion that provides a unified framework for different persuasion strategies. MAP is grounded in the Elaboration Likelihood Model that can select different persuasion strategies based on persuadee's feedback. With an evaluation of 26 elderly subjects, the result shows a MAP-based agents can change persuadee's attitude intentionally. However, the existing framework cannot personalize for persuadee's profile and existing beliefs. And the framework can only accounts for a limited number of persuasive strategies.\par
However, four challenges have not been addressed in the current state of knowledge in building computational persuasion system, including insufficient domain knowledge, constrained persuasion protocols, unrepresentative persuadee models and lack of optimal persuasion strategies []. An effective computational persuasion model requires sufficient formalize domain knowledge about persudee goals, persuade preference, and system action base etc to generate related persuasive arguments; an effective persuasion protocol that can best leverage the constrain of the meida and deliver generated persuasive messages efficiently; representative persuade model that allows the persuasion system to optimize a persuasion model algorithmically based on persuadee's beliefs and preferences; effective persuasion strategies that harness the perusadee model and produce optimal moves to persuade. \par
In our study, we focused on persuasion protocols and persuasion strategies. On one hand, we developed a persuasive framework can automatically generate persuasive messages in the comics form which provides a novel persuasion protocol that can be adapted in an computational persuasion system. Our comic-based persuasion protocols can produce individualized persuasive comics based on perusadee's model in terms of the text content, figure gestures and background shading. On the other hand, our experiment results suggest a unique persuasive strategy which leverage the text framing and visual comics representation. \par

\subsection{Persuasion Through Comics}
The simple and humorous nature of comic makes comics becomes an excellent media for delivering informative and memorable messages. While reading comics book is commonly recognized as entertaining, comics have been examined as an effective way of communicating abstract ideas to broad audiences []. Bromberg et. al used comics to illustrate complex scientific facts. In education, comics have been used and examined as an effective tool for reaching different populations with various background []. Meanwhile, the common usage of metaphor in comics can make the underlying meaning for vivid and therefore more memorable than using a straightforward description[]. Moreover, simple comics can express emotion and create empathy for readers. Lima Sanches et al. showed that there is a link between comic's content and the emotions felt by the readers. Thus, in the form of comics, complex messages can be easily interpreted and memorized. Given the advantage of using comics to deliver meanings, our study took one step further and considered the persuasive side of comic representation. \par
However, using comics to persuade is challenging. First, generating comics is not easy. Especially, persuasive messages are better to be personalized to deliver maximum persuasive power. Traditionally, comics are created by professional cartoonists which is very costly to produce personalized comics.  Although prior work has explored methods of algorithmically generating comics, no existing method is for generating persuasive comics. \par
