%!TEX root = proceedings.tex
Today, the world generates information all around us every second. We are surrounded by all sorts of messages trying to change what we think and what we do: our newsfeed is full of advertisements, our wearable devices are keeping telling us to exercise more, even our water bottle starts to push notification to remind us drinking more water. However, not all messages can successfully change our behavior: we won't buy an expensive car because of a short video, we often eat junk food even though we received a lot of articles about health eating and we are often in the status of dehydration after reading those notifications. Thus, how to make a message more persuasive has been a critical problem throughout the years.\par
While classical game theory suggests that recasting an informative message through a different form cannot increase the message's persuasive power, a rich body of research has demonstrated how susceptible we are to those fancy words. Hotel guests start to reuse their towels more because of a subtle change of a sign positioned on washroom towel racks. Households start to reduce their energy consumption because of an emoji on their energy bill. People are more willing to sign up for a prosocial peer-to-peer service because of a message on the sign-up page telling them explicitly what benefit they might get. Given the susceptible nature of our species, we believe changing the representation of a message can make the message more persuasive.\par
Comics, a medium to express ideas in a graphical form, are one of the most popular form of art across different cultures. The history of comics can be traced back to early precursors such as Trajan’s Column. Comics are usually in the form of juxtaposed sequences of panels of images or a standalone single image. Beside the graphical representation, textual elements such as speech balloons, captions, and onomatopoeia often communicate dialogue, narration, sound effects, or other information. Existing work shows that reader often prefer comics over plain text since comics are more attractive and more engaging than plain text. Yet the persuasiveness of the comics representation has not been investigated.\par
In this study, we present a research on the persuasiveness of comic style messages. First, we reviewed some of the previous works on persuasive technologies in three major fields: (1) framing and phrasing of messages, (2) comparison between text and comic, and (3) communicating through comic and graphics. Second, we discuss about the general composition of a comic message and its comparison with plain text. Lastly, to examine our hypothesis that comic messages are more persuasive than text messages, we conduct a field experiment on Amazon Mechanical Turk to investigate the persuasiveness of comic messages comparing with text messages. The result shows comics can be a better form of persuasive message comparing to plain text and three key comic elements, character gesture, inter-character distance, and background shading moderate the persuasiveness differently.\par
The contributions of this work are: 1) extending on previous research to not only whether comic messages can attract reader's attention, but also whether comic messages can actually motivate readers to take action. 2) among the first to examine the effect on reader's behavior for comic form messages.\par
