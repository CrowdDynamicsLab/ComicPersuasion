%!TEX root = cscw2018-comic.tex

% what is the problem?

% why is it important? what is the impact? what motivates?

% who else has done it?
% what did you do?

\section{introduction}
\label{sec:introduction}

Are statistical facts (e.g. ``you've exercised 25\% more than you did last week'') more persuasive when expressed in comic\footnote{We use ``comic form'' as opposed to the more formal ``graphic form'' to avoid any confusion with other visual representations of data, including charts and diagrams.} form? If so, what elements of the comic form are important? Furthermore, can these comics be algorithmically synthesized? This paper plans to address each of these questions.

The notion of behavior change motivates us to address these questions. As the quantified self movement~\cite{Epstein2014,Choe2014} illustrates, people have an enduring sense of curiosity about their lives, and develop different processes to instrument (e.g. using a wearable device) and to reflect on data gathered about their activities. individuals use this information, often in the form of statistics ( ``checked in at the gym twice this week''), to take decisions to change behavior (e.g. ``exercise more''; ``eat healthy''). In macro-scale collective action dilemmas, individual contributions to the dilemma (e.g. ``by traveling to Boston from Los Angeles on a plane, you generate 20\% of the greenhouse gases that your car emits each year'') are often presented as statistical facts. While visualizations often accompany these facts ( e.g. a graph of weight over time for a person interested in weight loss), the statistic itself is presented in textual form.

Human beings show significant resistance to changing their behavior based on statistical information. While there may be confounds that explain away poor adoption, including, message timing~\cite{Fogg2009} (i.e. when we show the fact), time scarcity~\cite{Janssen2016} (i.e. we don't have time to digest the information), viewing messages on smartphones~\cite{Kim2016} (screen is too small to communicate effective visualizations), the rethinking the textual form of the statistic for easier consumption in the context of behavior change has been largely unexplored. Simply: could statistical facts be more persuasive in comic form? At first blush, the comic may seem like ``cheap information'' in the framework of classical Game Theory, providing no extra information to the decision maker.

Three ideas underpin this paper: information framing, ``social proof,'' and the comic form. First, work in Behavioral Economics~\cite{tversky1992advances,tversky1981framing} shows that how we frame the text matters and can cause preference reversal. Briefly, individuals show different preferences to statements with identical information due to risk aversion (e.g. ``if you take the plane, there is a 50\% chance that you will fall ill'' vs. ``if you take the plane there is a 50\% chance that you will be healthy''; the former is more salient due to risk aversion). We plan to use frames to represent statistical facts. Second, work in Psychology shows that individuals' decisions are guided by social norms~\cite{goldstein2008room,schultz2007constructive}. In a well known experiment on the design of signs for towel reuse,~\textcite{goldstein2008room} showed that use of social norms in the message had a significant impact on towel re-use, when compared to the standard sign that asks us to re-use towels to save water. In a related study,~\textcite{schultz2007constructive} demonstrated the positive impact of comparing one's energy consumption patterns with that of one's neighbors. Today, many households in the U.S. receive bills from their utility companies comparing their energy use to their neighbors as a consequence of this study. In our paper, we use this idea of the ``social proof'', by comparing the performance of an individual to that of her friends. Finally, we use comics in abstract form to communicate. Comics are sophisticated form of art~\cite{scott1993understanding} popular across cultures allowing for use of humor and affect in communication. We focus on abstract ``XKCD-style'' ~\cite{munroe2009xkcd}  representations of comics. As~\textcite{scott1993understanding} points out, using abstract representations for the comic allows the reader to project themselves onto the comic character. Equally importantly, if we wish to algorithmically synthesize comic-style messages, abstract representations allows for a straightforward synthesis and for use in a wide variety of contexts. While comic form has found use in scientific communication~\cite{McDermottPB18} and in a multilingual classroom~\cite{cary2004going}, its use in behavior change through algorithmically synthesized comic messages is not yet well understood.

% one of the most popular form of art across different cultures.
%
%
% Comics are usually in the form of juxtaposed sequences of panels of images or a standalone single image. Beside the graphical representation, textual elements such as speech balloons, captions, and onomatopoeia often communicate dialogue, narration, sound effects, or other information. Existing work shows that comics can effectively convey meanings \cite{McDermottPB18,cary2004going,scott1993understanding}. Yet the persuasiveness of the comics representation has not been investigated.


% Hotel guests start to reuse their towels more because of a subtle change of a sign positioned on washroom towel racks \cite{goldstein2008room}. Households start to reduce their energy consumption because of an emoji on their energy bill \cite{schultz2007constructive}.






% Today, the world generates information all around us every second. We are surrounded by all sorts of messages trying to change what we think and what we do: our newsfeed is full of advertisements, our wearable devices are keeping telling us to exercise more, even our water bottle starts to push notification to remind us stay hydrated. However, not all messages can successfully make us change: we won't buy an expensive car because of a short video, we often eat junk food even though we received a lot of articles about health eating and we are often in the status of dehydration after reading those notifications. Thus, how to make a message more persuasive has been a critical problem throughout the years.\par

% While classical game theory suggests that recasting an informative message through a different form cannot increase the message's persuasive power, a rich body of research has demonstrated how susceptible we are to those fancy words \cite{tversky1992advances,tversky1981framing,goldstein2008room,schultz2007constructive}.
%
% Hotel guests start to reuse their towels more because of a subtle change of a sign positioned on washroom towel racks \cite{goldstein2008room}. Households start to reduce their energy consumption because of an emoji on their energy bill \cite{schultz2007constructive}.

% People are more willing to sign up for a prosocial peer-to-peer service because of a message on the sign-up page telling them explicitly what benefit they might get \cite{vaish2018s}. Given the susceptible nature of our species, we believe changing the representation of a message can make the message more persuasive.\par


% Comics, a medium to express ideas in a graphical form, are one of the most popular form of art across different cultures. The history of comics can be traced back to early precursors such as Trajan's Column \cite{o1971art}. Comics are usually in the form of juxtaposed sequences of panels of images or a standalone single image. Beside the graphical representation, textual elements such as speech balloons, captions, and onomatopoeia often communicate dialogue, narration, sound effects, or other information. Existing work shows that comics can effectively convey meanings \cite{McDermottPB18,cary2004going,scott1993understanding}. Yet the persuasiveness of the comics representation has not been investigated.


In this study, we want to answer the following research questions related to communication of statistical facts.

\begin{description}
 \item[RQ-1]: Are statistical facts about individual behavior more persuasive when presented through abstract comic form?
 \item [RQ-2]: What the effect of the elements of the comic form, specifically gesture, shading and distance between characters, in modulating comic persuasiveness?
 \item [RQ-3]: What is the effect of information framing in modulating comic persuasiveness?
 \item [RQ-4]: Can we algorithmically synthesize abstract comics for communicating statistical facts?
\end{description}

We used an iterative design framework to answer these questions. After IRB approval, we recruited 200 participants on Amazon Mechanical Turk as subjects in our study. We manipulated character gesture (3 cases), distance between characters (3 cases), shading (3 cases) and framing (2 cases), giving us a total of 54 conditions. We developed a javascript engine to automatically create abstract comic representations given the generative condition (gesture, distance between characters, shading and frame). We analyzed the results using a hierarchical Bayesian framework. Consistent with the observations by~\textcite{Kay2016}, we believe that beyond the role of Bayesian analysis on the issue of replicability, by shifting the question from the binary ``did it have an effect?'' to ``how strong is the effect?'' is especially important in small-$n$ studies common to HCI. Our findings show that comics are more persuasive than text, with a moderate effect size of 0.33. We see strong influence for gestures and shadings that are not neutral, and moderate influence when distance between characters is large. Negatively framed messages have a stronger influence than do positively framed messages. We performed an additional small scale study on Mechanical Turk, to understand the role of color, since comic strips found in newspapers (and XKCD) infrequently use color. Our findings are that the choice of color and which element to color (text, character or background line) matters, but we need a larger study to develop a nuanced understanding connecting color to the other comic elements.

To summarize: our main contributions lie in analyzing the effect of abstract comics in communicating statistical facts, including the effects of the different comic elements. We developed an algorithmic framework to synthesize the comic panels. We plan to release the code for Bayesian analysis and comic generator and the raw data under an appropriate open source license.

We organize the rest of this paper as follows: in the next section, we discuss related work, followed by a section discussing motivating ideas. Then, we present our study design, followed by hierarchical Bayesian analysis. We discuss the implications of our results, including study limitations, followed by conclusions.
