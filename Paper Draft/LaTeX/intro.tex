%!TEX root = cscw2018-comic.tex
Today, the world generates information all around us every second. We are surrounded by all sorts of messages trying to change what we think and what we do: our newsfeed is full of advertisements, our wearable devices are keeping telling us to exercise more, even our water bottle starts to push notification to remind us stay hydrated. However, not all messages can successfully make us change: we won't buy an expensive car because of a short video, we often eat junk food even though we received a lot of articles about health eating and we are often in the status of dehydration after reading those notifications. Thus, how to make a message more persuasive has been a critical problem throughout the years.\par
While classical game theory suggests that recasting an informative message through a different form cannot increase the message's persuasive power, a rich body of research has demonstrated how susceptible we are to those fancy words \cite{tversky1992advances,tversky1981framing,goldstein2008room,schultz2007constructive}. Hotel guests start to reuse their towels more because of a subtle change of a sign positioned on washroom towel racks \cite{goldstein2008room}. Households start to reduce their energy consumption because of an emoji on their energy bill \cite{schultz2007constructive}. People are more willing to sign up for a prosocial peer-to-peer service because of a message on the sign-up page telling them explicitly what benefit they might get \cite{vaish2018s}. Given the susceptible nature of our species, we believe changing the representation of a message can make the message more persuasive.\par
Comics, a medium to express ideas in a graphical form, are one of the most popular form of art across different cultures. The history of comics can be traced back to early precursors such as Trajan's Column \cite{o1971art}. Comics are usually in the form of juxtaposed sequences of panels of images or a standalone single image. Beside the graphical representation, textual elements such as speech balloons, captions, and onomatopoeia often communicate dialogue, narration, sound effects, or other information. Existing work shows that comics can effectively convey meanings \cite{McDermottPB18,cary2004going,scott1993understanding}. Yet the persuasiveness of the comics representation has not been investigated.\par

In this study, we want to answer the following research questions.

\begin{itemize}
    \item RQ-1: What's persuadee's reaction when a plain text persuasive message is synthesized into a piece of comics?
    \item RQ-2: What's the key elements of a persuasive comic that affects persuadee's preferential choice?
    \item RQ-3: Can we algorithmically synthesize a persuasive message into a piece of comics?
\end{itemize}
To answer those question, we investigated the persuasive power of comic style persuasive messages through a filed study with 200 participants on Amazon Mechanical Turk. First, we reviewed some of the previous works on persuasive technologies in three major fields: (1) current development of persuasive technology, (2) ways to approach computational persuasion, and (3) communicating through comic. Second, we discuss about the general composition of a comic message and its comparison with plain text. Lastly, to examine our hypothesis that persuasive messages in the comic form are preferred over plain text, we conduct a field experiment to investigate the persuasiveness of comic messages comparing with text messages. The result shows comics can be a better form of persuasive message comparing to plain text and three key comic elements, character gesture, inter-character distance, and background shading moderate the persuasiveness differently.\par

The contributions of this work are: 1) among the first to examine the effect on persuadee's preferential choice for comic form messages. 2) exploring the key elements in comic form persuasive messages. 3) providing a novel framework to approach computational persuasion by algorithmically synthesizing persuasive message into comic form. \par
