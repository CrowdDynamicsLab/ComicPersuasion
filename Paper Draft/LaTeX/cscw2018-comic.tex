\documentclass[format=acmsmall, natbib=false, review=false, anonymous=true, screen=true]{acmart}

% using biblatex
\let\citename\relax
\RequirePackage[abbreviate=true, dateabbrev=true, natbib=true, isbn=false, doi=false, urldate=comp, url=false, maxbibnames=9, backref=false, backend=biber, style=ACM-Reference-Format, language=american]{biblatex}

\addbibresource{sample.bib}
\renewcommand{\bibfont}{\Small}

\usepackage{booktabs} % For formal tables

\usepackage[ruled]{algorithm2e} % For algorithms
\renewcommand{\algorithmcfname}{ALGORITHM}
\SetAlFnt{\small}
\SetAlCapFnt{\small}
\SetAlCapNameFnt{\small}
\SetAlCapHSkip{0pt}
\IncMargin{-\parindent}


% Metadata Information
% \acmJournal{TWEB}
% \acmVolume{9}
% \acmNumber{4}
% \acmArticle{39}
% \acmYear{2010}
% \acmMonth{3}
% \copyrightyear{2009}
%\acmArticleSeq{9}

% Copyright
%\setcopyright{acmcopyright}
\setcopyright{acmlicensed}
%\setcopyright{rightsretained}
%\setcopyright{usgov}
%\setcopyright{usgovmixed}
%\setcopyright{cagov}
%\setcopyright{cagovmixed}

% DOI
\acmDOI{0000001.0000001}

% Paper history
\received{February 2007}
\received[revised]{March 2009}
\received[accepted]{June 2009}


% Document starts
\begin{document}
% Title portion. Note the short title for running heads
\title[Persuasion in Comic form]{It's Comical: When Facts in Comic Form Persuade}


\author{Ziang Xiao}
\email{zxiao5@illinois.edu, hs1@illinois.edu}
\author{Po-Shiun Ho}
\email{pho11@illinois.edu}
\author{Hari Sundaram}
\email{hs1@illinois.edu}
\affiliation{%
  \institution{University of Illinois}
  \department{Computer Science}
  \city{Urbana}
  \state{IL}
  \postcode{61801}
  \country{USA}}
  \author{Xinran Wang}
  \email{wangxr1108@126.com}
  \affiliation{%
    \institution{Tsinghua University}
    \department{Computer Science}
    \city{Beijing}
    \postcode{100084}
    \country{China}}




\begin{abstract}
Multifrequency media access control has been well understood in
general wireless ad hoc networks, while in wireless sensor networks,
researchers still focus on single frequency solutions. In wireless
sensor networks, each device is typically equipped with a single
radio transceiver and applications adopt much smaller packet sizes
compared to those in general wireless ad hoc networks. Hence, the
multifrequency MAC protocols proposed for general wireless ad hoc
networks are not suitable for wireless sensor network applications,
which we further demonstrate through our simulation experiments. In
this article, we propose MMSN, which takes advantage of
multifrequency availability while, at the same time, takes into
consideration the restrictions of wireless sensor networks. Through
extensive experiments, MMSN exhibits the prominent ability to utilize
parallel transmissions among neighboring nodes.
\end{abstract}


%
% The code below should be generated by the tool at
% http://dl.acm.org/ccs.cfm
% Please copy and paste the code instead of the example below.
%
% \begin{CCSXML}
% <ccs2012>
%  <concept>
%   <concept_id>10010520.10010553.10010562</concept_id>
%   <concept_desc>Computer systems organization~Embedded systems</concept_desc>
%   <concept_significance>500</concept_significance>
%  </concept>
%  <concept>
%   <concept_id>10010520.10010575.10010755</concept_id>
%   <concept_desc>Computer systems organization~Redundancy</concept_desc>
%   <concept_significance>300</concept_significance>
%  </concept>
%  <concept>
%   <concept_id>10010520.10010553.10010554</concept_id>
%   <concept_desc>Computer systems organization~Robotics</concept_desc>
%   <concept_significance>100</concept_significance>
%  </concept>
%  <concept>
%   <concept_id>10003033.10003083.10003095</concept_id>
%   <concept_desc>Networks~Network reliability</concept_desc>
%   <concept_significance>100</concept_significance>
%  </concept>
% </ccs2012>
% \end{CCSXML}
%
% \ccsdesc[500]{Computer systems organization~Embedded systems}
% \ccsdesc[300]{Computer systems organization~Redundancy}
% \ccsdesc{Computer systems organization~Robotics}
% \ccsdesc[100]{Networks~Network reliability}

%
% End generated code
%


\keywords{Wireless sensor networks, media access control,
multi-channel, radio interference, time synchronization}




\maketitle

% The default list of authors is too long for headers.
\section{Introduction}
%!TEX root = cscw2018-comic.tex
Today, the world generates information all around us every second. We are surrounded by all sorts of messages trying to change what we think and what we do: our newsfeed is full of advertisements, our wearable devices are keeping telling us to exercise more, even our water bottle starts to push notification to remind us stay hydrated. However, not all messages can successfully make us change: we won't buy an expensive car because of a short video, we often eat junk food even though we received a lot of articles about health eating and we are often in the status of dehydration after reading those notifications. Thus, how to make a message more persuasive has been a critical problem throughout the years.\par
While classical game theory suggests that recasting an informative message through a different form cannot increase the message's persuasive power, a rich body of research has demonstrated how susceptible we are to those fancy words \cite{tversky1992advances,tversky1981framing,goldstein2008room,schultz2007constructive}. Hotel guests start to reuse their towels more because of a subtle change of a sign positioned on washroom towel racks \cite{goldstein2008room}. Households start to reduce their energy consumption because of an emoji on their energy bill \cite{schultz2007constructive}. People are more willing to sign up for a prosocial peer-to-peer service because of a message on the sign-up page telling them explicitly what benefit they might get \cite{vaish2018s}. Given the susceptible nature of our species, we believe changing the representation of a message can make the message more persuasive.\par
Comics, a medium to express ideas in a graphical form, are one of the most popular form of art across different cultures. The history of comics can be traced back to early precursors such as Trajan's Column \cite{o1971art}. Comics are usually in the form of juxtaposed sequences of panels of images or a standalone single image. Beside the graphical representation, textual elements such as speech balloons, captions, and onomatopoeia often communicate dialogue, narration, sound effects, or other information. Existing work shows that comics can effectively convey meanings \cite{McDermottPB18,cary2004going,scott1993understanding}. Yet the persuasiveness of the comics representation has not been investigated.\par

In this study, we want to answer the following research questions.

\begin{itemize}
    \item RQ-1: What's persuadee's reaction when a plain text persuasive message is synthesized into a piece of comics?
    \item RQ-2: What's the key elements of a persuasive comic that affects persuadee's preferential choice?ß
\end{itemize}
To answer those question, we investigated the persuasive power of comic style persuasive messages through a filed study with 200 participants on Amazon Mechanical Turk. First, we reviewed some of the previous works on persuasive technologies in three major fields: (1) current development of persuasive technology, (2) ways to approach computational persuasion, and (3) communicating through comic. Second, we discuss about the general composition of a comic message and its comparison with plain text. Lastly, to examine our hypothesis that persuasive messages in the comic form are preferred over plain text, we conduct a field experiment to investigate the persuasiveness of comic messages comparing with text messages. The result shows comics can be a better form of persuasive message comparing to plain text and three key comic elements, character gesture, inter-character distance, and background shading moderate the persuasiveness differently.\par

The contributions of this work are: 1) among the first to examine the effect on persuadee's preferential choice for comic form messages. 2) exploring the key elements in comic form persuasive messages. 3) providing a novel framework to approach computational persuasion by algorithmically synthesizing persuasive message into comic form. \par

\newpage
\section{Related Work}
%!TEX root = proceedings.tex

Research on persuasive messages in HCI has been increasingly popular due to the fact of information explosion. Previous research has provided solid strategies that can make text messages more persuasive, including messaging framing, information-centric approaches, and personalize information context []. However, due to the advance in technology, we are immersed in the information ocean and overwhelmed with texts various sources. How to make individual message stand out among others and attract reader's attention while nudging the reader to make behavioral and attitude changes becomes an important problem. In this study, we looked into a novel form of persuasive message, visual comics. Then, we investigated the persuasive power of different visual cues in comics beyond pure text. In the following sections, we will introduce related studies categorized into groups: (1) Framing text-form persuasive messages, (2) Persuasion beyond text, and (3) Communicating through comics.\par

\subsection{Building Persuasive Technology}
Starting from Goehlert's discussion on persuasion and communications technology in 1980, HCI researchers have spent a lot of effort in leveraging technology in persuasion. Goehlert argues control and dissemination of information have the ability to make attitude and behavioral change. Inspired by this argument, two types of approaches have been used in constructing persuasive system. Information-centric approaches focused on delivering information which did not perceived or recognized by the user before. For example, Chi et al. created an intelligent kitchen that can provide nutritional information about ingredients while users are cooking. People start to adjust their ingredients usages to achieve a better nutritional composition. Waterbot persuades people to engage water-saving behavior by augmenting physical sink interface with water usage indicators.
Liao and Fu found showing a source expertise indicator can shape user's information seeking behavior and burst the filter bubble. While a lot of studies on persuasive technology evident the persuasive power of this information-centric approach, pervious studies also show the downside of information-center approach where the target receiver often failed to perceive and rationalize the persuasive information when the receiver is experiencing information overload. Additionally, information-centric approach often relays on the target receiver can make rational decision based on provided information whereas the premise of rationality does not always hold.\par

Adapting decision-making models from previous behavioral research, some persuasive technology emphasized on human motivation and biases, e.g. behavior-centric approach []. Lee et al. incorporated the idea of default bias in the design of the Snackbot robot and successfully persuade people with healthy snacking in the workplace. Vaish et. al used self-serving motivational framing of messages to persuade people to sign up for a prosocial peer-to-peer service. Borrowing from the theory of planned behavior, Scheneider et. al understood different motivational factors of mobile fitness coach users and delivered individualized messages to persuade users based on their motivations. Although the effectiveness of behavior-centric approach has been examined in multiple studies, behavior-centric approaches often requires prior knowledge of target subject in order to maximize the persuasion power, as Orji et. al and Schneider et. al suggested in their study that different people may be more valuable to one persuasive method than others.  Therefore, the common challenging faced by behavior-centric approach is scalability.\par

Our study, we tried to incorporate both information-centric and behavior-centric approaches. The comic is designed to show user's behavior stats (information-centric) but the message is framed beyond raw representation in order to achieve maximum persuasiveness (behavior-centric). The challenge here is mitigating the down sides of information-centric and behavior-centric approaches. We used novel representation, visual comic, to catch information receiver's attention and developed a tool that has the ability to algorithmically generate personalized persuasive comics.
\par

\subsection{Text Messages Persuade}
Approaches to make a message more attractive and persuasive has long been a focus for a variety of different fields including computational linguistic, social networking, and advertising. Among these tactics, message framing is one of the most basic and intuitive methods to generate memorable and persuasive messages. This is mainly because message framing and phrasing generally doesn't require additional information or data visualization. Its simplicity contributes to the various researches on the effect of message framing in memorability. Mizil et.al shows that using unusual word choices and more general theme makes it easier to connect with reader's daily life and makes the message more memorable. Using unexpected words and phrases, the message is more likely to capture reader's attention comparing to normal phrases. As for the theme of the message, it should be as general as possible to help readers connect with the message, so it can stay in the reader's mind longer.\par

Tversky and Kahneman takes on a different aspect of the effect of framing. Instead of focusing on the word selection and sentence structure of a quote, Tversky and Kahneman focused on how different reference point of a same sentence can result in reader's different response. It shows that variations of reference point of a decision can determine whether the reader will evaluate it as a gain or a loss, thus changing their decision. For example, choices involving gains are often risk averse and choices involving losses are often risk taking. Meanwhile, implying social norm through the message is another persuasive technique.  Goldstein et. al conducted an interest experiment in the hotel on motivating environmental conservation. They found employing descriptive norms (e.g., "the majority of guests reuse their towels") has more persuasive power than solely mentioning environmental protection. And this normative message gets more persuasiveness when the described setting is closer to individuals' immediate situational circumstances (e.g., "the majority of guests in this room reuse their towels").  All those techniques are essential for our design of messages to persuade behavioral change.\par


\subsection{Communicating through Comics}
The simple and humorous nature of comic make it becomes an excellent media for delivering informative and memorable messages. While reading comics book is commonly recognized as entertaining, comics has been examined as an effective way of communicating abstract ideas to broad audiences []. Bromberg et. al used comics to illustrate complex scientific facts. In the area of education, comics have been used and examined as an effective tool for reaching different populations with various background []. Meanwhile, the common usage of metaphor in comics can make the underlying meaning for vivid and therefore more memorable than using a straightforward description[]. Moreover, simple comics can express emotion and create empathy for readers. Lima Sanches et al. showed that there is a link between comic’s content and the emotions felt by the readers. Thus, with the form of comics, complex message can be easily interpreted and memorized. Given the advantage of using comics to deliver meanings, our study took one step further and considered the persuasive side of comic representation. \par

However, using comics to persuade is challenging. First of all, generating comics is not easy. Especially, persuasive message should be personalized to deliver maximum persuasive power. Traditionally, comics is created by professional cartoonists which is very costly to produce personalized comics.  Although prior work has explored methods of algorithmically generating comics, no existing method is for generating persuasive comics. \par

\newpage
\section{Method}
%!TEX root = proceedings.tex

To test our hypotheses, we designed and conducted a field study through Amazon Mechanical Turk. In the experiment, participants will see five persuasive messages in both plain-text form and comic form side by side. Then, the participant will be asked which form of the message is perceived as more persuasive and how persuasive is it.
\subsection{Persuasive Messages}
All persuasive messages were focused on motivating exercise behavior. Persuasive messages were constructed by providing descriptive normative information, e.g. "In the past week, you spent more time at the gym than did 65\% of your friends", or descriptive exercise data, "Congrats! You have reached your goal of exercising three times a week." Since existed research suggests the persuasive power of a normative messages may differ between receivers who did not meet the social norm and receivers who outperforms the social norm, we also constructed two set of messages that targeting receivers who meet descriptive norm (Positive framed messages) and others who fail (Negative framed messages) by negating some key words (Less vs. More/ Bottom vs. Top/ Reached vs. didn't Reached). In our study, five persuasive messages were either all positive framed or all negative framed.\par
The followings messages were presented in our study.\par
\textit{Positive framed messages:}
\begin{enumerate}
  \item In the past week, you spent more time at the gym than did 65\% of your friends
  \item Congrats! You have reached your goal of exercising three times a week.
  \item Over the past month, you exercised more than did 90\% of your friends.
  \item Your exercise activity is in the top 20\% of all your friends.
  \item Over the past three weeks, you went to the gym more often than 60\% of your friends did.
\end{enumerate}\par
\textit{Negative framed messages:}
\begin{enumerate}
\item	In the past week, you spent less time at the gym than did 65\% of your friends
\item Congrats! You did not reach your goal of exercising three times a week.
\item	Over the past month, you exercised less than did 90\% of your friends.
\item	Your exercise activity is in the bottom 20\% of all your friends.
\item	Over the past three weeks, you went to the gym less often than 60\% of your friends did.
\end{enumerate}\par
\subsection{Comic Style Messages Generation}
\subsection{Question Design}
\textit{Display Order}. To mitigate any potential biases toward the display order of the plain-text representation and comic-style representation. The display order is randomly assigned, which means both plain-text representation and comic-style representation have equal chance to be displayed on the left side or the right side of the comparison.\par
\textit{Attention Checker}. To control the data quality, we embedded two attention checkers in the experiment. The first one appears after the third comparison and the second one shows up after the last comparison. Both attention checkers asked subjects to choose a comic that matches a simple description, e.g. "Which of the following comics has two characters?".\par
\textit{Scale Design}. The goal of the scale design is to monimize any potential ambiguity and biases.The scale in this study is a 7-point binary adjective items scale with Neutral at the middle. Two ends of the scale indicates the strongest perference toward Plain-Text representation or Comic-style representation.
At the first iteration of the design, the direction of the scale changes corresponding to the display order of the two representations.
However, in our pilot testing of the scale, participants reported that the scale was confusing because of the direction change. Some participants even did not notice the direction of the scale is corresponding to the messages, instead, they treated it as a fixed direction scale based on their first impression.Therefore, we changed our scale to a fixed direction scale. To avoid any potential demand characteristics introduced by the scale, e.g. the large number is desired or selecting item on the right side is expected, our final design does not contain any number, and the direction of the scale is randomly assigned for each participant with either plain-text always on the left or comic-style message always on the left.\par
\subsection{Participants Recruitment}
We published our HITs on Amazon Mechanical Turk titled with "A short survey about your exercise motivation". The price tag for each HIT was \$0.50, which were the rewards the workers would get regardless of their performance. On the HIT page, participants would see a link to our experiment site and a text input box for them to enter a six-digit completion code. Repeated worker will be rejected as we instructed in the task description.

\newpage
%!TEX root = cscw2018-comic.tex

\section{Results}
\label{sec:Results}

\subsection{Raw Data}
\label{sub:Raw Data}
In this section we describe the raw data counts, the number of participants, the number of people whose responses we dropped. The final number of observations

\subsection{Bayesian Model}
\label{sub:Bayesian Model}
We use a Bayesian formulation of the problem of identifying suitable predictors for the messages in comic form.~\textcite{Kay2016} provide an nice introduction on the appropriateness of Bayesian analysis for the HCI community. Bayesian analysis is attractive in our experiment due to two advantages: shifting the conversation from ``did it work'' to ``how strong is the effect''; and benefits to small $n$ studies.

We manipulate five independent variables: gesture of the participants in the comic (3: neutral, moderate, extreme);  distance between the two characters (3:close, moderate, far); comic shading (3:white, light gray, gray); framing (2: whether the information was positively framed or negatively framed).  This gives us a total of $3 \times 3 \times 3 \times 2= 54$ experimental conditions. As a control against ordering effects, we randomly manipulate comic position (whether we presented the comic panel to the left or to the right).  Thus, we need to estimate the effect on the responses for each of these variables; the responses are on a 7 point Licket scale.

A challenge with using ordinal scale such as the Lickert scale: we do not know the ``width'' of each response. That is, while we may know that for example $1<2<3$, we don't know if the difference in the thresholds used by subjects to mark ``2'' on the scale, is the same as the difference in thresholds they use for ``1'' and ``3.''  We assume that each response by a subject: lies in a continuous metric space; is Normally distributed; and that the thresholds $\{\theta_i\}$ while unknown, are shared—all subjects use same set of thresholds to identify the appropriate ordinal value.

Formally let $z$ be the response of the subjects to the experiment where the comic panel was generated by the different conditions; each condition is obtained by setting each of the $k$ independent variables $\{x_j\}, j \in [1 \ldots k]$. The subjects first generate a Normally distributed metric variable $y$, and then use the thresholds $\{\theta_i\}$ to map $y$ to the ordinal variable $z$.

Then, since we assume that the metric variable $y$ is Normally distributed, our hierarchical Bayesian model is defined as follows (see~\Cref{fig:generative-main} for a graphical representation):

\begin{align}
 y                  & \sim N(\mu, \sigma_y)                    \label{eq:response-main}                       \\
 \sigma_y           & \sim U(L, H), \label{eq:main-sigma}                                                     \\
 \mu                & \sim \beta_0 +
 \underbrace{\sum_{j} \beta_{1,j} x_{1,j}(i)}_{\text{gesture}} +
 \underbrace{\sum_k \beta_{2,k} x_{1,k}(i)}_{\text{shading}} +
 \underbrace{\sum_l \beta_{3,l} x_{1,l}(i)}_{\text{distance}} +
 \underbrace{\sum_m \beta_{4,m} x_{1,m}(i)}_{\text{framing}} ,                  \label{eq:mu-main} \\
 \sum_j \beta_{i,j} & = 0, \qquad  \qquad  \quad \, i \in \{1, \ldots, 4\}, \label{eq:beta-equality}          \\
 \beta_{i,j}        & \sim N(0, \sigma_{\beta, i}), \qquad  i \in \{1, \ldots, 4\},\label{eq:main-beta-sigma} \\
 \sigma_{\beta, i}  & \sim \Gamma(s, r ). \label{eq:gamma-distribution}
\end{align}
\Cref{eq:response-main} says that the metric variable $y$ is Normally distributed\footnote{The ``$\sim$'' symbol means that the random variable on the left is drawn from the probability distribution on the right.} with mean $\mu$ and standard deviation $\sigma_y$.~\Cref{eq:mu-main} says that the mean response $\mu$ is a linear weighted combination of the predictors.~\Cref{eq:main-sigma} says that the standard deviation of the response is drawn from a Uniform distribution with constant parameters $\text{Low}=L, \text{High}=H$, where $L>0, H\gg L$.~\Cref{eq:main-beta-sigma} says that the predictor weight is drawn from a Normal distribution with $\mu=0$ and standard deviation $\sigma_{\beta, i}$. That is,  while \textit{each} predictor set $\beta_{i}$ is drawn from a \textit{different} Normal distribution, the $\beta_{i,j}$ values within the same predictor set $\beta_{i}$ are drawn from the \textit{same} Normal distribution.~\Cref{eq:beta-equality} says that the sum of the deflections $\sum_k \beta_{j,k}$ from the mean for any nominal predictor $j$ equals zero.~\Cref{eq:gamma-distribution} says that we draw all the variances $\sigma_{\beta, i}$ from a Gamma $\Gamma(s,r)$ distribution, where $s$ refers to the shape parameter and $r$ refers to the rate parameter. We set the variables $s,r$ to allows a wide range of values for $\sigma_{\beta, i}$.

By drawing the standard deviation variables $\sigma_{\beta, i}$ from a Gamma distribution, the values of each element of $\beta_i$ informs the other elements. Notice that we draw the variances $\sigma_{\beta, i}$ of each predictor $i$ from an independent Gamma distribution, implying that the variances (equivalently, the extent of deflections from the mean) for each predictor can be different. Furthermore, the ``information sharing'' among variables is common to hierarchical Bayesian models and is an important reason why Bayesian models work so well with small datasets\footnote{The sharing of information causes the variance of each individual element $\beta_{i,j}$ to move towards the group variance, a phenomena known as ``shrinkage.'' }. The main advantage of using a Gamma distribution is that we can specify a non-zero mode, important in controlling shrinkage in hierarchical models.

\begin{figure}
 \includegraphics[width=\textwidth]{./figures/generative_model.pdf}
 \caption{The figure shows the hierarchical Bayesian model specification, corresponding to~\Crefrange{eq:response-main}{eq:gamma-distribution}.}
 \label{fig:generative-main}
\end{figure}

Thus far, we have discussed how to generate a Normally distributed metric variable $y$. However, what we see in the experiment is not this metric variable, but an ordinal variable $r$. The subjects use internal thresholds $\{\theta_i\}$ to determine when to ``strongly disagree'', ``disagree'' etc. Thus with a 7 point Likert scale, we have 6 thresholds. The probability that we will see an ordinal response $r=k$ is $P(r=k | \mu, \sigma, \{\theta_i\})$, where, $\{\theta_i\}$ is the set of thresholds used by the subjects. We assume that while these thresholds are unknown, all subjects use the same thresholds. Since the underlying metric response $y$ is Normally distributed, we can compute the probabilities for observing each ordinal response $r=k$ as follows:

\begin{equation}
 P(r=k | \mu, \sigma, \{\theta_i\}) = \Phi \left (\frac{\theta_k - \mu}{\sigma} \right) - \Phi \left(\frac{\theta_{k-1} - \mu}{\sigma} \right).
\end{equation}
Where, $\Phi$ represents the cumulative density function for the Normal distribution corresponding to the underlying metric variable $y$. In other words, the probability that we will see ordinal response $k$ is the area under the Normal distribution with parameters $\mu, \sigma$ between thresholds $\theta_{k-1}$ and $\theta_k$.

The thresholds $\theta_i, i \in \{2, 3, 4, 5\}$ have two degrees of freedom in that a simple translation of the response will translate the thresholds. Consistent with~\textcite[][p. 674]{Kruschke2014}, we set $\theta_1\equiv1.5$ and $\theta_6\equiv6.5$, leaving us with four hidden threshold parameters. We draw these remaining four $\{ \theta_i\}$ from a Normal distribution as follows:
\begin{equation}
 \theta_i \sim N(i+0.5, 1/2), i \in \{2, 3, 4, 5\}.
\end{equation}

In this section, we developed a hierarchical Bayesian formulation to model the subject ordinal response to analyze the effect of different predictors for three comic elements (gesture, distance, shading) and information framing. We have a total of $54$ experimental conditions. We also modeled the thresholds for the different ordinal outcomes as a hidden variable. In the next section, we present and analyze the results.

\subsection{Analysis}
\label{sub:Analysis}

We analyzed the data using PyMC3~\cite{Salvatier2016}, a popular framework for Bayesian inference. Computational techniques for Bayesian inference use a stochastic sampling technique called Markov Chain Monte Carlo (MCMC) that samples the posterior distribution $P(\theta | D)$, where we want to estimate the parameters $\theta$ given the observations $D$. In particular, we used the Metropolis-Hastings sampler. The Gelman-Rubin statistic $\hat{R}$ was around 1, indicating that the different sampling chains converged. The modal values of the coefficients are as follows:

\begin{table}[htb]%\footnotesize
 \centering
 \caption{Modal coefficient values $\beta_{0-4}$. Some coefficients are vectors as they represent the displacement from the mean for different conditions of that variable. For example, since gesture has three experimental conditions, $\beta_1$ is a vector of length 3.}\label{tab:modal values}
 \begin{tabular}{@{}rl@{}} \toprule
  Coefficient                     & Values                     \\ \midrule
  Intercept ($\beta_0$)           & $4.524$                    \\
  Gesture ($\beta_1$)             & $[-0.218, 0.101, 0.118]$   \\
  Shading ($\beta_2$)             & $ [-0.218 , 0.101, 0.118]$ \\
  Distance ($\beta_3$)            & $[-0.043, -0.057, 0.099]$  \\
  Framing ($\beta_4$)             & $[ 0.051, -0.051]$         \\
  Standard Deviation ($\sigma_y$) & $1.614$                    \\ \bottomrule
 \end{tabular}
\end{table}


\begin{figure}
 \subfloat[The mean effect and the effect size\label{subfig-1:mean-effect}]{%
  \includegraphics[width=0.6\textwidth]{./hari-code/factors_mean_effect_main-noint.pdf}
  } \hfill
 \subfloat[Information framing contrast\label{subfig-2:framing}]{%
  \includegraphics[width=0.33\textwidth]{./hari-code/factors_framing_contrasts_main-noint.pdf}
 }
 \caption{~\Cref{subfig-1:mean-effect} showsHigh Posterior Density (HPD) intervals for the mean response $\mu$ and effect sizes $\sigma_y$. HPD represent the region with 95\% of the density. Notice that the HPD interval for $\mu$ is $[4.35, 4.70]$ and excludes 4 (the neutral response value), implying that on average, the response to the comic panel was more persuasive than the text. The figure for effect size shows a moderate effect with mode $0.33$; since the HPD interval $[0.21, 0.44]$ excludes 0, we can be confident about the effect.~\Cref{subfig-2:framing} shows the contrasts between negatively framed message with a positively framed message. The modal value is $0.08$, but since the HPD interval $[-0.12, 0.36]$ overlaps with 0, there is no appreciable effect (interestingly 80\% of the density lies in the region greater than 0.)}
 \label{fig:main-experiment-effect}
\end{figure}

\begin{figure}
 \includegraphics[width=\textwidth]{./hari-code/factors_gesture_contrasts_main-noint.pdf}
 \caption{gesture contrasts}
 \label{fig:gesture-contrasts-main}
\end{figure}

\begin{figure}
 \includegraphics[width=\textwidth]{./hari-code/factors_shading_contrasts_main-noint.pdf}
 \caption{shading contrasts}
 \label{fig:shading-contrasts-main}
\end{figure}

\begin{figure}
 \includegraphics[width=\textwidth]{./hari-code/factors_distance_contrasts_main-noint.pdf}
 \caption{distance contrasts}
 \label{fig:distance-contrasts-main}
\end{figure}

\newpage
\section{Discussion}

\printbibliography
% \bibliographystyle{ACM-Reference-Format}
% \bibliography{sample}

\end{document}
