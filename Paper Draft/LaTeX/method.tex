%!TEX root = cscw2018-comic.tex
In this study, we composed persuasive messages in both plain text and comic representations to compare the difference in their persuasive power and potential elements in the comics that may influence the persuasiveness. The main goal of all messages are the same, persuading readers to engage more exercise. The main reason to choose this goal are 1) As a basic daily activity everyone has the need to exercise. 2) People can easily understand the message and relate to themselves. 3) Engaging more exercise is mostly based on audience own willingness instead of any other subjective resources.\par
\subsection{Composing Persuasive Messages in Plain Text}
Borrowing the idea from Psychology and Behavioral Economics, the key persuasive technique we adopted here is implying social norm through the messages (e.g. how participant's friends are doing). Also, we incorporate the idea from Tversky and Kahneman that people will be influenced differently if the same message is framed as risk averse or risk taking. Therefore, we created two sets of messages that either framed from a positive standpoint or a negative standpoint. \par
The followings messages were presented in our study.\par
\textit{Positive framed messages:}
\begin{enumerate}
 \item In the past week, you spent more time at the gym than did 65\% of your friends
 \item Congrats! You have reached your goal of exercising three times a week.
 \item Over the past month, you exercised more than did 90\% of your friends.
 \item Your exercise activity is in the top 20\% of all your friends.
 \item Over the past three weeks, you went to the gym more often than 60\% of your friends did.
\end{enumerate}\par
\textit{Negative framed messages:}
\begin{enumerate}
 \item	In the past week, you spent less time at the gym than did 65\% of your friends
 \item Congrats! You did not reach your goal of exercising three times a week.
 \item	Over the past month, you exercised less than did 90\% of your friends.
 \item	Your exercise activity is in the bottom 20\% of all your friends.
 \item	Over the past three weeks, you went to the gym less often than 60\% of your friends did.
\end{enumerate}

\subsection{Communicating through Comics}
As a form of art, the creation of comics has few limitations. Although there is no common template that could describe all comics, if we take a closer look at each comic, it is not hard to see that every comic consists of several fundamental components. We categorize these comic elements into different groups:
\begin{enumerate}
 \item	characters,
 \item gesture,
 \item	background color/shading,
 \item	word bubble,
\end{enumerate}\par
In this study, to represent persuasive messages in a comic form, we need to determine each of those four parameters.\par
As Scott McCloud mentioned, the reader is more likely to project him/herself onto the character in the comic when the comic getting abstract [citation]. By taking the perspective of the character, the reader will internalize the information his/her character trying to express or receive. If the information is persuasive, the internalization will imply a higher chance of expected behavior change. Therefore, in this study, we choose to use an abstract yet well-recognized comic style, the xkcd style created by Randall Munroe, in our generated persuasive comic messages.\par
Beside the abstractness of the character, we believe the relationship between characters is also important. In real world, previous research suggests that messages are more persuasive if the person communicating the ideas is someone the receiver related [citation]. People are more likely to believe their close friends than strangers [citation]. In abstract comics, the relationship between characters is usually modeled by the distance between characters [citation]. So, it is reasonable to believe the link still holds in the world of comics as the reader tends to project his/herself onto the character. Therefore, we hypothesized that the distance between characters in a comic may influence the persuasive power.\par
The gesture of a character is another important component in the comic. The gesture of a character can help reader to understand what happens and the emotion of the character. Different gestures also imply the intensity of an emotion. As a common technique, cartoonists often use the gesture to intensify the feeling that they want to express to the reader. For a persuasive message, the intensified emotion may make the message more memorable than a plain tone. Thus, in this study, the gesture is another key element that we believe may moderate the effectiveness of a comic message.\par
A rich body of research has demonstrated the relationship between color or background shading and the emotion. In comics, the color of elements or the background shading contributes significantly to the feelings as well. However, as xkcd style are mostly seen in black and white, we suspect and confirmed in our pilot study that color backgronud does not go well with generated comics (see pilot test 1). In the main study, we only manipulate the background shading in gray scale to show its affect on the persuasiveness. \par
The word bubble is the most common place in comics to incorporate text information. In a persuasive comic, the word bubble expresses the text content of the message.  \par

\subsection{Generating Comics}
In this study, we create a based template for all comic messages that includes two characters in a conversation and the scenario is 'One day, your friend has something to tell you.'\par

We developed an algorithmic comic generator that based on Comix I/O, an open source project that creates comics with stick figures using HTML markup. To maximizing the flexibility, we further developed the existing build with Canvas and rough.js. The generator allows us to create the comic representation of a persuasive message with variations in character’s gesture, inter-character distance and background shading.\par
For gesture, we created the gesture library in a JSON format with two main categories: positive and negative corresponding to how the original message is framed, each with three levels of intensity. \par
For the distance between two character, we have three levels of variance from close to far as well. The three level of distance represents the relationship between to characters as close friends, friends, acquaintance, from close to far. \par
For the background shading, we have a total of three levels from white to dark grey. Each level represents a level of emotional intensity, white as lowest. \par
To make a fair comparison between plain text representation and the comic form, the content of text information is the same in both conditions.  \par
In total, for each message we have a total of 27 variations in terms of character gesture, inter-character, and background shading. In this study, 270 comics was created corresponding to 10 plain text messages. \par

\subsection{Study Design}
To test our hypotheses, we designed and conducted a between-subject field study through Amazon Mechanical Turk. In the experiment, participants will see a total of five persuasive messages in both plain-text form and comic form side by side. Then, the participant will be asked which form of the message is perceived as more persuasive and how persuasive is it.\par
Once participants agreed to join our study, they will be randomly assigned to two conditions 1) positive message condition where all persuasive messages are framed in a positive way and 2) negative condition where all persuasive messages are framed in a negative way. In both condition, Participants will compare five persuasive messages.\par
For each persuasive message, the participant will compare the same message in plain text and in a comic form on a 7 item Likert scale. \par
\textit{Display Order} To mitigate any potential bias toward the display order of the plain-text representation and comic-style representation. The display order is randomly assigned. Both plain-text representation and comic-style representation have equal chance to be displayed first on the left side.\par
\textit{Attention Checker} To control the data quality, we embedded two attention checkers in the experiment. The first one appears after the third comparison and the second one shows up after the last comparison. Both attention checkers asked subjects to choose a comic that matches a simple description, e.g. "Which of the following comics has two characters?"\par
Rating Scale Design. The 7 item Likert scale is ranging from -3 (left) to 3 (right) where 0 means neutral. The direction of the scale flips corresponding to the position of the comic and plain text. \par
\subsection{Pilot Tests on the Study Design}
Before the actual experiment, we first tested our design in a small scale. We deployed our study on Amazon Mechanical Turk and received total of 10 feedbacks in 5 hours. Surprisingly, 5 of 10 participants reports the scale is confusing. 3 of them reports the flipping scale makes they spend more time on figuring out which side is which. One participants said "When I am answering my last question, I suddenly noticed the scale is different. I noticed the order of text and graph is changing but never notice the scale is changing as well. ``That's really confusing!!''


Based on the valuable feedback from our pilot test, we reiterated our study design. In the final design, we fixed the order of the rating scale. However, since the order is fixed, potential demand characteristics may be introduced by the scale, e.g. the researcher may want me to choose the larger number or items on the right/left side is expected. To minimize potential biases, in our final design, another layer of randomness was added.  The direction of the scale no longer changes respect to the position of the messages, but the direction of the scale is randomly assigned for each participant. Also, the number on the scale is replaced by text as neutral, slightly persuasive, text/comics is more persuasive and strongly persuasive.
\subsection{Participants}
We published our HITs on Amazon Mechanical Turk titled with "A short survey about your exercise motivation". The price tag for each HIT was \$0.50, which were the rewards the workers would get regardless of their performance. The threshold for participant to join our study is a 95\% Approval Rate. On the HIT page, participants would see a link to our experiment site and a text input box for them to enter a six-digit completion code. Repeated worker will be rejected as we instructed in the task description.
