%!TEX root = cscw2018-comic.tex
\section{Conclusion}
\label{sec:Conclusion}

This paper examined how algorithmically synthesized abstract comics were more persuasive for communicating statistical facts than were the corresponding plain text. Furthermore, we examined how elements of the comic---gesture, inter-character distance and shading---influenced the effect of the comic. We also examined the effect of message framing on the ability of the comic to persuade. Three ideas were key to our work: ``social proof'' (that we adopt behaviors of our friends or people similar to us), ``information framing'' (individuals have asymmetric utility functions), and the role of abstract comics to allow readers to project themselves onto the comic. We performed an empirical study to answer our research questions by recruiting subjects on Amazon Mechanical Turk. We analyzed the results using a hierarchical Bayesian framework that allows for understanding effects, as well are helpful in small-$n$ studies. The results show that comics have a clear but moderate effect (effect size: 0.33) on persuasiveness compared to text. While no comic element has significant effect, we can observe that non-neutral gestures and shading as well as negatively framed messages have a strong influence. We conducted a smaller study with color and the result indicates that color too has a strong influence. Finally, we developed an abstract comic panel generator that takes as input the different comic elements, the valence of the frame and the message. We plan to release the code for Bayesian analysis and comic generator and the raw data under an appropriate open source license.

As next steps, we plan to further analyze interaction effects between comic elements and color as well as examine theoretical frameworks for generating the character gesture. A larger goal is to focus on storytelling, where individuals receive single comic panels over time, but the panels are connected with a storyline.
