%!TEX root = proceedings.tex

This paper looks at the ability of algorithmically synthesized comic-style messages to persuade individuals to adopt behaviors. In collective action dilemmas including climate change and public heath, persuasive messages are an important aspect of enabling individuals to adopt behaviors that improve their utility as well as benefit the larger group as a whole. Whether comic representations despite widespread use in popular culture, offer any tangible benefits over plain text messages containing the same information remains unclear: standard results from game theory suggest that recasting informative messages through comics cannot make any difference to individuals’ preferences. Our contributions are as follows. We draw on a rich history of the comic book form, as well as behavioral psychology to synthesize persuasive messages in comic book form. We focus on more abstract comic-forms that allow individuals to project themselves onto the character. Through experiments on Amazon Mechanical Turk, we analyze the effects of three variables: gestures, space between the characters and shading on their ability to persuade individuals. Our results suggest that people perceive the comic representation as more persuasive than the plain text and that variables of character gesture, inter-character distance and shading moderate the persuasive power differently. 
