%!TEX root = proceedings.tex

This paper looks at the ability of algorithmically synthesized comic-style messages to persuade individuals to adopt behaviors. In collective action dilemmas including climate change and public health, persuasive messages are an important aspect of enabling individuals to adopt behaviors that benefit themselves and the larger group. Whether comic representations, despite widespread use in popular culture, offer any tangible benefits over plain text messages containing the same information remains unclear: standard results from game theory suggest recasting informative messages through comics cannot make any difference. Drawing on a rich history of the comic book form and theories from behavioral psychology, we synthesize persuasive messages in an abstract comic form. Through an online experiment, we analyze the effects of three key elements in comics: gestures inter-character distance, and background shading on their ability to persuade individuals. Our results suggest  people perceive the comic representation as more persuasive than the plain text and that three key comic elements moderate the persuasiveness differently.
