%!TEX root = cscw2018-comic.tex
Now, we discuss related work in: 1) building persuasive technology and 2) persuasion through visual stimuli.

\subsection{Building Persuasive Technology}
Starting from~\textcite{goehlert1980information}, HCI researchers have spent a lot of effort in leveraging technology in persuasion.~\textcite{goehlert1980information} argues control and dissemination of information have the ability to make attitude and behavioral changes. Inspired by this argument, two approaches have been explored in constructing the persuasive systems. Information-centric approaches focused on delivering hidden or new information which has not been perceived by the user before~\cite{LeeKF11}. For example,~\textcite{chi2007enabling} changed people's nutritional composition by creating an intelligent kitchen that can provide nutritional information about ingredients while participants are cooking.  ~\textcite{liao2014expert} found showing a source expertise indicator can shape user's information seeking and burst the filter bubble. While lots of studies on persuasive technology showcase the persuasive power of the information-centric approach, studies also show the downside of information-center approach where the target receiver often failed to perceive and internalize the persuasive information ~\cite{goehlert1980information,LeeKF11}.

Adapting decision-making models from previous behavioral research, behavior-centric approach emphasized on human motivation and biases ~\cite{LeeKF11}.~\textcite{vaish2018s} used self-serving motivational framing of messages to persuade people to join a prosocial peer-to-peer service. Although the effectiveness of behavior-centric approach has been examined in multiple studies, behavior-centric approaches often require prior knowledge of the persuadee in order to unleash the persuasion power, as~\textcite{orji2014developing} and~\textcite{schneider2016understanding} suggested that different people may be more amenable to one persuasive method than others. Therefore, due to the cost of knowing people, scalability is the key challenge.

\subsection{Persuasion Through Visual Stimuli}
Prior research shows that using visual representations are more attractive than text messages~\cite{selker2015sweetbuildinggreeter,consolvo2008activity}. ~\textcite{selker2015sweetbuildinggreeter} retrieves motivational images from 9GAG, Google, with energy-saving messages, to motivate people's energy saving behaviors and found those images has higher persuasive power comparing to plain text. However, images used in persuasion are costly to generate. 

The simple and humorous nature of comic makes comics becomes an unique media for delivering informative and memorable messages. Beyond entertainment, comics can be used in scientific communication~\cite{McDermottPB18}, and for teaching in multilingual schools~\cite{cary2004going}. To the best of our knowledge, this is the first study to look at abstract comic representations and persuasion.
% The simple and humorous nature of comic makes comics becomes an unique media for delivering informative and memorable messages. While reading comics book is commonly recognized as entertaining, comics have been examined as an effective way of communicating abstract ideas to broad audiences \cite{McDermottPB18,cary2004going,scott1993understanding}. McDermott et. al used comics to illustrate complex scientific facts \cite{McDermottPB18}. In education, comics have been used and examined as an effective tool for reaching different populations with various background \cite{McDermottPB18,cary2004going,scott1993understanding}.

% Similar to images, the simple and humorous nature of comic makes comics becomes an unique media for delivering informative and memorable messages. Comics have also been examined as an effective way of communicating abstract ideas to broad audiences \cite{McDermottPB18,cary2004going,scott1993understanding}. To our knowledge, no prior study examined the persuasive power of the comic.

% We aim to fill this gap by comparing
%
% the persuasiveness between
%
% Our work aims to fill this gap by comparing the quality ofsurvey content and respondents’ engagement between theusage of a conversational survey and a traditional survey.With a few exceptions [15,16], previous work evaluatingchatbot technologies tend to rely on small-scale lab studies.Instead, we deploy a survey chatbot and study its real-worldusage of conducting marketing research survey. Our analysisis performed on the response contents and behavioral logs(e.g., time stamps), and focus on measures that are crucialfor the purpose of gathering high-quality survey data
% behavior-centric approach is scalability.
% This study mainly focuses on different comic elements' effect on persuading readers.
%
% Although persuasion can be better achieved through visual stimulus, the cost of creating personalized visual stimuli is high. Our approach leverages abstract comics taking advantage of visual stimuli while allowing for straightforward, personalized synthesis.
%
%
% Beyond entertainment, comics can be used in scientific communication~\cite{McDermottPB18}, and for teaching in multilingual schools~\cite{cary2004going}.
% The simple and humorous nature of comic makes comics becomes an unique media for delivering informative and memorable messages. While reading comics book is commonly recognized as entertaining, comics have been examined as an effective way of communicating abstract ideas to broad audiences \cite{McDermottPB18,cary2004going,scott1993understanding}. McDermott et. al used comics to illustrate complex scientific facts \cite{McDermottPB18}. In education, comics have been used and examined as an effective tool for reaching different populations with various background \cite{McDermottPB18,cary2004going,scott1993understanding}.
% The use of metaphor in comics can make the underlying meaning vivid and more memorable than using a straightforward description \cite{McDermottPB18,scott1993understanding}. Moreover, comics can contain a personal story incredibly powerful for persuasion~\cite{weaver2017losing}. With a personal story, comics can create empathy for readers.~\textcite{matsubara2016emotional} showed a link between comic's content and the emotions felt by the readers. Thus complex messages can be easily interpreted and memorized though use of the comic form.
