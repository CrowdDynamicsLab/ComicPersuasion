%!TEX root = cscw2018-comic.tex

In the following sections we discuss related work in: 1) building persuasive technology and 2) persuasion through visual stimuli.

\subsection{Building Persuasive Technology}
Starting from Goehlert's discussion on persuasion and communications technology in 1980, HCI researchers have spent a lot of effort in leveraging technology in persuasion~\cite{goehlert1980information}. Goehlert argues control and dissemination of information have the ability to make attitude and behavioral changes~\cite{goehlert1980information}. Inspired by this argument, two types of approaches have been used in constructing persuasive systems. Information-centric approaches focused on delivering hidden or new information which has not been perceived or recognized by the user before~\cite{LeeKF11}. For example, Chi et al. created an intelligent kitchen that can provide nutritional information about ingredients while users are cooking~\cite{chi2007enabling}. People start to adjust their ingredients usages to achieve a better nutritional composition. Waterbot persuades people to engage water-saving behavior by augmenting physical sink interface with water usage indicators~\cite{arroyo2005waterbot}. Liao and Fu found showing a source expertise indicator can shape user's information seeking behavior and burst the filter bubble \cite{liao2014expert}. While a lot of studies on persuasive technology showcase the persuasive power of the information-centric approach, studies also show the downside of information-center approach where the target receiver often failed to perceive and rationalize the persuasive information when the receiver is experiencing information overload~\cite{goehlert1980information,LeeKF11}. Additionally, the information-centric approach often relies on the target receiver to make rational decisions based on provided information whereas the premise of rationality does not always hold.

Adapting decision-making models from previous behavioral research, some persuasive technology emphasized on human motivation and biases, e.g. behavior-centric approach ~\cite{LeeKF11}. Lee et al. incorporated the idea of default bias in the design of the Snackbot robot and successfully persuade people with healthy snacking in the workplace ~\cite{LeeKF11}. Vaish et. al used self-serving motivational framing of messages to persuade people to sign up for a prosocial peer-to-peer service~\cite{vaish2018s}. Borrowing from the theory of planned behavior, Schneider et. al understood different motivational factors of mobile fitness coach users and delivered individualized messages to persuade users based on their motivations~\cite{schneider2016understanding}. Although the effectiveness of behavior-centric approach has been examined in multiple studies, behavior-centric approaches often require prior knowledge of target subject in order to maximize the persuasion power, as Orji et. al and Schneider et. al suggested in their study that different people may be more amenable to one persuasive method than others~\cite{schneider2016understanding,orji2014developing}. Therefore, the common challenging faced by behavior-centric approach is scalability. In our study, we leverage both information-centric and behavior-centric approaches. The comic is designed to show user's behavior statistical facts (information-centric) but the message is framed beyond simple fact to achieve higher persuasiveness (behavior-centric). The challenge here is mitigating the downsides of information-centric and behavior-centric approaches. We address those challenges by using a novel representation, an abstract comic, to catch the information receiver's attention. And we build a framework that allows for  automated persuasive comics generation.

\subsection{Persuasion Through Visual Stimuli}
Throughout the years, research on evaluating the effect of visual and graphic stimulus in persuasion have been more and more popular~\cite{selker2015sweetbuildinggreeter,consolvo2008activity}. Previous research shows using visual representations are more attractive to readers comparing to text messages ~\cite{selker2015sweetbuildinggreeter,consolvo2008activity}. Selker et al. retrieves motivational images from 9GAG, Google, with energy-saving messages, to motivate people's energy saving behaviors and found those comics has higher persuasive power comparing to plain text ~\cite{selker2015sweetbuildinggreeter}. Consolvo et. al built Ubifit Garden that successfully encourage physical activity by present user's exercise data as different elements in a garden ~\cite{consolvo2008activity}. Lin ~\cite{lin2013impact} compared two models of presenting information about wind energy in brochure form: (1) photographs and (2) using cartoons as visual aids. To evaluate the effect, the research focused on comparing three measures: (1) audience's knowledge of, (2) attitudes toward, and (3) behavioral intentions regarding wind energy. Results show that there is no significant difference between using photographs or cartoon in terms of knowledge and attitudes. However, visual aids shown in the cartoon version showed stronger behavioral intentions (e.g., greater willingness to support changes) than the photo group. Abstract comics can better engage readers compared to photographs, making readers more willing to adopt changes. This study mainly focuses on different comic elements' effect on persuading readers.

Although persuasion can be better achieved through visual stimulus, the cost of creating personalized visual stimuli is high. Our algorithmic approach leverages abstract comics taking advantage of visual stimuli while allowing for straightforward, personalized synthesis.
