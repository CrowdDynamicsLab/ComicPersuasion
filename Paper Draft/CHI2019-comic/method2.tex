%!TEX root = cscw2018-comic.tex
\section{Study on Behavior: Method}
\label{sec:Method2}

In the second study, we further examined the ability of abstract comics in persuading people to make decisions in the real life. We conducted a field study on Amazon Mechanical Turk and compared the persuasive power between pure text messages and abstract comic messages in persuading people to donate to the Organization for Autism Research (OAR). In this section, we will introduce the experiment design and describe our study participants recruiting process.

\subsection{Experiment Design}
Since the main goal for this study is to compare power of a persuasive message on behaviors in two forms, the abstract comic and the pure text, we first constructed two experimental conditions, abstract comic and pure text. In the abstract comic condition, participants will receive a persuasive message in a three-panel abstract comic strip, whereas in the pure text condition, participants will receive the same message in pure text. Additionally, we are also interested in if the persuasive technique used in the text messages can also be applied in the comic form. Hence, we incorporated the idea of social proof and created the third condition, abstract comics with social proof.

The objective of the persuasive message in all three conditions were the same, persuading participants to donate their perspective bonus rewards (10\% chance of winning \$5 bonus) to a charity.Study participant will be randomly assigned into one of the three conditions and make decision on the amount of donation based on their will.

\paragraph{Study Procedure} Once the participant consents to join the study, they will be asked if he/she is familiar with the Autism Spectrum Disorder (ASD). Then, participant will watch a short-video produced by the Organization for Autism Research that promotes its fundraising activity "RUN FOR AUTISM". After watch the video, participant will briefly summarize the video and provide their opinion about the effectiveness of the video, which is the task we mentioned in the recruiting message.

Then participants will be randomly assigned to each of the three conditions and read the persuasive message in different forms. In the message, the participant were provided a 10\% chance of winning \$ 5 additional compensation. They also have the opportunity to donate to the Autism Spectrum Disorder (ASD) which is the charity in the video they have watched before.

To best demonstrate the persuasiveness of the message itself, we diffused the responsibility of donation amount all the partcipants. So, before the partcipants make their decision, they will read "The total amount of money allocated to [the charity] by all the winning participants will be aggregated and donated at the end of the study."~\cite{} Then the participants will be asked to allocate the amount of money they were willing to donate on a slider bar with \$0 and \$5 as two ends. The default position of the slider bar is at \$0 end.

Before leaving the study, the participants were asked to fill demographic questions about their gender, age and education.

We first validated the study procedure with two conditions, pure text and abstract comic before the actual experiment starts.

\paragraph{Organization for Autism Research (OAR)}
To increase the realism of our study, we chose a real charitable organization, the Organization for Autism Research (OAR). We chose OAR for two reasons, 1) The Autism Spectrum Disorder (ASD), a developmental disorder that affects communication and behavior, is a well-known disorder that can affect people in general instead of people in specific area or with specific demographic characters. In other word, ASD is a problem potentially related to every participants in our study, which provides the basic interest for them to support related charitable organization and 2) the Organization for Autism Research is one of the most reputable organization that helps individuals with autism and provides assistance to parents, families, teachers and caregivers. The goal of OAR is clear and reputable so participants won't question the authenticity of our message's motive.

\paragraph{Persuasive Messages}

The persuasive messages communicate three major objectives, 1) Participants will have 10 \% of chance winning \$ 5 bonus upon the completion of the study. 2) Participants are free to use the money as they please. and 3) Participants can donate this bonus to the Organization for Autism Research (OAR). Therefore, in the text condition, study participants will read the following message,
\begin{quote}
  \textit{You have a 10\% chance of winning a five dollar bonus in this study. You are free to use this money as you please. If you win the prize, would you like to donate to support the Organization for Autism Research?}
\end{quote}
In the two comic conditions, we created a comic strip with three panels to communicate each of the major objective. The comic strip is created in the similar fashion as in the first study on perference. In the abstract comics with social proof, we created social proof by adding an additional sentence about the percentage of people in this study chose to donate in the third panel after telling partcipants the opportunity to donate. The percentage is based on number of people chose to donate in the pilot study.

%will add two comics pic to demonstrate

\subsubsection{Participants}
We published our HITs on Amazon Mechanical Turk titled with ``A short survey about communicating autism campaign ads''. Similar to the Study 1, the price tag is \$8/hr, the workers would get these rewards regardless of their performance, the threshold for participant to join was a 95\% Approval Rate. On the HIT page, and repeated responses will be rejected as instructed, participants would see a link to our experiment site and a text input box for them to enter a six-digit completion code.
