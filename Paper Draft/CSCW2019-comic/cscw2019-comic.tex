\documentclass[format=acmsmall, natbib=false, review=false, authordraft=false, anonymous=true, screen=true]{acmart}
\settopmatter{printccs=false,printacmref=false}

% using biblatex
\let\citename\relax
\RequirePackage[abbreviate=true,
 dateabbrev=true,
 natbib=true,
 isbn=false,
 doi=false,
 eprint=false,
 urldate=comp,
 url=false,
 maxbibnames=9,
 maxcitenames=1,
 backref=false,
 backend=biber,
 style=ACM-Reference-Format,
language=american]{biblatex}

\addbibresource{cscw-comic.bib}
\addbibresource{hari-cscw-comic.bib}
\renewcommand{\bibfont}{\Small}

\usepackage{wrapfig}
\usepackage{booktabs} % For formal tables
\usepackage{cleveref} % for better references
\usepackage{graphicx,subfig,caption}
\usepackage[ruled]{algorithm2e} % For algorithms
\renewcommand{\algorithmcfname}{ALGORITHM}
\SetAlFnt{\small}
\SetAlCapFnt{\small}
\SetAlCapNameFnt{\small}
\SetAlCapHSkip{0pt}
\IncMargin{-\parindent}

\usepackage{subfig} % subcaptions
\usepackage{wasysym} % smiley faces
\usepackage{xfrac}

% Metadata Information
% \acmJournal{TWEB}
% \acmVolume{9}
% \acmNumber{4}
% \acmArticle{39}
% \acmYear{2010}
% \acmMonth{3}
% \copyrightyear{2009}
%\acmArticleSeq{9}

% Copyright
%\setcopyright{acmcopyright}
% \setcopyright{acmlicensed}
%\setcopyright{rightsretained}
%\setcopyright{usgov}
%\setcopyright{usgovmixed}
%\setcopyright{cagov}
%\setcopyright{cagovmixed}
\setcopyright{none}

% DOI
% \acmDOI{0000001.0000001}
\acmDOI{}

% % Paper history
\acmConference[]{Submitted to CSCW}{2019}{Austin, TX.}
% \received{February 2007}
% \received[revised]{March 2009}
% \received[accepted]{June 2009}


% Document starts
\begin{document}
% Title portion. Note the short title for running heads
\title[The Abstract Comic Form for Persuasion]{Should We Use an Abstract Comic Form to Persuade? Experiments in Charitable Donation}


\author{Ziang Xiao}
\email{zxiao5@illinois.edu, hs1@illinois.edu}
\author{Po-Shiun Ho}
\email{pho11@illinois.edu}
\author{Hari Sundaram}
\email{hs1@illinois.edu}
\affiliation{%
 \institution{University of Illinois}
 \department{Computer Science}
 \city{Urbana}
 \state{IL}
 \postcode{61801}
 \country{USA}}
\author{Xinran Wang}
\email{wangxr1108@126.com}
\affiliation{%
 \institution{Tsinghua University}
 \department{Computer Science}
 \city{Beijing}
 \postcode{100084}
 \country{China}}




\begin{abstract}
This paper examines if persuasive messages when expressed in abstract comic form are more powerful in encouraging people to cooperate in collective action dilemmas, e.g. making charitable donation decisions. Due to the non-exclusive nature of public goods, persuading individuals to contribute is important. Although persuasive messages have shown success in multiple scenarios, free-riders still exist. In this study, we explored a novel form of persuasive messages, abstract comic. Whether comic representations, despite widespread use in popular culture, offer any tangible benefits over plain text messages in persuasion is unclear and motivates our study. Drawing on a rich history of the comics, we synthesize persuasive messages in an abstract comic form. We conduct a field study with 307 participants on the use of comic form to persuade in a charitable donation scenario. Motivated by its use in small-sized studies, we use a hierarchical Bayesian model to analyze the results. Our experiments reveal that abstract comic form is more persuasive than text (medium effect size=0.44). Incorporating social proof improve comic effectiveness, but the effect is minor. Our studies have design implications: we ought to consider the abstract comic form to communicate messages intended for nudging people to cooperate in collection actions.
\end{abstract}


%
% The code below should be generated by the tool at
% http://dl.acm.org/ccs.cfm
% Please copy and paste the code instead of the example below.
%
% \begin{CCSXML}
% <ccs2012>
%  <concept>
%   <concept_id>10010520.10010553.10010562</concept_id>
%   <concept_desc>Computer systems organization~Embedded systems</concept_desc>
%   <concept_significance>500</concept_significance>
%  </concept>
%  <concept>
%   <concept_id>10010520.10010575.10010755</concept_id>
%   <concept_desc>Computer systems organization~Redundancy</concept_desc>
%   <concept_significance>300</concept_significance>
%  </concept>
%  <concept>
%   <concept_id>10010520.10010553.10010554</concept_id>
%   <concept_desc>Computer systems organization~Robotics</concept_desc>
%   <concept_significance>100</concept_significance>
%  </concept>
%  <concept>
%   <concept_id>10003033.10003083.10003095</concept_id>
%   <concept_desc>Networks~Network reliability</concept_desc>
%   <concept_significance>100</concept_significance>
%  </concept>
% </ccs2012>
% \end{CCSXML}
%
% \ccsdesc[500]{Computer systems organization~Embedded systems}
% \ccsdesc[300]{Computer systems organization~Redundancy}
% \ccsdesc{Computer systems organization~Robotics}
% \ccsdesc[100]{Networks~Network reliability}

%
% End generated code
%


\keywords{statistical facts, abstract comics, persuasion, information framing, social proof, hierarchical bayesian models}




\maketitle

% The default list of authors is too long for headers.

%!TEX root = cscw2018-comic.tex
Today, the world generates information all around us every second. We are surrounded by all sorts of messages trying to change what we think and what we do: our newsfeed is full of advertisements, our wearable devices are keeping telling us to exercise more, even our water bottle starts to push notification to remind us stay hydrated. However, not all messages can successfully make us change: we won't buy an expensive car because of a short video, we often eat junk food even though we received a lot of articles about health eating and we are often in the status of dehydration after reading those notifications. Thus, how to make a message more persuasive has been a critical problem throughout the years.\par
While classical game theory suggests that recasting an informative message through a different form cannot increase the message's persuasive power, a rich body of research has demonstrated how susceptible we are to those fancy words \cite{tversky1992advances,tversky1981framing,goldstein2008room,schultz2007constructive}. Hotel guests start to reuse their towels more because of a subtle change of a sign positioned on washroom towel racks \cite{goldstein2008room}. Households start to reduce their energy consumption because of an emoji on their energy bill \cite{schultz2007constructive}. People are more willing to sign up for a prosocial peer-to-peer service because of a message on the sign-up page telling them explicitly what benefit they might get \cite{vaish2018s}. Given the susceptible nature of our species, we believe changing the representation of a message can make the message more persuasive.\par
Comics, a medium to express ideas in a graphical form, are one of the most popular form of art across different cultures. The history of comics can be traced back to early precursors such as Trajan's Column \cite{o1971art}. Comics are usually in the form of juxtaposed sequences of panels of images or a standalone single image. Beside the graphical representation, textual elements such as speech balloons, captions, and onomatopoeia often communicate dialogue, narration, sound effects, or other information. Existing work shows that comics can effectively convey meanings \cite{McDermottPB18,cary2004going,scott1993understanding}. Yet the persuasiveness of the comics representation has not been investigated.\par

In this study, we want to answer the following research questions.

\begin{itemize}
    \item RQ-1: What's persuadee's reaction when a plain text persuasive message is synthesized into a piece of comics?
    \item RQ-2: What's the key elements of a persuasive comic that affects persuadee's preferential choice?ß
\end{itemize}
To answer those question, we investigated the persuasive power of comic style persuasive messages through a filed study with 200 participants on Amazon Mechanical Turk. First, we reviewed some of the previous works on persuasive technologies in three major fields: (1) current development of persuasive technology, (2) ways to approach computational persuasion, and (3) communicating through comic. Second, we discuss about the general composition of a comic message and its comparison with plain text. Lastly, to examine our hypothesis that persuasive messages in the comic form are preferred over plain text, we conduct a field experiment to investigate the persuasiveness of comic messages comparing with text messages. The result shows comics can be a better form of persuasive message comparing to plain text and three key comic elements, character gesture, inter-character distance, and background shading moderate the persuasiveness differently.\par

The contributions of this work are: 1) among the first to examine the effect on persuadee's preferential choice for comic form messages. 2) exploring the key elements in comic form persuasive messages. 3) providing a novel framework to approach computational persuasion by algorithmically synthesizing persuasive message into comic form. \par

%!TEX root = proceedings.tex

Research on persuasive messages in HCI has been increasingly popular due to the fact of information explosion. Previous research has provided solid strategies that can make text messages more persuasive, including messaging framing, information-centric approaches, and personalize information context []. However, due to the advance in technology, we are immersed in the information ocean and overwhelmed with texts various sources. How to make individual message stand out among others and attract reader's attention while nudging the reader to make behavioral and attitude changes becomes an important problem. In this study, we looked into a novel form of persuasive message, visual comics. Then, we investigated the persuasive power of different visual cues in comics beyond pure text. In the following sections, we will introduce related studies categorized into groups: (1) Framing text-form persuasive messages, (2) Persuasion beyond text, and (3) Communicating through comics.\par

\subsection{Building Persuasive Technology}
Starting from Goehlert's discussion on persuasion and communications technology in 1980, HCI researchers have spent a lot of effort in leveraging technology in persuasion. Goehlert argues control and dissemination of information have the ability to make attitude and behavioral change. Inspired by this argument, two types of approaches have been used in constructing persuasive system. Information-centric approaches focused on delivering information which did not perceived or recognized by the user before. For example, Chi et al. created an intelligent kitchen that can provide nutritional information about ingredients while users are cooking. People start to adjust their ingredients usages to achieve a better nutritional composition. Waterbot persuades people to engage water-saving behavior by augmenting physical sink interface with water usage indicators.
Liao and Fu found showing a source expertise indicator can shape user's information seeking behavior and burst the filter bubble. While a lot of studies on persuasive technology evident the persuasive power of this information-centric approach, pervious studies also show the downside of information-center approach where the target receiver often failed to perceive and rationalize the persuasive information when the receiver is experiencing information overload. Additionally, information-centric approach often relays on the target receiver can make rational decision based on provided information whereas the premise of rationality does not always hold.\par

Adapting decision-making models from previous behavioral research, some persuasive technology emphasized on human motivation and biases, e.g. behavior-centric approach []. Lee et al. incorporated the idea of default bias in the design of the Snackbot robot and successfully persuade people with healthy snacking in the workplace. Vaish et. al used self-serving motivational framing of messages to persuade people to sign up for a prosocial peer-to-peer service. Borrowing from the theory of planned behavior, Scheneider et. al understood different motivational factors of mobile fitness coach users and delivered individualized messages to persuade users based on their motivations. Although the effectiveness of behavior-centric approach has been examined in multiple studies, behavior-centric approaches often requires prior knowledge of target subject in order to maximize the persuasion power, as Orji et. al and Schneider et. al suggested in their study that different people may be more valuable to one persuasive method than others.  Therefore, the common challenging faced by behavior-centric approach is scalability.\par

Our study, we tried to incorporate both information-centric and behavior-centric approaches. The comic is designed to show user's behavior stats (information-centric) but the message is framed beyond raw representation in order to achieve maximum persuasiveness (behavior-centric). The challenge here is mitigating the down sides of information-centric and behavior-centric approaches. We used novel representation, visual comic, to catch information receiver's attention and developed a tool that has the ability to algorithmically generate personalized persuasive comics.
\par

\subsection{Text Messages Persuade}
Approaches to make a message more attractive and persuasive has long been a focus for a variety of different fields including computational linguistic, social networking, and advertising. Among these tactics, message framing is one of the most basic and intuitive methods to generate memorable and persuasive messages. This is mainly because message framing and phrasing generally doesn't require additional information or data visualization. Its simplicity contributes to the various researches on the effect of message framing in memorability. Mizil et.al shows that using unusual word choices and more general theme makes it easier to connect with reader's daily life and makes the message more memorable. Using unexpected words and phrases, the message is more likely to capture reader's attention comparing to normal phrases. As for the theme of the message, it should be as general as possible to help readers connect with the message, so it can stay in the reader's mind longer.\par

Tversky and Kahneman takes on a different aspect of the effect of framing. Instead of focusing on the word selection and sentence structure of a quote, Tversky and Kahneman focused on how different reference point of a same sentence can result in reader's different response. It shows that variations of reference point of a decision can determine whether the reader will evaluate it as a gain or a loss, thus changing their decision. For example, choices involving gains are often risk averse and choices involving losses are often risk taking. Meanwhile, implying social norm through the message is another persuasive technique.  Goldstein et. al conducted an interest experiment in the hotel on motivating environmental conservation. They found employing descriptive norms (e.g., "the majority of guests reuse their towels") has more persuasive power than solely mentioning environmental protection. And this normative message gets more persuasiveness when the described setting is closer to individuals' immediate situational circumstances (e.g., "the majority of guests in this room reuse their towels").  All those techniques are essential for our design of messages to persuade behavioral change.\par


\subsection{Communicating through Comics}
The simple and humorous nature of comic make it becomes an excellent media for delivering informative and memorable messages. While reading comics book is commonly recognized as entertaining, comics has been examined as an effective way of communicating abstract ideas to broad audiences []. Bromberg et. al used comics to illustrate complex scientific facts. In the area of education, comics have been used and examined as an effective tool for reaching different populations with various background []. Meanwhile, the common usage of metaphor in comics can make the underlying meaning for vivid and therefore more memorable than using a straightforward description[]. Moreover, simple comics can express emotion and create empathy for readers. Lima Sanches et al. showed that there is a link between comic’s content and the emotions felt by the readers. Thus, with the form of comics, complex message can be easily interpreted and memorized. Given the advantage of using comics to deliver meanings, our study took one step further and considered the persuasive side of comic representation. \par

However, using comics to persuade is challenging. First of all, generating comics is not easy. Especially, persuasive message should be personalized to deliver maximum persuasive power. Traditionally, comics is created by professional cartoonists which is very costly to produce personalized comics.  Although prior work has explored methods of algorithmically generating comics, no existing method is for generating persuasive comics. \par

\input{motivatingideas.tex}
%!TEX root = proceedings.tex

To test our hypotheses, we designed and conducted a field study through Amazon Mechanical Turk. In the experiment, participants will see five persuasive messages in both plain-text form and comic form side by side. Then, the participant will be asked which form of the message is perceived as more persuasive and how persuasive is it.
\subsection{Persuasive Messages}
All persuasive messages were focused on motivating exercise behavior. Persuasive messages were constructed by providing descriptive normative information, e.g. "In the past week, you spent more time at the gym than did 65\% of your friends", or descriptive exercise data, "Congrats! You have reached your goal of exercising three times a week." Since existed research suggests the persuasive power of a normative messages may differ between receivers who did not meet the social norm and receivers who outperforms the social norm, we also constructed two set of messages that targeting receivers who meet descriptive norm (Positive framed messages) and others who fail (Negative framed messages) by negating some key words (Less vs. More/ Bottom vs. Top/ Reached vs. didn't Reached). In our study, five persuasive messages were either all positive framed or all negative framed.\par
The followings messages were presented in our study.\par
\textit{Positive framed messages:}
\begin{enumerate}
  \item In the past week, you spent more time at the gym than did 65\% of your friends
  \item Congrats! You have reached your goal of exercising three times a week.
  \item Over the past month, you exercised more than did 90\% of your friends.
  \item Your exercise activity is in the top 20\% of all your friends.
  \item Over the past three weeks, you went to the gym more often than 60\% of your friends did.
\end{enumerate}\par
\textit{Negative framed messages:}
\begin{enumerate}
\item	In the past week, you spent less time at the gym than did 65\% of your friends
\item Congrats! You did not reach your goal of exercising three times a week.
\item	Over the past month, you exercised less than did 90\% of your friends.
\item	Your exercise activity is in the bottom 20\% of all your friends.
\item	Over the past three weeks, you went to the gym less often than 60\% of your friends did.
\end{enumerate}\par
\subsection{Comic Style Messages Generation}
\subsection{Question Design}
\textit{Display Order}. To mitigate any potential biases toward the display order of the plain-text representation and comic-style representation. The display order is randomly assigned, which means both plain-text representation and comic-style representation have equal chance to be displayed on the left side or the right side of the comparison.\par
\textit{Attention Checker}. To control the data quality, we embedded two attention checkers in the experiment. The first one appears after the third comparison and the second one shows up after the last comparison. Both attention checkers asked subjects to choose a comic that matches a simple description, e.g. "Which of the following comics has two characters?".\par
\textit{Scale Design}. The goal of the scale design is to monimize any potential ambiguity and biases.The scale in this study is a 7-point binary adjective items scale with Neutral at the middle. Two ends of the scale indicates the strongest perference toward Plain-Text representation or Comic-style representation.
At the first iteration of the design, the direction of the scale changes corresponding to the display order of the two representations.
However, in our pilot testing of the scale, participants reported that the scale was confusing because of the direction change. Some participants even did not notice the direction of the scale is corresponding to the messages, instead, they treated it as a fixed direction scale based on their first impression.Therefore, we changed our scale to a fixed direction scale. To avoid any potential demand characteristics introduced by the scale, e.g. the large number is desired or selecting item on the right side is expected, our final design does not contain any number, and the direction of the scale is randomly assigned for each participant with either plain-text always on the left or comic-style message always on the left.\par
\subsection{Participants Recruitment}
We published our HITs on Amazon Mechanical Turk titled with "A short survey about your exercise motivation". The price tag for each HIT was \$0.50, which were the rewards the workers would get regardless of their performance. On the HIT page, participants would see a link to our experiment site and a text input box for them to enter a six-digit completion code. Repeated worker will be rejected as we instructed in the task description.

%!TEX root = cscw2018-comic.tex

\section{Results}
\label{sec:Results}

\subsection{Raw Data}
\label{sub:Raw Data}
In this section we describe the raw data counts, the number of participants, the number of people whose responses we dropped. The final number of observations

\subsection{Bayesian Model}
\label{sub:Bayesian Model}
We use a Bayesian formulation of the problem of identifying suitable predictors for the messages in comic form.~\textcite{Kay2016} provide an nice introduction on the appropriateness of Bayesian analysis for the HCI community. Bayesian analysis is attractive in our experiment due to two advantages: shifting the conversation from ``did it work'' to ``how strong is the effect''; and benefits to small $n$ studies.

We manipulate five independent variables: gesture of the participants in the comic (3: neutral, moderate, extreme);  distance between the two characters (3:close, moderate, far); comic shading (3:white, light gray, gray); framing (2: whether the information was positively framed or negatively framed).  This gives us a total of $3 \times 3 \times 3 \times 2= 54$ experimental conditions. As a control against ordering effects, we randomly manipulate comic position (whether we presented the comic panel to the left or to the right).  Thus, we need to estimate the effect on the responses for each of these variables; the responses are on a 7 point Licket scale.

A challenge with using ordinal scale such as the Lickert scale: we do not know the ``width'' of each response. That is, while we may know that for example $1<2<3$, we don't know if the difference in the thresholds used by subjects to mark ``2'' on the scale, is the same as the difference in thresholds they use for ``1'' and ``3.''  We assume that each response by a subject: lies in a continuous metric space; is Normally distributed; and that the thresholds $\{\theta_i\}$ while unknown, are shared—all subjects use same set of thresholds to identify the appropriate ordinal value.

Formally let $z$ be the response of the subjects to the experiment where the comic panel was generated by the different conditions; each condition is obtained by setting each of the $k$ independent variables $\{x_j\}, j \in [1 \ldots k]$. The subjects first generate a Normally distributed metric variable $y$, and then use the thresholds $\{\theta_i\}$ to map $y$ to the ordinal variable $z$.

Then, since we assume that the metric variable $y$ is Normally distributed, our hierarchical Bayesian model is defined as follows (see~\Cref{fig:generative-main} for a graphical representation):

\begin{align}
 y                  & \sim N(\mu, \sigma_y)                    \label{eq:response-main}                       \\
 \sigma_y           & \sim U(L, H), \label{eq:main-sigma}                                                     \\
 \mu                & \sim \beta_0 +
 \underbrace{\sum_{j} \beta_{1,j} x_{1,j}(i)}_{\text{gesture}} +
 \underbrace{\sum_k \beta_{2,k} x_{1,k}(i)}_{\text{shading}} +
 \underbrace{\sum_l \beta_{3,l} x_{1,l}(i)}_{\text{distance}} +
 \underbrace{\sum_m \beta_{4,m} x_{1,m}(i)}_{\text{framing}} ,                  \label{eq:mu-main} \\
 \sum_j \beta_{i,j} & = 0, \qquad  \qquad  \quad \, i \in \{1, \ldots, 4\}, \label{eq:beta-equality}          \\
 \beta_{i,j}        & \sim N(0, \sigma_{\beta, i}), \qquad  i \in \{1, \ldots, 4\},\label{eq:main-beta-sigma} \\
 \sigma_{\beta, i}  & \sim \Gamma(s, r ). \label{eq:gamma-distribution}
\end{align}
\Cref{eq:response-main} says that the metric variable $y$ is Normally distributed\footnote{The ``$\sim$'' symbol means that the random variable on the left is drawn from the probability distribution on the right.} with mean $\mu$ and standard deviation $\sigma_y$.~\Cref{eq:mu-main} says that the mean response $\mu$ is a linear weighted combination of the predictors.~\Cref{eq:main-sigma} says that the standard deviation of the response is drawn from a Uniform distribution with constant parameters $\text{Low}=L, \text{High}=H$, where $L>0, H\gg L$.~\Cref{eq:main-beta-sigma} says that the predictor weight is drawn from a Normal distribution with $\mu=0$ and standard deviation $\sigma_{\beta, i}$. That is,  while \textit{each} predictor set $\beta_{i}$ is drawn from a \textit{different} Normal distribution, the $\beta_{i,j}$ values within the same predictor set $\beta_{i}$ are drawn from the \textit{same} Normal distribution.~\Cref{eq:beta-equality} says that the sum of the deflections $\sum_k \beta_{j,k}$ from the mean for any nominal predictor $j$ equals zero.~\Cref{eq:gamma-distribution} says that we draw all the variances $\sigma_{\beta, i}$ from a Gamma $\Gamma(s,r)$ distribution, where $s$ refers to the shape parameter and $r$ refers to the rate parameter. We set the variables $s,r$ to allows a wide range of values for $\sigma_{\beta, i}$.

By drawing the standard deviation variables $\sigma_{\beta, i}$ from a Gamma distribution, the values of each element of $\beta_i$ informs the other elements. Notice that we draw the variances $\sigma_{\beta, i}$ of each predictor $i$ from an independent Gamma distribution, implying that the variances (equivalently, the extent of deflections from the mean) for each predictor can be different. Furthermore, the ``information sharing'' among variables is common to hierarchical Bayesian models and is an important reason why Bayesian models work so well with small datasets\footnote{The sharing of information causes the variance of each individual element $\beta_{i,j}$ to move towards the group variance, a phenomena known as ``shrinkage.'' }. The main advantage of using a Gamma distribution is that we can specify a non-zero mode, important in controlling shrinkage in hierarchical models.

\begin{figure}
 \includegraphics[width=\textwidth]{./figures/generative_model.pdf}
 \caption{The figure shows the hierarchical Bayesian model specification, corresponding to~\Crefrange{eq:response-main}{eq:gamma-distribution}.}
 \label{fig:generative-main}
\end{figure}

Thus far, we have discussed how to generate a Normally distributed metric variable $y$. However, what we see in the experiment is not this metric variable, but an ordinal variable $r$. The subjects use internal thresholds $\{\theta_i\}$ to determine when to ``strongly disagree'', ``disagree'' etc. Thus with a 7 point Likert scale, we have 6 thresholds. The probability that we will see an ordinal response $r=k$ is $P(r=k | \mu, \sigma, \{\theta_i\})$, where, $\{\theta_i\}$ is the set of thresholds used by the subjects. We assume that while these thresholds are unknown, all subjects use the same thresholds. Since the underlying metric response $y$ is Normally distributed, we can compute the probabilities for observing each ordinal response $r=k$ as follows:

\begin{equation}
 P(r=k | \mu, \sigma, \{\theta_i\}) = \Phi \left (\frac{\theta_k - \mu}{\sigma} \right) - \Phi \left(\frac{\theta_{k-1} - \mu}{\sigma} \right).
\end{equation}
Where, $\Phi$ represents the cumulative density function for the Normal distribution corresponding to the underlying metric variable $y$. In other words, the probability that we will see ordinal response $k$ is the area under the Normal distribution with parameters $\mu, \sigma$ between thresholds $\theta_{k-1}$ and $\theta_k$.

The thresholds $\theta_i, i \in \{2, 3, 4, 5\}$ have two degrees of freedom in that a simple translation of the response will translate the thresholds. Consistent with~\textcite[][p. 674]{Kruschke2014}, we set $\theta_1\equiv1.5$ and $\theta_6\equiv6.5$, leaving us with four hidden threshold parameters. We draw these remaining four $\{ \theta_i\}$ from a Normal distribution as follows:
\begin{equation}
 \theta_i \sim N(i+0.5, 1/2), i \in \{2, 3, 4, 5\}.
\end{equation}

In this section, we developed a hierarchical Bayesian formulation to model the subject ordinal response to analyze the effect of different predictors for three comic elements (gesture, distance, shading) and information framing. We have a total of $54$ experimental conditions. We also modeled the thresholds for the different ordinal outcomes as a hidden variable. In the next section, we present and analyze the results.

\subsection{Analysis}
\label{sub:Analysis}

We analyzed the data using PyMC3~\cite{Salvatier2016}, a popular framework for Bayesian inference. Computational techniques for Bayesian inference use a stochastic sampling technique called Markov Chain Monte Carlo (MCMC) that samples the posterior distribution $P(\theta | D)$, where we want to estimate the parameters $\theta$ given the observations $D$. In particular, we used the Metropolis-Hastings sampler. The Gelman-Rubin statistic $\hat{R}$ was around 1, indicating that the different sampling chains converged. The modal values of the coefficients are as follows:

\begin{table}[htb]%\footnotesize
 \centering
 \caption{Modal coefficient values $\beta_{0-4}$. Some coefficients are vectors as they represent the displacement from the mean for different conditions of that variable. For example, since gesture has three experimental conditions, $\beta_1$ is a vector of length 3.}\label{tab:modal values}
 \begin{tabular}{@{}rl@{}} \toprule
  Coefficient                     & Values                     \\ \midrule
  Intercept ($\beta_0$)           & $4.524$                    \\
  Gesture ($\beta_1$)             & $[-0.218, 0.101, 0.118]$   \\
  Shading ($\beta_2$)             & $ [-0.218 , 0.101, 0.118]$ \\
  Distance ($\beta_3$)            & $[-0.043, -0.057, 0.099]$  \\
  Framing ($\beta_4$)             & $[ 0.051, -0.051]$         \\
  Standard Deviation ($\sigma_y$) & $1.614$                    \\ \bottomrule
 \end{tabular}
\end{table}


\begin{figure}
 \subfloat[The mean effect and the effect size\label{subfig-1:mean-effect}]{%
  \includegraphics[width=0.6\textwidth]{./hari-code/factors_mean_effect_main-noint.pdf}
  } \hfill
 \subfloat[Information framing contrast\label{subfig-2:framing}]{%
  \includegraphics[width=0.33\textwidth]{./hari-code/factors_framing_contrasts_main-noint.pdf}
 }
 \caption{~\Cref{subfig-1:mean-effect} showsHigh Posterior Density (HPD) intervals for the mean response $\mu$ and effect sizes $\sigma_y$. HPD represent the region with 95\% of the density. Notice that the HPD interval for $\mu$ is $[4.35, 4.70]$ and excludes 4 (the neutral response value), implying that on average, the response to the comic panel was more persuasive than the text. The figure for effect size shows a moderate effect with mode $0.33$; since the HPD interval $[0.21, 0.44]$ excludes 0, we can be confident about the effect.~\Cref{subfig-2:framing} shows the contrasts between negatively framed message with a positively framed message. The modal value is $0.08$, but since the HPD interval $[-0.12, 0.36]$ overlaps with 0, there is no appreciable effect (interestingly 80\% of the density lies in the region greater than 0.)}
 \label{fig:main-experiment-effect}
\end{figure}

\begin{figure}
 \includegraphics[width=\textwidth]{./hari-code/factors_gesture_contrasts_main-noint.pdf}
 \caption{gesture contrasts}
 \label{fig:gesture-contrasts-main}
\end{figure}

\begin{figure}
 \includegraphics[width=\textwidth]{./hari-code/factors_shading_contrasts_main-noint.pdf}
 \caption{shading contrasts}
 \label{fig:shading-contrasts-main}
\end{figure}

\begin{figure}
 \includegraphics[width=\textwidth]{./hari-code/factors_distance_contrasts_main-noint.pdf}
 \caption{distance contrasts}
 \label{fig:distance-contrasts-main}
\end{figure}

%!TEX root = cscw2019-comic.tex
\section{Model Criticism}
% \section{Model Criticism}
\label{sec:Model Criticism}

In this section, we  begin with a short section explaining our choice to use Bayesian analysis. Then in~\ref{sub:Posterior Checks, Convergence and Normality}, we examine model convergence and Normality.

% how well the proposed models explain the observed data, including model convergence.

\subsection{Why Bayesian?}
\label{sub:Why Bayesian?}
In a recent paper,~\textcite{Kay2016}, make a persuasive argument that Bayesian methods are better suited to the HCI community, including making the case that Bayesian methods allow for replicating the results and improving the strength of the conclusions by using previous outcomes as priors. We would add two reasons, in addition to those by~\textcite{Kay2016} to explain our decision to use Bayesian models. 
\begin{description}
    \item[Transparency:] With a Bayesian model, the researcher foregrounds all the aspects of the model; there are no modeling assumptions that need checking, not already foregrounded in the model description. Non-Bayesian statistics are powerful tools, and when used by an experienced statistician, they can dramatically reduce the process of inference. However, for researchers who wish to investigate their findings without access to a statistician, they need to be careful of the assumptions of the different tests: Normality ($t$-test); heteroscedasticity (e.g., ANOVA) and ensuring that the data satisfy the assumptions. Omitting the right sequence of analysis can lead to inferences not supported by the data.
    \item[Small $n$ studies:] A Bayesian model is valid at \textit{every} value of $n$; we do not have to wait for $n\geq 30$ to satisfy assumptions of say Normality. For small $n$ values, the result is of course affected by the choice of the prior; but by using weakly informative priors, we can ensure that the prior doesn't dominate inference. Furthermore, when Bayesian models use maximum entropy likelihood functions (e.g., members of the exponential family, that include the Normal distribution and the gamma distribution), we make the \textit{most conservative} inference given the data. See~\textcite[][Chapter 9]{McElreath2015} for an excellent description of the use of maximum entropy models in Bayesian analysis.
\end{description}


\subsection{Model Convergence and Normality}
\label{sub:Posterior Checks, Convergence and Normality}

\hs{In this section, we examine model convergence and our assumptions about Normality.}

The model shows good convergence, as evidenced by the traceplot in~\Cref{fig:traceplot}. The Gelman-Rubin statistic $\hat{R}$ was around 1, indicating that the different sampling chains converged. Furthermore, the effective sample size of all parameters was greater than 10,000.

\hs{We used the $t$-distribution to model likelihood motivated by the high contributions at \$0 and \$5 in the data (c.f.~\Cref{fig:contributions across conditions}}) implying that a heavy-tailed distribution may be a better likelihood function than a  Normal distribution. Let us examine the posterior distribution for $\nu$, the degrees of freedom of the $t$-distribution. As a reminder, the $t$-distribution is equivalent to the Normal distribution when $\nu=\infty$. We can see that while the 95\% HPD lies between [17.37, 147.19], less than 7\% of the posterior lies below $\nu=30$, the traditional rule-of-thumb in non-Bayesian statistics for use of the Normal distribution. \textcolor{red}{Since there is only a 7\% chance that the degrees of freedom $\nu \leq 30$, instead of using the Student-t likelihood function, instead, we could use a Normally distributed likelihood function with unequal variance.}


\begin{figure}[htb]
    \includegraphics[width=0.5\textwidth]{./hari-code/robust_normality.pdf}
    \caption{The posterior distribution for $\nu$, the normality parameter of the $t$-distribution. When $\nu=\infty$ the $t$-distribution is identical to the Normal distribution. The posterior distribution shows a vertical green bar for $\nu=30$, the traditional cut-off condition on degrees of freedom, for using a Normal distribution. The mode $\nu=53.42$, and the posterior probability distribution implies that only about 7\% of the posterior lies to the left of the green line (i.e. $P(\nu \leq 30) \approx 0.07$), implying that the assuming that the likelihood function to be Normally distributed will give similar inference.}
    \label{fig:normality}
\end{figure}

% Having discussed posterior-prediction checks, model convergence, and normality, next, we discuss alternatives to the model.

% \subsection{Alternative Models}
% \label{sub:Alternative Models}

% As a first instance, consider a similar model, except that the scale parameter of the likelihood function is \textit{not nested} like our current model. Instead, we consider the equal variance case, where all the variances are equal (similar to ANOVA), and that the variance is drawn from a uniform distribution. In other words, $\sigma_j = \sigma \sim U(L, H)$, where $L>0$ and where $H$ is a large constant. The main effect of the equal variance assumption is that there is no information sharing among the groups as would be the case when each $\sigma_j$ is drawn from the same distribution, whose parameters have hyper-priors; the latter is our current model.

% The main effect of constraining our simplified model is that we are slightly poorer in predicting the observed data since all the variances are guaranteed by the model to be equal, whereas we can see from~\Cref{fig:traceplot} that the mean scale (or equivalently variance) in the text condition is lower. We are skipping the traceplot and the contrast plots in this case, as they are similar, to~\Cref{fig:traceplot} and~\Cref{fig:robustcontrasts}, except that the effect size for the combined case is slightly lower due to the equal variance assumption. 

% Instead, we compare the two models using WAIC (Widely Applicable Information Criterion), a principled way to compare models when they have identical likelihood functions~\parencite{Gelman2014a}. WAIC uses the predictive loss to compare two models with different parameters. First, WAIC computes the average log likelihood of each training data point (over the posterior distribution) less the variance of the log likelihood for the same data point; and then it computes the sum over all data points. That is, WAIC for a model: 

% \begin{equation*}
%     \mathrm{WAIC} = -2 \left (\sum_i^N \log \mathrm{Pr}(y_i) - V(y_i) \right) 
% \end{equation*}

% where, $N$ is the total number of training points, $y_i$ is the observation, $\mathrm{Pr}(y_i)$ is the averaged data likelihood over the posterior, and where $V(y_i)$ is the variance of the data likelihood over the posterior. When we compare the two models, one with unequal variances, and one with equal variances. We show our results in~\Cref{tab:WAIC comparison}:

% \npdecimalsign{.}
% \nprounddigits{2}

% \begin{table}[htb]%\footnotesize
%     \centering
%         \caption{WAIC comparison between the model with equal variances against the case when the variances are not constrained to be equal (i.e. we use a hierarchical model). Both cases assume the Student-$t$ likelihood function, the function used in this paper. The columns show respectively, WAIC, pWAIC (the effective number of parameters; also: $\mathrm{pWAIC}=\sum_i V(y_i)$), dWAIC (the difference between the WAIC scores of the other models with the best model), weight (the relative probability that the model explains the data) SE, the standard error of the WAIC estimate, dSE is the standard error of the difference of the current model against the top model. The table shows that the hierarchical model with unequal variances better explains the observations. }\label{tab:WAIC comparison}
%         \begin{tabular}{rcccccc} \toprule
%             Model & WAIC & pWAIC & dWAIC & weight & SE & dSE \\ \midrule
%              Unconstrained variances, hierarchical    & 1102.48    & 3.95 &     0.00 &     1.00 &     14.61 &     0.00    \\
%             Constrained, equal variances & 1104.17 & 3.42 & 1.69 & 0.00 & 14.25 & 1.54        \\ \bottomrule
%         \end{tabular}
    
%     \end{table}

% The results in~\Cref{tab:WAIC comparison} say that model with the unconstrained variances is better at explaining the data than the model with constrained variances; the relative probability that the hierarchical model with unconstrained variances better explains the observation is 1.0 (refer to the weight column in~\Cref{tab:WAIC comparison})

% How useful was the choice of the Student-t drawing distribution~\Cref{eq:bayesian formulation}, instead of assuming that the outcomes are drawn from a Normal distribution? Our analysis of the posterior distribution of the degrees of freedom parameter $\nu$ shows that there is only a small probability ($P(\nu \leq 30) \approx 0.07$) that $\nu \leq 30$. Thus, we may use the Normal likelihood function, without meaningfully affecting the conclusions. Indeed, in our experiments, when we do model the observations with a Normal likelihood, assume equal variances, we find no meaningful differences in the contrasts or the effect sizes, with the Student-$t$ model with equal variances (results omitted due to space constraints).

% Since the charitable donations are bounded to lie between \$0.0 and \$5.0, might we benefit from using bounded likelihood functions like the Beta distribution $Y \sim Beta(\alpha_j, \beta_j)$ to represent the charitable donations $Y_j$ under the different conditions $j$? While a $Beta(\alpha, \beta)$ distribution lies between $[0,1]$, we can scale down the contributions to lie in $[0,1]$ to use with the $Beta(\alpha, \beta)$ distribution. But notice from~\Cref{fig:contributions across conditions} that in each experimental condition, there is a central lobe, and heavy tails at each extreme, notably at \$0.0 and at \$5.0. 

% Our view of models motivated by~\textcite{McElreath2015} is that they represent an \textit{epistemological} claim, not an \textit{ontological} claim (i.e. a physical assumption about the world). Since our goal is to understand the average tendency to give to charity under the different conditions, and not to make predictions (as might be the case if we were trying to model donation with age as a predictor), the fact that the $t$-distribution is not bounded is less relevant here. 



\textcolor{red}{Our posterior predictive check shows that the $t$-distribution models well the heavy tails, and shows that the variances for the comic conditions are different from the text condition. Furthermore, the comparison between the hierarchical model with unconstrained variances and the model with equal variances shows that the hierarchical model is better at explaining the observations. We discuss posterior prediction checks (\Cref{sub:Posterior Predictive Check}) and alternative models (\Cref{sub:Alternative Models}) in detail in the Appendix. }

To summarize, we discussed transparency and utility in small-$n$ studies as motivation for our use of Bayesian modeling. The model shows good convergence with the Gelman-Rubin statistic $\hat{R}$ was virtually identical to 1.0 for all parameters; the effective sample size was greater than 10,000 for all parameters. The modal value of $\nu$, the degrees of freedom parameter was around 53, suggesting that we could also use the Normal likelihood function. 

%!TEX root = cscw2019-comic.tex
\section{Discussion}
\label{sec:Discussion}
We will first summarize our findings. Then, we will propose a framework for algorithmically synthesizing persuasive messages into the abstract comic form, discuss design implications, and identify limitations.

From asking individuals to act to appealing for charitable donations, text messages have been widely used in simulating behaviors. Scholars from psychology have shown how variations in the construction of text messages alter decisions. We explored and examined the role of the abstract comic form, a highly expressive, affective medium in communicating persuasive messages. To test the effectiveness of abstract comic persuasive messages, we persuaded individuals to make online charitable donations, a common public good dilemma. In the dilemma, due to the non-exclusive and non-rivalrous nature of public goods, persuading individuals to contribute is hard and crucial. Also, online charitable donation task not only avoids confounding factors such as habit formation which exists in other persuasion tasks such as exercise and healthy diet but also assures the ecological validity of our study as charity and organizations often solicit donations online. In our study, we compared the persuasive power of text messages, comic messages, and comic messages with social proof in asking charitable donations to public health research (e.g., the Organization for Autism Research).

Our results show that study participants prefer persuasive messages in abstract comic form over plain text. When making the charitable donation between $\$0$ to $\$5$, study participants donated $\$ 0.86$ more if they read the persuasive message in an abstract comic form. The results demonstrate the persuasive power of abstract comic in stimulating behaviors in pro-social decisions. Our findings are consistent with prior research on visual stimuli in persuasion and the benefits of comics in communication. One potential explanation is that study participants were more attracted by the comic strip and projected themselves onto the character. When the projection happens, the persuadee may be able to digest the information better which stimulated them to donate more to the Organization for Autism Research. However, when comparing between the comic condition and comic with the social proof condition, although study participants who read the comic with social proof donated more on average, our analysis results did not imply meaningful effect size. Although the use of social proof showed a significant impact on other forms of persuasive messages (e.g., plain text), in the abstract comic messages, the effect is not substantial. One of the potential explanation is that the design of our comic strip did not signify the idea of social proof other than stating in the text bubble. It is worth to explore other comic designs, e.g., adding other donators as comic characters, that can make the social proof more salient. \textcolor{red}{Another potential explanation is that the influence from other study participants is not strong enough. As \textcite{goldstein2008room} discovered that the closer the relationship is, the stronger the social proof will be, comparing to the neighbors or people stayed at the same room in \textcite{Cialdini2004} and \textcite{goldstein2008room}'s work, the relationship among participants in our study may be much weaker. So, we did not observe a strong influence from abstract comics with social proof.}

Another natural question to ask is whether the maximum donation amount of  \$5  in our study has any influence on our findings (e.g., will people make the same donation decision if they can donate more?).  Prior studies in behavioral economics suggested that small-stake experiment in developed countries can be replicated in developing countries where the stake becomes a significant portion of participants' weekly income \cite{binswanger1980attitudes,binswanger1981attitudes,kachelmeier1992examining}. Additionally, \textcite{post2008deal} showed that the results from small-stake experiments would hold with higher stakes in developed countries as well. Those results seem to suggest that our findings will hold with higher donation amount. However, the best way to confirm is through actual experiments. 

\textcolor{red}{When considering persuasive messages in non-textual forms, besides abstract comics, graphical representation and video messages are often discussed. Although the scope our study is not to compare the persuasive power between comics and other non-textual forms, the abstract nature of the comics allows low-cost synthesizing comparing to other non-textual forms, especially when the original message is personalized, which makes the abstract comic persuasive message worth to consider.}

Although in our study, the persuasive goal is making charitable donation decisions, we believe our findings suggest future research on longitudinal behavior change (e.g., health) via the abstract comic. 

 \subsection{Framework for Algorithmically Synthesized Abstract-Comic Persuasive Messages}
 
Our study showed the persuasive power of abstract comics in encouraging people to make pro-social decisions. However, one drawback of using visual stimulus in persuasion is cost during the creation process. Although compared to creating other persuasive visual stimulus such as videos or complex graphical illustration, the simplicity of abstract requires less effort, the creation process is not easy. In this section, we propose a framework that allows full/semi- automatic generation of abstract persuasive messages and identify crucial features that need to be addressed in future work. We believe such a tool will lower the barrier for the persuader to take advantage of the abstract comic (as demonstrated in our study) in encouraging individuals to act in public good dilemmas.

In our study, we created a comic generator with existing packages including ``cmx.io'' \cite{cmx.io} and ``rough.js'' \cite{rough.js} to generate the three-panel abstract comic strip. With several pre-defined character gestures (see~\Cref{figur:figures}), the generator only requests the text input from the persuader to create the comic message.  

We believe the comic generator built in this study can be further developed as a framework for algorithmic synthesis. Now, we identify crucial features that need to be addressed in future research. First, the framework should be able to automatically select the character gesture that best fit the persuasive message's context. For example, when the message receiver was told good news, his/her gesture should reflect that conflict. The appropriate mapping will create a natural and coherent comic message which is the key for an expressive message. We need future research to realize such method. Second, the framework should be able to use other abstract comic elements, such as inter-character distance and shading, to create persuasive comics. To achieve this feature, we need to understand how different elements affect the persuasiveness of the abstract comic messages. Third, the framework should be able to personalize the persuasive comics by using persuadee's behavioral data to inform comic contents. We could derive the statistics for the social proof from the behaviors of friends and the data from a person's own activity.

\begin{figure}[t]
    \centering
    \begin{tabular}{cc}
        \subfloat[Gestures for positive framed messages]{\label{figur:1a}\includegraphics[width = 0.4\columnwidth]{figures/pos_figures}} &
        \subfloat[Gesture for negative framed messages ]{\label{figur:1b}\includegraphics[width = 0.4 \columnwidth]{figures/neg_figures}}\\
    \end{tabular}
    \caption{Different character gestures to communicate various levels of emotional intensity. The left figure shows gestures from neutral to the happiest. The right figure shows gestures from neutral to the most frustrated. }
    \label{figur:figures}
\end{figure}

\subsection{Design Implications}
Our main design implications are two folded. First, when encouraging individuals to contribute to public goods, non-profit organizations and governmental agencies ought to consider to use abstract comic in their online messaging strategies to persuade. From our results, using abstract comics can, in particular, persuade people to act in public-goods dilemmas that require single-shot decisions (online charitable donations). Additionally, when constructing persuasive comics, it is worthwhile to consider to incorporate social norms. \textcolor{red}{Although in the study, the purpose of the video is to inform participants about the context, in the real world, the abstract comic message can be combined with campaign videos online to solicit donation.  Organizations can also consider to combine abstract comic messages with other introductory materials such as texts in an email or a physical miller to persuade for contribution.}  
%; taking the flu shot
Second, our results also inform the development of a framework that can algorithmically synthesis textual persuasive messages into abstract comic form and uses data-driven methods to personalize the message to further increase messages' persuasive power. With such a framework, agencies can create their abstract comic persuasive messages easily and incorporate them to alleviate real-world public goods dilemmas. 

\subsection{Limitations \& Future Work}
\begin{description}

 \item[Distant and Non-exclusive Task:]  Although our study asked participants to make the decision with a real cost, the decision domain is limited to only one scenario, online charitable giving. In this task, the participant's reward is distant and non-exclusive (e.g., participant's donation decision won't bring any immediate reward and won't exclude the participant from the research outcome). Although both characteristics help us clearly test the persuasive benefits of abstract comics, they also limited our generalizability. We need future research to understand the pros and cons of using abstract comic messages in persuasive tasks with different reward characteristics. For example, for persuasive goals in the quantified self movement~\cite{Epstein2014,Choe2014} such as exercise and dieting, the reward is distant (e.g., people's health won't be improved immediately after exercise) but exclusive (e.g., healthy life situation mainly benefit the individual him/herself). Moreover, our persuasion task is for an online charitable donation. Future work is needed to extend our result to offline charitable donation solicitation where persuasive text messages, such as mailers, also often used. 

\item[Study on Amazon Mechanical Turk:] We recruited our participants from Amazon Mechanical Turk. Although research studies show populations on Amazon Mechanical Turk are diverse and mirror the US population \cite{buhrmester2011amazon,behrend2011viability,berinsky2012evaluating}, some researchers raises concerns about Amazon Mechanical Turk's sample representativeness \cite{landers2015inconvenient,paolacci2010running}. One potential solution is to use panel population where the panel company claimed to offer a more diverse and representative sample.
However, this method also has concerns in that the researcher can not directly cross-validate the sample's representativeness we have to trust the company's assertion. Another solution is using stratified sampling, but it is challenging to classify every member of the population into a subgroup properly. 

\textcolor{red}{Another limitation from Amazon Mechanical Turk participants on our study is that their primary motivation is monetary rewards \cite{paolacci2010running,paolacci2014inside}. Although many studies \cite{lee2013does,saunders2016no,sussman2015framing,arechar2017turking,branas2018gender} chosen Amazon Mechanical Turk to study charitable donations, compared to other recruiting methods, study participants from Amazon Mechanical Turk are more sensitive to monetary rewards. Especially in our study, we asked participants to donate from their perspective rewards, which makes the role of the persuasive message is more crucial in the decision making process. With a sample that is less sensitive to monetary rewards such as the researcher's own social network, we may expect a higher donation amount among all conditions. However, besides the potential sample representatives problem (e.g. the size of researcher's own social network), such sample may bring unexpected confounding factors such as their relationship with the researcher and knowledge about researcher's study topic.}     
  
 \item[Comic Message Construction:] In our study, we constructed the comic messages in the XKCD comic style. Although the XKCD style is easily recognizable, there are many different ways to create abstract comics.  Comic, as a creative art form, has rich semantics and vocabularies to communicate ideas \cite{scott1993understanding}. Out of necessity, we limited the complexity our comics strip (e.g., number of characters, gestures, and the number of panels). Future research is needed to explore and understand how other elements in the comic, for example, inter-character distance and character's gesture, affect the persuasiveness of the comic. Furthermore, while simplicity grants us the possibility to automate the generation process, future technology may allow us to generate more complicated persuasive abstract comics.
  
\end{description}

%need to update this part after the revision of RQ in the introduction
%Our experiment answers RQ-1 affirmatively in that comics are more persuasive than text, with a moderate effect of 0.33. Our answer to RQ-2 is that while the effect of no element is significant, shading and gesture show strong influence, but surprisingly inter-character distance is most effective when distance is large. For RQ-3, we show that negative messages are more influential than positive messages. We have developed a prototypical comic generator (answering RQ-4) that can be used in deploying comic messages.

%  \item[No interaction effects in the model]: Our model does not include any interaction effects. This is by design, since in the first study we have 54 experimental conditions making any analysis interaction effects difficult with our small observational study. The raw data suggest an interaction between shading and gesture, but given our limited dataset, there is little point in modeling this interaction. We plan to study interaction effects in future studies by limiting the number of main predictor conditions.
 %\item[Ecological Validity:] There is a legitimate question if our experiment on Amazon Mechanical Turk has ecological validity.  In real life, many factors affect a person's decision, such as where they are and who they are interacting with. Those factors may interact with the abstract comic's persuasiveness. However, these concerns are also present in other standard studies conducted in the more familiar lab experiments. We plan to conduct field studies in real-world contexts (e.g., shopping at a grocery store) to explore this issue.

% followed Our analysis shows that subjects prefer persuasive messages in comic form over plain text. We found in a persuasive comic, different character's gestures and background shading can influence subjects' perception of the persuasiveness whereas no strong effect was found in inter-character distance. This was consistent with previous research on visual stimulus in persuasion and the benefits of comics in communication.
%
% \subsection{Inter-Character Distance}
% \label{sub:Inter-Character Distance}
% There may be two explanations to the odd result that the farthest distance between the two characters was more influential.
% % However, previous studies on comics composition suggests the inter-character distance can affect reader's perceived relationship between characters and therefore influence their perception of the persuasive comics.
% First, it may be the case that the subjects did not project themselves onto the comic as one of the characters and did not recognize the distance between characters as reflecting closeness of the relationship. For example, a subject may read the comic from a third-person narrative. We looked into some feedbacks from in the pilot study. One participant ($p_1$) mentioned \textit{``I really like the comic I just saw and I feel bad that someone told me ...''} which suggests the subject does think that they are in the comic and having a conversation with someone else. Yet, we don't know if they perceive their relationship with the persuader based on the inter-character distance. A second explanation is that closer inter-character distance causes a cluttered visual composition and thus the subjects perceive these comics as less visually pleasing.

% \subsection{Comics with Color}
% \begin{figure}[t]
%  \centering
%  \begin{tabular}{ccc}
%   \subfloat[Colored Text]{\label{figur:3o}\includegraphics[width = 0.27\columnwidth]{figures/o1}} &
%   \subfloat[Colored Ground Line]{\label{figur:31}\includegraphics[width = 0.27 \columnwidth]{figures/o2}}      &
%   \subfloat[Colored Figure]{\label{figur:32}\includegraphics[width = 0.27 \columnwidth]{figures/o3}}\\
%  \end{tabular}
%  \caption{Colored different elements in a comic}
%  \label{figur:color}
% \end{figure}
%
% Our model suggests the background shading in a persuasive comic affects its persuasive power which makes us wonder the role of color as previous studies suggest the usage of color communicates the emotional intensity similar to the background shading. We ran a small scale study comparing the perceived persuasiveness between black-white comics in our study and their corresponding colored version,see figure~\ref{figur:color}. With 60 participants from the Mechanical Turk,  we found using an identical hierarchical Bayesian formulation that our subjects perceive colored version as more persuasive and there is potential interaction effect between negative-positive framing and different colors (see~\Cref{fig:color-experiment-effect}). We plan to run larger experiments that include color and study the interaction with framing and other elements.
%
% \begin{figure}
%  \subfloat[The mean effect and the effect size with color\label{subfig-1:color-mean-effect}]{%
%   \includegraphics[width=0.6\textwidth]{./hari-code/factors_mean_effect_color-no-interaction.pdf}
%   } \hfill
%  \subfloat[Color contrast\label{subfig-2:color-contrast}]{%
%   \includegraphics[width=0.33\textwidth]{./hari-code/factors_color_contrasts_color-no-interaction.pdf}
%  }
%  \caption{~\Cref{subfig-1:color-mean-effect} shows the High Posterior Density (HPD) intervals for the mean response $\mu$ and effect sizes $\sigma_y$ in the presence of color. HPD represent the region with 95\% of the density. Notice that the HPD interval for $\mu$ is $[3.94, 4.27]$ and includes a ROPE of $[4\pm 0.1]$ (the interval includes 4, the neutral response value). Thus while there no significant effect, we note that nearly 94\% of the HPD lies to the right of 0. The figure for effect size shows a small effect with mode $0.19$; since the HPD interval $[-0.04, 0.43]$ includes a ROPE of $[0\pm 0.1]$, there is no significant effect.~\Cref{subfig-2:color-contrast} shows the contrasts between the use of the two colors. The modal value is $0.24$, but since the HPD interval $[-0.13, 0.69]$ overlaps with 0, there is no appreciable effect (but notice that 89\% of the density lies in the region greater than 0.)}
%  \label{fig:color-experiment-effect}
% \end{figure}


%  Large scale field study is needed to demonstrate how the abstract comic can persuade in the real-life decision making and what will affect its persuasiveness.
 %Since we ask the subjects (who may lack interest in exercise) to evaluate persuasiveness when the topic is on exercise. Notice that our goal is not to persuade experimental subjects to exercise more, but to evaluate if the comic is a more persuasive form of communication of a statistical fact. We should expect---since we don't know if the subjects are interested in exercise---an increase in the variance in the estimates of the parameters (in particular,  $\sigma_y$). Despite this, the analysis shows a significant affirmative result for RQ-1.
%  \item[Appropriate gestures]: The authors determined gestures used by the characters in the experiment through trial and error. It may be useful to examine  theoretical frameworks from dance as well as theater art forms as well as examine work on design of sign languages.
% \end{description}
 % \item[Single Panel]: Our experiment limits our comics to a single panel which hinders one of the most fascinating aspect of comics---storytelling. Comparing to the comic strip, single panel comics find it harder to show the dynamic among characters.

%The main implication of our study lies in how to deliver persuasive messages. 

%Given the increasing popularity of wearable devices including smartwatches, one could consider presenting persuasive messages in the form of an abstract comic, where the three-panel comic is shown in sequence, perhaps even in animation. 

%Also, ~\textcite{scott1993understanding} identifies several fundamental components that influence reader reaction: character gestures, inter-character distance, and shading intensity. For example, the gesture of a character can help the reader to understand the interaction between characters and the emotion of the character. As a common technique, cartoonists often use the gesture to intensify the feeling that they want to communicate to the reader \cite{scott1993understanding}. Therefore, one of the future direction should be understanding how the different comic components moderate comic persuasiveness. If different comic elements car persuades differently, taking advantage of data-driven methods, the abstract-comic can better persuade. To maximize the persuasive power, future research on computational persuasion should leverage the receiver's personal model to construct the persuasive abstract comics with elements best fit to the persuadee.

%!TEX root = cscw2019-comic.tex
\section{Conclusion}
\label{sec:Conclusion}




%The main implications of our work lie in how persuasive messages are delivered, especially to wearable devices. As next steps, we plan to develop an algorithmic framework that automatically maps a person's behavioral data (e.g. amount walked this week) to a three-panel persuasive comic. We also plan to conduct longitudinal field experiments with an emphasis on storytelling where individuals receive three-panel comics over time, and comics are connected with a storyline.


%Three ideas were key to our work: the role of abstract comics to allow readers to project themselves onto the comic and ``social proof'' (that we adopt the decisions from other donaters) .


%The study on charitable giving examined if the abstract comic form was more persuasive than text. We analyzed the results using a hierarchical Bayesian framework that allows for understanding effect sizes, as well are helpful in small-$n$ studies. The results shows convincingly that the three-panel abstract comic is more persuasive than the text (medium effect size=0.44). The presence of the social proof is also effective, but provides only a minor improvement to the use of the abstract comic (effect size =0.12; not a meaningful improvement). The main implications of our work lie in how persuasive messages are delivered, especially to wearable devices.

% comics have a significant but moderate effect (effect size: 0.33) on persuasiveness compared to text. While no comic element has significant effect, we can observe that non-neutral gestures and shading as well as negatively framed messages have a strong influence. We conducted a smaller study with color and the result indicates that color too has a strong influence. Finally, we developed an abstract comic panel generator that takes as input the different comic elements, the valence of the frame and the message. We plan to release the code for Bayesian analysis and comic generator and the raw data under an appropriate open source license.

As next steps, we plan to develop an algorithmic framework that automatically maps a person's behavioral data (e.g. amount walked this week) to a three-panel persuasive comic. We also plan to conduct longitudinal field experiments with an emphasis on storytelling where individuals receive three-panel comics over time, and comics are connected with a storyline.

\printbibliography
\end{document}
