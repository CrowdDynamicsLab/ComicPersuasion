%!TEX root = cscw2019-comic.tex
\section{Related Work}
\label{sec:relatedwork}
Now, we discuss related work in: 1) Nudge for The Public Good. 2) Persuasion through Visual Stimuli. and 3) Comics Convey.

\subsection{Nudge for the Public Good}

Given the two characteristics of public goods, non-excludability and non-rivalry, it is irrational for individuals to contribute in collective action dilemmas such as charitable donations ~\cite{MarwellandAmes1981}. The free-rider problem will result in an underprovision of the public goods ~\cite{}. Therefore, external nudges play an important role in encouraging people to contribute. Researchers and policymakers have extensively studied who contributes and how to persuade people to contribute ~\cite{}. Besides education and income, key factors that drive people to contribute were intrinsic motives including altruism ~\cite{Olson1965}, sympathy ~\cite{Becker1974}, social pressure, religiousness ~\cite{Pessemier1977} and self-esteem ~\cite{Burnett1981}. Therefore, to construct effective persuasive messages, it is important to make the persuadee reflect upon those motives.  Midden et al. ~\cite{} reported strong persuasive power for environmental sustainable behavior when signaling personal goals in persuasive applications. Feiler et al. found emphasizing altruistic reason in donation request can elicit more donations from people. Therefore, when nudging people for collective actions, it is important to make the persuadee reflect upon their intrinsic motives in the messages. In our study, we believe the abstract comics allows people to project themselves onto the character which gives us the oppurtunities to nudge them to reflect upon their intrinsic motives when making decisions about cooperating in collective actions. 

Pure textual messages were one of the most widely used method to persuade individuals for behavior change or voluntarily chartibles donations. It is easy to create and dissminate. Long history of reserach shows using textual messages were effective in persuasion. ~\textcite{damgaard2017now} successfully used a simple email reminder with the decision deadline to elicit chartibles donation from the participants. With a simple sign like "Turn off tap when not in use", people reduced water consumption and engaged in energy conservation behavior change ~\cite{mckenzie2011fostering}. ~\textcite{cotterill2010impact} showed sending a pledge card with simple text "A list of everyone who donates a book will be displayed locally." encourged 22 \% more households to donate books for Children's Book Week. However, comparing persuasive messages in other forms such as graphics, texual messages were harder to catch perusadee's attention, especailly when persudaee's attention is limited. Moreover, when comes to memorability, one of the key measure of persuasive effect, studies found comparing to other medium such as graphics and video, text is most difficult to recall and recollect. Therefore, persusaisve messages in other meidum such as abstact comic is worth to inviestigate. 

\subsection{Persuasion Through Visual Stimuli}
Beyond the realm of textual forms, prior research shows that using visual representations, e.g. graphics, video, and comics, are attractive in persuasion. ~\textcite{selker2015sweetbuildinggreeter} used motivational graphics or memes from 9GAG and Google to attract people's attention and persuade people for energy saving behaviors. \textcite{consolvo2008activity} presents user's exercise data as visual elements in the Ubifit Garden to persuade people exercise. Sometimes, the use of visual can evoke strong emotions that makes the message more persuasive. ~\textcite{iyer2006picture} found that people who saw the images of the Kenneth Bigley kidnapping were more engaged in the later civic chamapign than those who read about the kidnapping from the texts of the newspaper. The strong emotion carried by the photographs brought people not only fear but also engagement and concern. \textcite{}'s visual rhetoric effectvely persuade users to use neep-to-date antivirus protection. Visual stimuli were more memorable as well \cite{rossandniestt1980}. The use of image makes an advertisements more memoriable and appleaing. The visuals can leave a strong trace which may later on influence people's decision making, especailly when people making judgements by the availability heuristics. ~\textcite{sanortia} found the video in crowed funding campaign plays an important role in persauding people to support. However, creating visual stimuli is often costly. The persuador needs to put time, effort and resource to create persuasive visual stimuli. No need to mention, thousands of considerations needs to be cautious about in order to make the viusal persuasive messages appealing. In our study, we chose abstract comic, a simple visual stimuli that can be made easibly and exmianed its persuasiveness in encourageing people to partcipate collective actions. And we also proposed a general framework that allows the persaudor to systhezsie their persuasive messages into abstract comic form easily. 


\subsection{Comics Convey}
While reading comics book is commonly recognized as entertaining, comics have been examined as an effective way of communicating abstract and complex ideas to broad audiences \textcite{McDermottPB18,cary2004going,scott1993understanding}. The simple and humorous nature of comic makes comics becomes an unique media for delivering informative and memorable messages. By combining visual elements and texts, comics makes the story more appealing. \textcite{McDermottPB18} used comics to illustrate complex scientific facts. In education, comics have been used and examined as an effective tool for reaching different populations with various background \textcite{McDermottPB18,cary2004going,scott1993understanding}. The use of metaphor in comics can make the underlying meaning vivid and more memorable than using a straightforward description \textcite{McDermottPB18,scott1993understanding}. Moreover, comics can contain a personal story incredibly powerful for persuasion~\textcite{weaver2017losing}. With a personal story, comics can create empathy for readers.~\textcite{matsubara2016emotional} showed a link between comic's content and the emotions felt by the readers. Thus complex messages can be easily interpreted and memorized though use of the comic form. Although comics has shown its strength in communicating complext ideas, its utlity in persuasion has not yet been fully explored. To our knowledge, we are the first study explored the abstact comics in persuading people to cooperate in collective action deliminas.  


% \subsection{Building Persuasive Technology}
% Starting from~\textcite{goehlert1980information}, HCI researchers have spent a lot of effort in leveraging technology in persuasion.~\textcite{goehlert1980information} argues control and dissemination of information have the ability to make attitude and behavioral changes. Inspired by this argument, two approaches have been explored in constructing the persuasive systems. Information-centric approaches focused on delivering hidden or new information which has not been perceived by the user before~\cite{LeeKF11}. For example,~\textcite{chi2007enabling} changed people's nutritional composition by creating an intelligent kitchen that can provide nutritional information about ingredients while participants are cooking.  ~\textcite{liao2014expert} found showing a source expertise indicator can shape user's information seeking and burst the filter bubble. While lots of studies on persuasive technology showcase the persuasive power of the information-centric approach, studies also show the downside of information-center approach where the target receiver often failed to perceive and internalize the persuasive information ~\cite{goehlert1980information,LeeKF11}.

%Adapting decision-making models from previous behavioral research, behavior-centric approach emphasized on human motivation and biases ~\cite{LeeKF11}.~\textcite{vaish2018s} used self-serving motivational framing of messages to persuade people to join a prosocial peer-to-peer service. Although the effectiveness of behavior-centric approach has been examined in multiple studies, behavior-centric approaches often require prior knowledge of the persuadee in order to unleash the persuasion power, as~\textcite{orji2014developing} and~\textcite{schneider2016understanding} suggested that different people may be more amenable to one persuasive method than others. Therefore, due to the cost of knowing people, scalability is the key challenge.

% \subsection{Persuasion Through Visual Stimuli}
% Prior research shows that using visual representations are more attractive than text messages~\cite{selker2015sweetbuildinggreeter,consolvo2008activity}. ~\textcite{selker2015sweetbuildinggreeter} retrieves motivational images from 9GAG, Google, with energy-saving messages, to motivate people's energy saving behaviors and found those images has higher persuasive power comparing to plain text. However, images used in persuasion are costly to generate. 

% The simple and humorous nature of comic makes comics becomes an unique media for delivering informative and memorable messages. Beyond entertainment, comics can be used in scientific communication~\cite{McDermottPB18}, and for teaching in multilingual schools~\cite{cary2004going}. To the best of our knowledge, this is the first study to look at abstract comic representations and persuasion.
% The simple and humorous nature of comic makes comics becomes an unique media for delivering informative and memorable messages. While reading comics book is commonly recognized as entertaining, comics have been examined as an effective way of communicating abstract ideas to broad audiences \cite{McDermottPB18,cary2004going,scott1993understanding}. McDermott et. al used comics to illustrate complex scientific facts \cite{McDermottPB18}. In education, comics have been used and examined as an effective tool for reaching different populations with various background \cite{McDermottPB18,cary2004going,scott1993understanding}.

% Similar to images, the simple and humorous nature of comic makes comics becomes an unique media for delivering informative and memorable messages. Comics have also been examined as an effective way of communicating abstract ideas to broad audiences \cite{McDermottPB18,cary2004going,scott1993understanding}. To our knowledge, no prior study examined the persuasive power of the comic.

% We aim to fill this gap by comparing
%
% the persuasiveness between
%
% Our work aims to fill this gap by comparing the quality ofsurvey content and respondents’ engagement between theusage of a conversational survey and a traditional survey.With a few exceptions [15,16], previous work evaluatingchatbot technologies tend to rely on small-scale lab studies.Instead, we deploy a survey chatbot and study its real-worldusage of conducting marketing research survey. Our analysisis performed on the response contents and behavioral logs(e.g., time stamps), and focus on measures that are crucialfor the purpose of gathering high-quality survey data
% behavior-centric approach is scalability.
% This study mainly focuses on different comic elements' effect on persuading readers.
%
% Although persuasion can be better achieved through visual stimulus, the cost of creating personalized visual stimuli is high. Our approach leverages abstract comics taking advantage of visual stimuli while allowing for straightforward, personalized synthesis.
%
%
% Beyond entertainment, comics can be used in scientific communication~\cite{McDermottPB18}, and for teaching in multilingual schools~\cite{cary2004going}.
% The simple and humorous nature of comic makes comics becomes an unique media for delivering informative and memorable messages. While reading comics book is commonly recognized as entertaining, comics have been examined as an effective way of communicating abstract ideas to broad audiences \cite{McDermottPB18,cary2004going,scott1993understanding}. McDermott et. al used comics to illustrate complex scientific facts \cite{McDermottPB18}. In education, comics have been used and examined as an effective tool for reaching different populations with various background \cite{McDermottPB18,cary2004going,scott1993understanding}.
% The use of metaphor in comics can make the underlying meaning vivid and more memorable than using a straightforward description \cite{McDermottPB18,scott1993understanding}. Moreover, comics can contain a personal story incredibly powerful for persuasion~\cite{weaver2017losing}. With a personal story, comics can create empathy for readers.~\textcite{matsubara2016emotional} showed a link between comic's content and the emotions felt by the readers. Thus complex messages can be easily interpreted and memorized though use of the comic form.
