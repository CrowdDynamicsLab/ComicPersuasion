%!TEX root = cscw2019-comic.tex
\section{Related Work}
\label{sec:relatedwork}
Now, we discuss related work in 1) Using Comics to Communicate Ideas, 2) Persuasion through Visual Stimuli, 3) Persuasion for Public Goods, and 4) Social Proof in Persuasive Messages.

\subsection{Using Comics to Communicate Ideas}
\textcite{scott1993understanding} defines comics as ``juxtaposed pictorial and other images in deliberate sequence, intended to convey information and/or to produce an aesthetic response in the viewer''. Beyond obvious entertainment value, comics have been examined as an effective way of communicating abstract and complex ideas to broad audiences \cite{McDermottPB18,cary2004going,scott1993understanding, Zhang-Kennedy:2017:SCI:3206217.3206282}. On the one hand, the simple and humorous nature of comic makes comics becomes a unique media for delivering informative and memorable messages. By combining visual elements and texts, comics make the story more appealing. \textcite{McDermottPB18} used comics to illustrate complex scientific facts. In education, comics have been used to reach populations with various backgrounds \cite{McDermottPB18,cary2004going,scott1993understanding}. \textcite{Zhang-Kennedy:2017:SCI:3206217.3206282} created ``Secure Comics'' to educate end-users with computer security knowledge. On the other hand, the use of metaphor in comics can make the underlying meaning vivid and more memorable than using a straightforward description \cite{McDermottPB18,scott1993understanding}. Moreover, comics can contain a personal story incredibly powerful for creating empathy, a key factor in persuasion, for readers ~\cite{weaver2017losing}. ~\textcite{matsubara2016emotional} showed a link between the comic's content and the emotions felt by the readers. Thus complex messages can be easily interpreted and memorized through the use of the comic form. Although comics have shown strength in communicating complex and memorable ideas, their utility in persuasion has not yet been fully explored. To the best of our knowledge, we are the first study to examine the use of abstract comics in persuading people to cooperate in public goods dilemmas.  

In our study, we chose abstract comics to persuade instead of other comic forms for the following reasons, First, as the comic becomes more abstract, readers will be more likely to project themselves onto the character and thus allowing the readers to empathizing with the character \cite{scott1993understanding}. Second, comparing to other forms of the comic, the abstract comic contains the fewest visual elements; the simplicity also reduces reader's cognitive load to consume the message; important since persuadee's attention is scarce \cite{Janssen2016}. Moreover, the simplicity of abstract comics allows us to explore the idea of algorithmically synthesizing persuasive messages algorithmically into comic form. Therefore, in this study, we examined the persuasive power of messages in abstract comic form.

\subsection{Persuasion Through Visual Stimuli}
Beyond the realm of textual forms, prior research shows that using visual representations, e.g., graphics, video, and comics, are attractive in persuasion. \textcite{selker2015sweetbuildinggreeter} used motivational graphics or memes from 9GAG and Google to attract people's attention and persuade people for energy saving behaviors. \textcite{consolvo2008activity} presents the user's exercise data as visual elements in the Ubifit Garden to persuade people to exercise. Sometimes, the use of visual can evoke strong emotions that makes the message more persuasive. ~\textcite{iyer2006picture} found that people who saw the images of the Kenneth Bigley kidnapping were more engaged in the later civic campaign than those who read about the kidnapping from the texts of the newspaper.\textcite{zhang2014stop}'s visual rhetoric effectively persuade users to use up-to-date antivirus protection. Visual stimuli were more memorable as well \cite{nisbett1980human}. The use of images makes advertisements more memorable and appealing. The visuals can leave a strong trace which may later on influence people's decision making, especially when people making judgments by the availability heuristics. ~\textcite{dey2017art} found the video in the crowdfunding campaign plays a vital role in persuading people to support. However, creating visual stimuli is often costly. The persuader needs to put time, effort and resources to create persuasive visual stimuli. In our study, we looked into the abstract comic, simple visual stimuli that can be algorithmically synthesized and examined its persuasiveness in encouraging people to participate in online charitable donations. 
%The strong emotion carried by the photographs brought people not only fear but also engagement and concern.
%No need to mention, thousands of considerations needs to be cautious about in order to make the visual persuasive messages appealing. 
\subsection{Persuasion for Public Goods}
Given the two characteristics of public goods, non-exclusive and non-rivalrous, individuals will receive no tangible benefits when acting in public goods dilemmas such as charitable donations ~\cite{marwell1981economists,isaac1982public}. Therefore, external nudges play an important role in encouraging people to contribute. Researchers and policymakers have extensively studied who contributes and how to persuade people to contribute ~\cite{olson2009logic,becker1974theory,andreoni1990impure,miguel2005ethnic,burnett1981psychographic,pessemier1977willingness,burnett1981psychographic}. ~\textcite{midden2008using} reported strong persuasive power for environmental sustainable behavior when signaling personal goals in persuasive applications. \textcite{feiler2012mixed} found emphasizing altruistic reasons in donation requests can elicit more donations. \textcite{mankoff2010stepgreen} successfully used social technologies to leverage public commitment and competition in appealing energy-saving behaviors. Due to the distant or non-reward nature of individuals' public goods contribution, persuasive messages were the key. 
%Therefore, soliciting online charitable donations, a common case in collective action dilemmas, is suitable for us to test the persuasive power of abstract comics.
% when nudging people for collective actions, it is important to make the persuadee reflect upon their intrinsic motives in the messages. In our study, we believe the abstract comics allows people to project themselves onto the character which gives us the oppurtunities to nudge them to reflect upon their intrinsic motives when making decisions about cooperating in collective actions. 
%Besides education and income ~\cite{pessemier1977willingness,burnett1981psychographic}, key factors that drive people to contribute were intrinsic motives including altruism ~\cite{olson2009logic,andreoni1990impure}, sympathy ~\cite{becker1974theory}, social pressure ~\cite{miguel2005ethnic}, religiousness ~\cite{pessemier1977willingness,burnett1981psychographic} and self-esteem ~\cite{burnett1981psychographic}.

Persuasive text messages are one of the most widely used methods to persuade individuals for voluntary charitable donations. They are easy to create and disseminate. ~\textcite{damgaard2017now} successfully used simple email reminders with the decision deadline to elicit charitable donation from the participants. With a simple sign like ``Turn off the tap when not in us'', people reduced water consumption and engaged in energy conservation behavior change ~\cite{mckenzie2011fostering}. ~\textcite{cotterill2010impact} showed sending a pledge card with simple text ``A list of everyone who donates a book will be displayed locally.'' encouraged 22\% more households to donate books for Children's Book Week. However, comparing persuasive messages in other forms such as graphics, textual messages were harder to catch perusadee's attention, especially when persudaee's attention is limited. Moreover, when it comes to memorability, a key measure of persuasive effectiveness, studies found that in comparison to other media such as graphics and video, the text is most difficult to recall and recollect. Therefore, persuasive messages in other media form such as the abstract comic are worth investigating. 


\subsection{Social Proof in Persuasive Messages }
By \textit{social proof}, we refer to the idea that individuals observing either their friends or people with whom they can relate adopted a behavior is persuasive for them to adopt the same behavior~\cite{Cialdini1993, Cialdini2004}. The use of social proof is widely used in encouraging individuals to cooperate in collective action dilemmas \cite{goldstein2008room,schultz2007constructive}. ~\textcite{goldstein2008room} conducted a famous experiment in a hotel on motivating environmental conservation. They found that descriptive norms (e.g., ``the majority of guests reuse their towels'') has more persuasive power than solely mentioning environmental protection. And this normative message gets more effective when the statement is about a provincial norm (e.g.,``the majority of guests in this room reuse their towels''). \textcite{amblee2011harnessing} studied social proof among online book reader communities and found that 
``electronic word of mouth '' affects a book's quality, an author's reputation, and a book category's popularity which eventually influenced people's buying decision. Since the use of social proof is one of the most widely used influence weapons in creating persuasive messages, we want to see their effect when included in the persuasive messages in the abstract comic form. 


%Also, we also proposed a general framework that allows the persuader to synthesize their persuasive messages into an abstract comic form easily. 




%As we presented in earlier sections, the comic allows for delivering memorable messages~\cite{scott1993understanding, Zhang-Kennedy:2017:SCI:3206217.3206282} and communicate complex ideas and emotion~\cite{McDermottPB18,cary2004going,scott1993understanding, Zhang-Kennedy:2017:SCI:3206217.3206282,weaver2017losing,matsubara2016emotional}. We believe comic is a powerful media to deliver persuasive messages.  



% \subsection{Building Persuasive Technology}
% Starting from~\textcite{goehlert1980information}, HCI researchers have spent a lot of effort in leveraging technology in persuasion.~\textcite{goehlert1980information} argues control and dissemination of information have the ability to make attitude and behavioral changes. Inspired by this argument, two approaches have been explored in constructing the persuasive systems. Information-centric approaches focused on delivering hidden or new information which has not been perceived by the user before~\cite{LeeKF11}. For example,~\textcite{chi2007enabling} changed people's nutritional composition by creating an intelligent kitchen that can provide nutritional information about ingredients while participants are cooking.  ~\textcite{liao2014expert} found showing a source expertise indicator can shape user's information seeking and burst the filter bubble. While lots of studies on persuasive technology showcase the persuasive power of the information-centric approach, studies also show the downside of information-center approach where the target receiver often failed to perceive and internalize the persuasive information ~\cite{goehlert1980information,LeeKF11}.

%Adapting decision-making models from previous behavioral research, behavior-centric approach emphasized on human motivation and biases ~\cite{LeeKF11}.~\textcite{vaish2018s} used self-serving motivational framing of messages to persuade people to join a prosocial peer-to-peer service. Although the effectiveness of behavior-centric approach has been examined in multiple studies, behavior-centric approaches often require prior knowledge of the persuadee in order to unleash the persuasion power, as~\textcite{orji2014developing} and~\textcite{schneider2016understanding} suggested that different people may be more amenable to one persuasive method than others. Therefore, due to the cost of knowing people, scalability is the key challenge.

% \subsection{Persuasion Through Visual Stimuli}
% Prior research shows that using visual representations are more attractive than text messages~\cite{selker2015sweetbuildinggreeter,consolvo2008activity}. ~\textcite{selker2015sweetbuildinggreeter} retrieves motivational images from 9GAG, Google, with energy-saving messages, to motivate people's energy saving behaviors and found those images has higher persuasive power comparing to plain text. However, images used in persuasion are costly to generate. 

% The simple and humorous nature of comic makes comics becomes an unique media for delivering informative and memorable messages. Beyond entertainment, comics can be used in scientific communication~\cite{McDermottPB18}, and for teaching in multilingual schools~\cite{cary2004going}. To the best of our knowledge, this is the first study to look at abstract comic representations and persuasion.
% The simple and humorous nature of comic makes comics becomes an unique media for delivering informative and memorable messages. While reading comics book is commonly recognized as entertaining, comics have been examined as an effective way of communicating abstract ideas to broad audiences \cite{McDermottPB18,cary2004going,scott1993understanding}. McDermott et. al used comics to illustrate complex scientific facts \cite{McDermottPB18}. In education, comics have been used and examined as an effective tool for reaching different populations with various background \cite{McDermottPB18,cary2004going,scott1993understanding}.

% Similar to images, the simple and humorous nature of comic makes comics becomes an unique media for delivering informative and memorable messages. Comics have also been examined as an effective way of communicating abstract ideas to broad audiences \cite{McDermottPB18,cary2004going,scott1993understanding}. To our knowledge, no prior study examined the persuasive power of the comic.

% We aim to fill this gap by comparing
%
% the persuasiveness between
%
% Our work aims to fill this gap by comparing the quality ofsurvey content and respondents’ engagement between theusage of a conversational survey and a traditional survey.With a few exceptions [15,16], previous work evaluatingchatbot technologies tend to rely on small-scale lab studies.Instead, we deploy a survey chatbot and study its real-worldusage of conducting marketing research survey. Our analysisis performed on the response contents and behavioral logs(e.g., time stamps), and focus on measures that are crucialfor the purpose of gathering high-quality survey data
% behavior-centric approach is scalability.
% This study mainly focuses on different comic elements' effect on persuading readers.
%
% Although persuasion can be better achieved through visual stimulus, the cost of creating personalized visual stimuli is high. Our approach leverages abstract comics taking advantage of visual stimuli while allowing for straightforward, personalized synthesis.
%
%
% Beyond entertainment, comics can be used in scientific communication~\cite{McDermottPB18}, and for teaching in multilingual schools~\cite{cary2004going}.
% The simple and humorous nature of comic makes comics becomes an unique media for delivering informative and memorable messages. While reading comics book is commonly recognized as entertaining, comics have been examined as an effective way of communicating abstract ideas to broad audiences \cite{McDermottPB18,cary2004going,scott1993understanding}. McDermott et. al used comics to illustrate complex scientific facts \cite{McDermottPB18}. In education, comics have been used and examined as an effective tool for reaching different populations with various background \cite{McDermottPB18,cary2004going,scott1993understanding}.
% The use of metaphor in comics can make the underlying meaning vivid and more memorable than using a straightforward description \cite{McDermottPB18,scott1993understanding}. Moreover, comics can contain a personal story incredibly powerful for persuasion~\cite{weaver2017losing}. With a personal story, comics can create empathy for readers.~\textcite{matsubara2016emotional} showed a link between comic's content and the emotions felt by the readers. Thus complex messages can be easily interpreted and memorized though use of the comic form.
