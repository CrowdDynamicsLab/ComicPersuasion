This paper examines whether persuasive messages when expressed in abstract comic form are more powerful in encouraging people to make charitable donation decisions online regarding health research. Since donation to charitable causes as a specific case of collective action, inherits the non-exclusive and non-rivalry nature of public goods, persuading individuals to contribute is important and hard. Responses to fund-raising appeals by charities remain low. In this study, we explore a novel form of persuasive messaging, the abstract comic. Whether comic representations, despite widespread use in popular culture, offer any tangible benefits over plain text messages in persuasion is unclear and motivates our study. Drawing on a rich history of the comics, we synthesize persuasive messages in an abstract comic form. We conduct a between-subject Amazon Mechanical Turk study with 307 participants on the use of abstract comic form to appeal for charitable donation. We compared the amount of donation solicited by three messages, a message in pure-textual form, a message in abstract comic form, and a message in abstract comic form but incorporating the idea of social proof. Motivated by model transparency and use in small-sized studies, we use a hierarchical Bayesian model to analyze the results. Our experiments reveal that abstract comic form solicits more donations than text (medium to large effect size=0.59). Incorporating social proof in the abstract comic message did not show meaningful effect. Our studies have design implications: charities ought to consider including messages in the abstract comic form as part of their overall communications intended for soliciting donations online.