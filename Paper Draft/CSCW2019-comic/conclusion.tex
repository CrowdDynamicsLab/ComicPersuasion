%!TEX root = cscw2019-comic.tex
\section{Conclusion}
\label{sec:Conclusion}

This paper examined if the abstract comic form was more persuasive that the corresponding plain text in encouraging participants donate for chartible cause.  Three ideas were key to our work: the role of abstract comics to allow readers to project themselves onto the comic and ``social proof'' (that we adopt the decisions from other donaters) .

We conducted a field study on Amazon Mechanical Turk. The study on charitable giving examined if the abstract comic form was more persuasive than text. We analyzed the results using a hierarchical Bayesian framework that allows for understanding effect sizes, as well are helpful in small-$n$ studies. The results shows convincingly that the three-panel abstract comic is more persuasive than the text (medium effect size=0.44). The presence of the social proof is also effective, but provides only a minor improvement to the use of the abstract comic (effect size =0.12; not a meaningful improvement). The main implications of our work lie in how persuasive messages are delivered, especially to wearable devices.

% comics have a significant but moderate effect (effect size: 0.33) on persuasiveness compared to text. While no comic element has significant effect, we can observe that non-neutral gestures and shading as well as negatively framed messages have a strong influence. We conducted a smaller study with color and the result indicates that color too has a strong influence. Finally, we developed an abstract comic panel generator that takes as input the different comic elements, the valence of the frame and the message. We plan to release the code for Bayesian analysis and comic generator and the raw data under an appropriate open source license.

As next steps, we plan to develop an algorithmic framework that automatically maps a person's behavioral data (e.g. amount walked this week) to the three panel comic. We also plan to conduct longitudinal field experiments with an emphasis on storytelling where individuals receive three-panel comics over time, but the panels are connected with a storyline.
