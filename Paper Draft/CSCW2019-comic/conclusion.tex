%!TEX root = cscw2019-comic.tex
\section{Conclusion}
\label{sec:Conclusion}

Inspired by a rich history in persuasive message construction and benefits of the abstract comic in communication, this paper examined if messages in abstract comic form are more persuasive than the corresponding plain text in the context of online charitable donations. Persuading individuals to contribute to charitable causes online is hard and responses to appeals are typically low, as charitable donations share the structure of public goods dilemmas where the rewards are distant and non-exclusive. Consider the low conversion (0.3 \%) rate reported in Wikipedia's online fundraising campaign \cite{wikimeta}, even a small increase in the conversion rate will make a huge difference. 

Considering the abstract comics as a highly expressive and affective medium, we algorithmically synthesized a three-panel abstract comic to create our appeal. We then conducted a field study on Amazon Mechanical Turk with 307 participants on the use of abstract comic form to appeal for charitable donation. We compared the average donation to the charity under three conditions: the plain text message, an abstract comic that includes the plain text, and an abstract comic that additionally includes the social norm. Motivated by the model transparency and its use in small-sized studies, we analyzed the results using a hierarchical Bayesian framework. The results show convincingly that the abstract comic is more persuasive than the text (a medium to large effect size = $0.59$). We also show that while the comic with social norm increases the donation level over the comic without the norm, the effect size is very small ($0.11$) and the increase is not meaningful. To summarize, the comic form meaningfully increases donations over the plain text, but the presence of the norm is not effective. We caution that the result holds for single-shot, public goods tasks; the value of the social proof in the comic, for exclusive tasks with distant rewards such as exercise, or dieting needs future research. The main implication of our work is that non-profits and governmental agencies ought to consider using abstract comic in their online campaign as they work to alleviate public goods dilemmas. We believe that these agencies can easily include the use of the comic form as part of their overall messaging strategy because the simplicity of the abstract comic form allows it to be algorithmically synthesized. 

As next steps, we plan to develop an algorithmic framework that automatically synthesizes text persuasive messages and maps a person's behavioral data (e.g., amount walked this week) to a three-panel persuasive comic. We also plan to conduct longitudinal field experiments with an emphasis on storytelling where individuals receive three-panel comics over time, and comics are connected with a storyline.

%The main implications of our work lie in how persuasive messages are delivered, especially to wearable devices. As next steps, we plan to develop an algorithmic framework that automatically maps a person's behavioral data (e.g. amount walked this week) to a three-panel persuasive comic. We also plan to conduct longitudinal field experiments with an emphasis on storytelling where individuals receive three-panel comics over time, and comics are connected with a storyline.


%Three ideas were key to our work: the role of abstract comics to allow readers to project themselves onto the comic and ``social proof'' (that we adopt the decisions from other donaters) .


%The study on charitable giving examined if the abstract comic form was more persuasive than text. We analyzed the results using a hierarchical Bayesian framework that allows for understanding effect sizes, as well are helpful in small-$n$ studies. The results shows convincingly that the three-panel abstract comic is more persuasive than the text (medium effect size=0.44). The presence of the social proof is also effective, but provides only a minor improvement to the use of the abstract comic (effect size =0.12; not a meaningful improvement). The main implications of our work lie in how persuasive messages are delivered, especially to wearable devices.

% comics have a significant but moderate effect (effect size: 0.33) on persuasiveness compared to text. While no comic element has significant effect, we can observe that non-neutral gestures and shading as well as negatively framed messages have a strong influence. We conducted a smaller study with color and the result indicates that color too has a strong influence. Finally, we developed an abstract comic panel generator that takes as input the different comic elements, the valence of the frame and the message. We plan to release the code for Bayesian analysis and comic generator and the raw data under an appropriate open source license.

