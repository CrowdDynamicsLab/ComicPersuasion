\subsection{Computational Persuasion}
With the advance in human technology and artificial intelligence, in very recent years, researcher starts to consider persuasion as an automated process where a computational model can algorithmically persuade target persuadee with effective persuasion strategies, e.g. engaging conversation with virtual agents\cite{huang2007design,nguyen2008designing,KangT15}. Huang and Lin proposed a virtual sales agent to persuade potential customers to offer a better price \cite{huang2007design}. By using machine-learning-based approach, an augment graph is trained based on simulated scenarios. Through a laboratory and online experiment, the results show this virtual sales agent can increase buyer's product valuation and willingness-to-pay efficiently. However, the persuasive power may be traded if such virtual agent provides inappropriate augmentation. Nguyen and Masthoff reviewed a number of argumentation-based systems and concludes comparing to the confrontational approach, arguments that based on social relationship and intrinsic motivation may be more efficacious \cite{nguyen2008designing}. Kang et. al developed a computational model called Model for Adaptive Persuasion that provides a unified framework for different persuasion strategies \cite{KangT15}. MAP is grounded in the Elaboration Likelihood Model that can select different persuasion strategies based on persuadee's feedback. With an evaluation of 26 elderly subjects, the result shows a MAP-based agents can change persuadee's attitude intentionally \cite{KangT15}. However, the existing framework cannot personalize for persuadee's profile and existing beliefs. And the framework can only accounts for a limited number of persuasive strategies.\par
However, four challenges have not been addressed in the current state of knowledge in building computational persuasion system, including insufficient domain knowledge, constrained persuasion protocols, unrepresentative persuadee models and lack of optimal persuasion strategies\cite{huntertowards}. An effective computational persuasion model requires sufficient formalize domain knowledge about persudee goals, persuade preference, and system action base etc to generate related persuasive arguments; an effective persuasion protocol that can best leverage the constrain of the meida and deliver generated persuasive messages efficiently; representative persuade model that allows the persuasion system to optimize a persuasion model algorithmically based on persuadee's beliefs and preferences; effective persuasion strategies that harness the perusadee model and produce optimal moves to persuade \cite{huntertowards}. \par
In our study, we focused on persuasion protocols and persuasion strategies. On one hand, we developed a persuasive framework can automatically generate persuasive messages in the comics form which provides a novel persuasion protocol that can be adapted in an computational persuasion system. Our comic-based persuasion protocols can produce individualized persuasive comics based on perusadee's model in terms of the text content, figure gestures and background shading. On the other hand, our experiment results suggest a unique persuasive strategy which leverage the text framing and visual comics representation. \par

Our study builds on comic's characteristic of attracting reader's attention. We want to further extend the research by understanding if comic can not only engage the readers, but also persuade them to take action. From the study by Haughney, it proposed a comic-style solution to present qualitative research findings. The study proves that the modified comic panel layout for user experience reports can make it more likely for readers to read the whole report through \cite{haughney2008using}. This study offers an innovative idea to present visual information through comic layout. In addition, text bubbles in the layout highlights additional information about the research and can better attract reader's attention compared to text label by the side of a report. However, this design is specifically for presenting visual information such as user interface feedbacks. It is difficult to apply it to other forms of information or research findings. In our study, we utilize comics to present persuasive text posts and extend the research to examine the effect in persuading behavioral change.\par

Our study, we leveraged both information-centric and behavior-centric approaches. The comic is designed to show user's behavior stats (information-centric) but the message is framed beyond simple fact to achieve maximum persuasiveness (behavior-centric). The challenge here is mitigating the downsides of information-centric and behavior-centric approaches. We used novel representation, visual comic, to catch information receiver's attention.


However, using comics in the context of persuasion is challenging. First, generating persuasive comics is not easy. Especially, persuasive messages are better to be personalized to deliver maximum persuasive power. Traditionally, comics are created by professional cartoonists which is very costly to produce personalized comics.  Although prior work has explored methods of algorithmically generating comics, no existing method is for generating persuasive comics.

Aside from the comparison between text and comics, research has also focused on comparison between other forms of data and their effect in persuading behavioral change. Lin \cite{lin2013impact} compared two models of presenting information about wind energy in brochure form: (1) photographs and (2) using cartoons as visual aids. To evaluate the effect, the research focused on comparing three measures: (1) audience's knowledge of, (2) attitudes toward, and (3) behavioral intentions regarding wind energy. Results show that there is no significant difference between using photographs or comics in terms of knowledge and attitudes. However, visual aids shown in the cartoon/comics version showed stronger behavioral intentions (e.g., greater willingness to support changes) than the photo group. Because of the abstractness of comics, it can better engage its readers in the messages compared to photographs, making readers more willing to adopt changes. In this study, it mainly focuses on different visual form's effect on persuading readers to agree on changes conducted by others.

However, this study retrieves the comics from the internet on a certain topic instead of customized generated comic. The drawback of this approach is that it is harder to control the quality and characteristic of the comic. Different authors of comic might have very different drawing style, resulting in differing effects on persuasion.

side from the comparison between text and comics, research has also focused on comparison between other forms of data and their effect in persuading behavioral change. Lin compared two models of presenting information about wind energy in brochure form: (1) photographs and (2) using cartoons as visual aids. To evaluate the effect, the research focused on comparing three measures: (1) audience's knowledge of, (2) attitudes toward, and (3) behavioral intentions regarding wind energy. Results show that there is no significant difference between using photographs or comics in terms of knowledge and attitudes. However, visual aids shown in the cartoon/comics version showed stronger behavioral intentions (e.g., greater willingness to support changes) than the photo group. Because of the abstractness of comics, it can better engage its readers in the messages compared to photographs, making readers more willing to adopt changes. In this study, it mainly focuses on comics' effect on persuading readers to agree on changes conducted by others. However, in our study, we focus on comic's effect on persuading readers themselves to adopt behavior change.\par

Our work builds on top of the findings of this study that comic can better attract reader's attention. We generate customized comic messages from text messages to unify the comic style and evaluate the persuading effect.

Previous research shows using visual representations are more attractive to readers comparing to text messages \cite{selker2015sweetbuildinggreeter,consolvo2008activity}. Selker et al. retrieves motivational images from 9GAG, Google, with energy-saving messages, to motivate people's energy saving behaviors and found those comics has higher persuasive power comparing to plain text \cite{selker2015sweetbuildinggreeter}. Consolvo et. al built Ubifit Garden that successfully encourage physical activity by present user's exercise data as different elements in a garden \cite{consolvo2008activity}.

Persuasive message in a visual form can raise persuadee's awareness and have a greater chance to persuade change. However, as a common visual form, no one has looked at the perceived persuasiveness of comics.

According to Scott McCloud, the reader is more likely to project him/herself onto the character in the comic when the comic getting more abstract \cite{scott1993understanding}. By taking the perspective of the character, the reader will internalize the information his/her character trying to express or receive. If the information is persuasive, the internalization will imply a higher chance of target behavior change. Therefore, in this study, we choose to use an abstract yet well-recognized comic style, the xkcd style created by Randall Munroe, in our generated persuasive comic messages \cite{munroe2009xkcd},see Figure~\ref{fig:xkcd}.


The word bubble is the most common place in comics to incorporate text information. In a persuasive comic, the word bubble expresses the text content of the message.
