%!TEX root = cscw2019-comic.tex


\begin{abstract}
This paper examines the use of the abstract comic form for making online charitable donations. Persuading individuals to contribute to charitable causes online is hard and responses to appeals are typically low; charitable donations share the structure of public goods dilemmas where the rewards are distant and non-exclusive. In this paper, we examine if comics in abstract form are more persuasive than the plain text. Drawing on a rich literature on comics, we algorithmically synthesized a three-panel abstract comic to create our appeal. We conducted a between-subject Amazon Mechanical Turk study with 307 participants on the use of abstract comic form to appeal for charitable donation. As part of our experimental procedure, we sought to persuade individuals to contribute to a real charity focused on Autism research with monetary cost. We compared the average amount of donation to the charity under three conditions: the plain text message, an abstract comic that includes the plain text, and an abstract comic that additionally includes the social norm. We use Bayesian modeling to analyze the results, motivated by model transparency and its use in small-sized studies. Our experiments reveal that the message in abstract comic form elicits more donations than text (medium to large effect size=0.59) significantly. Incorporating social proof in the abstract comic message did not show a significant effect. Our studies have design implications: non-profits and governmental agencies interested in alleviating public goods dilemmas that share a similar structure to our experiment (single-shot task, distant, non-exclusive reward) ought to consider including messages in the abstract comic form as part of their online fundraising campaign.
\end{abstract}