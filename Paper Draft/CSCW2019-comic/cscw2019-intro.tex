%!TEX root = cscw2019-comic.tex

\section{Introduction}
\label{sec:Introduction}
Are persuasive text messages more effective when expressed in comic form? Text messages abound, asking us to act, either towards personal wellness goals (e.g. in an exercise app:``you need 2,000 more steps to reach 10,000 steps for the day''), reminders (e.g. push notifications from a task tracking app: ``pick up prescription from the pharmacy at 6PM''), or appeals from charities online (e.g. from a recent Wikipedia campaign on the Wikipedia page: ``If everyone reading this donated \$5 our fundraiser would end today. Please donate to keep Wikipedia free.''). 

Making these short text messages more effective has important implications in behavior change.

There is a rich history of prior work in Psychology that has examined how variations in the construction of the text message alters decisions. In a seminal paper, tversky & kahneman, examined information framing: how we can elicit different decisions, by describing the outcome either as a loss or a gain, even though the two messages are indistinguishable in terms of expected utility. Goldstein and Cialdini examined the introduction of social norms in towel re-use in hotels, as well as their use in power consumption. In the former study, replacing the standard text message in hotels asking the customer to re-use towels, with the social norm (e.g. ``75\% of the people in this hotel have re-used their towels''), increased towel re-use. The latter study had a intriguing result---while simply introducing the social norm in the message did not reduce overall energy consumption, adding emoticons made a difference. Specifically, adding a smiley face when the customer consumed less energy than average, or adding a frowny face when their energy consumption was worse than average. In both cases, average energy consumption decreased. The Schultz study motivates us to examine the role of the comic form, a highly expressive, affective medium, in communicating messages intended to stimulate a specific behavior.

Despite the importance of the comic form in contemporary culture, and the use of the comic form in educational settings, we don't fully understand the effect of the comic form on persuasion. 

Summarize past work on the use of comics.


Introduce the abstract comic form; use the mccloud justification. But also point out that the abstract comic form because of how spare it is, allows for synthesis (though we need to research the effect of character gesture, inter-character spacing), consequently personalization of persuasive messages.

we use a three panel comic.

explain why it is not a slam dunk: on one hand, comic communicate affect, and are an important element of our visual culture; on the other, from rational actor theory in Economics, since the comic includes the text message used in the text-only condition and thus cannot alter the expected utility, the comic ought to serve as an irrelevant factor. 
 

There are many different behavioral contexts where we could examine the effects of the abstract comic form: personal wellness goals (e.g. diet, exercise), mundane tasks (e.g. "pick up dry cleaning), as well as broader collective action goals (e.g. ``take the flu shot;'' ``donate to cure cancer'')

Four design principles guide us, in our choice of the experimental context: nature of the reward; single shot tasks that preclude habit formation (a potential confound); an ecologically valid task; absence of specialized knowledge to perform the task. First, we would like the rewards to be distant, and non-exclusive, rather than proximal and exclusive so that individuals don't perform the task in anticipation of the immediate reward. Thus public goods dilemmas (e.g. ``reducing carbon footprint;'' ``taking the flu shot''; ``contributing to public knowledge'') are all candidates. Second, while some longitudinal tasks (e.g. losing weight; eating healthy) have distant rewards (losing weight, or maintaining a diet takes time), and can positively affect the public good (with more healthy people, in the long-run, insurance rates will fall), these tasks are prone to habit formation, a potential confound. Furthermore, single-shot tasks such as ``pick up yoghurt at the grocery store today'', often prompted by text reminders from our calendars or task-tracking apps, have an immediate, exclusive reward. Third, we would like to ensure that the experimental task is ecologically valid---a task that these individuals would be actually asked to perform in the wild, outside of the experimental context. Fourth, we would like the task to not require specialized knowledge (e.g. ``asking doctors to make a decision''), so that other researchers could easily replicate and scale our experiment. Online charitable donation tasks satisfy these design principles as they are single-shot tasks, contribute to the public good with distant, non-exclusive rewards, occurs frequently, and easily tested for at scale. We chose a charity associated with Autism (Organization for Autism Research) for our experiment.

Thus, in this work we answer two research questions in the context of the online charitable donation task:
\begin{description}
    \item[RQ1:] Does the use of the abstract comic form increase the level of donation over the plain text message?
    \item[RQ2:] What is the effect of introducing a social norm in the abstract comic, when compared to the comic without the social norm?
\end{description}

We ran the experiment using Amazon Mechanical Turk ($n=307$; we paid each participant \$8.0/hr) where we randomly assigned each participant to one of the three conditions ('a plain text message', 'abstract comic', 'abstract comic with social norm'). The two comic conditions include the text used in the plain text condition. The social norm case adds one additional phrase to the comic that reveals the norm.

We perform a careful Bayesian Analysis to analyze the data. We do so because replicability, transparency, small n. We emphasize that that non-Bayesian methods are powerful, and are highly effective in the hands of an experienced Statistician. 

Our findings: We show that using the comic has a meaningful increase in donations over the plain text conditions, with a medium to large effect size of $0.59$. Thus we can answer \textbf{RQ1} affirmatively. To answer \textbf{RQ2}, we show that while the comic with social norm increases the donation level over the comic without the norm, the effect size is very small ($0.11$) and the increase is not meaningful. To summarize, the comic form meaningfully increases donations over the plain text, but the presence of the norm is not effective. We caution that the result holds for single-shot, public goods tasks; specific longitudinal tasks such as exercise, or dieting may see a effect due to the social norm.  

Design Implications: The primary design implication of our findings: helping non-profits and Governmental agencies with their online messaging strategies as they work to alleviate public goods dilemmas. In particular, public-goods dilemmas that require single-shot decisions (online charitable donations; taking the flu shot) are opportune candidates for intervention. We believe that these agencies can easily include the use of the comic form as part of their overall messaging strategy, because the simplicity of the abstract comic form allows it to be easily synthesized (as we discuss towards of the end of this paper) and to additionally incorporate social norms. 








